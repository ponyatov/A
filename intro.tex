\section*{Введение}\addcontentsline{toc}{section}{Введение}

Первоначально этот материал задумывался как комплект документации к платам
BlackSwift и VoCore, но постепенно превратился в толстенный учебник для
студентов ВУЗов и научных работнков по специлизациям, связанным с применением
цифровой электроники и компьютерной техники.

Большой упор был сделан на использование открытого некоммерческого программного
обеспечения, с целью удешевления учебного процесса, уменьшения себестоимости
ваших проектов\note{вряд ли ли у вас окажется лишняя пачка килобаксов на покупку
пары коммерческих САПР, по крайней мере пока ваш стартап не взлетит в
Top\$100K}, и стимулирования вашего участия в развитии этих программных пакетов.

\bigskip
\textbf{Лицензия на эту книгу пока не выбрана, так что она пока просто пишется в
духе OpenSource: любой может использовать ее часть, изменять или дополнять, до
тех пор, пока не накладываются какие-либо административные, финансовые или
юридические ограничения на распространение и развитие оригинальной версии или ее
открытых форков.}
\bigskip

Приглашаем всех желающих участвовать в развитии этого учебного пособия на форум
\href{https://groups.google.com/forum/\#!forum/openwrt2ru}{ruOpenWrt}, нам нужна
обратная связь по качеству материала, результаты тестирования на вас или ваших
студентах, дополнения и замечания.
