\secrel{Введение}

Это учебное пособие было создано для интересующихся любительской электроникой,
самодельными цифровыми системами управления (Arduino, устройствами на
микроконтроллерах и т.п.), и программистов-любителей. В связи с полной
деградацией системы образования пособие также рекомедуется для применения при
обучении в ВУЗах по специализациям, связанным с применением цифровой электроники
и компьютерной техники.

Большой упор был сделан на использование открытого некоммерческого программного
обеспечения, с целью удешевления учебного процесса, уменьшения себестоимости
ваших проектов\note{вряд ли ли у вас окажется лишняя пачка килобаксов на покупку
пары коммерческих САПР, по крайней мере пока ваш стартап не взлетит в
Top\$100K}, и стимулирования вашего участия в развитии этих программных пакетов.

Книга очень объемна и разнообразна по материалу, в разделе \ref{learnplans}\
приведены примеры учебных планов с выборкой отдельных тем.

\bigskip
\textbf{Лицензия на эту книгу пока не выбрана, так что она пока просто пишется в
духе OpenSource: любой может использовать ее часть, изменять или дополнять, до
тех пор, пока не накладываются какие-либо административные, финансовые или
юридические ограничения на распространение и развитие оригинальной версии или ее
открытых форков: \url{https://github.com/ponyatov/Azbuka}}
\bigskip

Приглашаем всех желающих участвовать в развитии этого учебного пособия на форум
\href{https://groups.google.com/forum/\#!forum/openwrt2ru}{ruOpenWrt} и в группу
\url{http://vk.com/samarahackerspace}, нам нужна обратная связь по качеству
материала, результаты тестирования на вас или ваших студентах, дополнения и
замечания.
\bigskip

Мы признательны Bill Collis за разрешение использовать материалы его книги
<<\href{www.techideas.co.nz}{An Introduction to
Practical Electronics,
Microcontrollers and
Software Design}>>\cite{bcollis} в
русскоязычном варианте <<Азбуки>> (\ref{bcollis}), и конечно он вполне
заслуженно включен в основные соавторы этой книги.

\bigskip
Также были нагло скопипащены и переведены:
\begin{itemize}
  \item Joe Martin, Craig Libuse: Tabletop Machining \ref{tabletop}
  \item Andreas Fester: Electronic circuit simulation with gEDA and NG-Spice by
Example \ref{spice}
\end{itemize}
