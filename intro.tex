\secrel{Введение}

Это учебное пособие было создано для интересующихся любительской электроникой,
самодельными цифровыми системами управления (Arduino, устройствами на
микроконтроллерах и т.п.), и программистов-любителей. В связи с полной
деградацией системы образования пособие также рекомедуется для применения при
обучении в ВУЗах по специализациям, связанным с применением цифровой электроники
и компьютерной техники.

Большой упор был сделан на использование открытого некоммерческого программного
обеспечения, для удешевления учебного процесса, уменьшения себестоимости ваших
проектов\note{вряд ли ли у вас окажется лишняя пачка килобаксов на покупку пары
коммерческих САПР, по крайней мере пока ваш стартап не взлетит в Top\$100K}, и
стимулирования вашего участия в развитии этих программных пакетов.

Книга очень объемна и разнообразна по материалу, и построена как справочник с
группировкой материала по тематике. Для тех, кто только начинает, в разделе
\ref{learnplans}\ расписаны \termdef{пошаговые учебные планы}{учебный план} с
точки зрения параллельного изучения нескольких предметов с постепенным
усложнением\note{как это происходит при традиционном offline обучении}. Как
известно, главная часть любого обучения\ --- практическая. Особое внимание
уделено набору лабораторных работ.

\bigskip
Мы признательны Bill Collis за разрешение использовать материалы его книги
<<\href{www.techideas.co.nz}{An Introduction to
Practical Electronics,
Microcontrollers and
Software Design}>> \cite{bcollis} в
русскоязычном варианте <<Азбуки>> (\ref{bcollis}), и конечно он вполне
заслуженно включен в основные соавторы этой книги.

\bigskip
Так как для работы в области электроники необходимо владение технологиями
изготовления конструктива, в книгу включен соответствующий раздел. 
Эти книги рекомендуются популярным поставщиком хоббийных настольных
микро-станков \href{http://sherline.com/}{Sherline Products}. Так как от
владельцев авторских прав не получено разрешение на полный официальный перевод,
для этих книг сделан только перевод-подстрочник, который поможет вам читать
оригинал:
\begin{itemize}
  \item Joe Martin, Craig Libuse \textbf{Tabletop Machining}
  \cite{tabletop} (\ref{tabletop})
  \item Doug Briney \textbf{Home Machinists Handbook}
  \cite{briney} (\ref{briney})
\end{itemize}

Отечественных книг по использованию маленьких ``часовых'' и настольных станков
просто не существует, хотя они и выпускались серийно. Исключение\ --- книга
Евгений Васильев \textbf{Маленькие станки} \cite{vasil}, по согласованию в
автором включена как отдельный раздел (\ref{vasil}).

\bigskip
\textbf{Лицензия на эту книгу пока не выбрана, так что она пока просто пишется в
духе OpenSource: любой может использовать ее часть, изменять или дополнять, до
тех пор, пока не накладываются какие-либо административные, финансовые или
юридические ограничения на распространение и развитие оригинальной версии или ее
открытых форков: \url{https://github.com/ponyatov/Azbuka}}
\bigskip

Приглашаем всех желающих участвовать в развитии этого учебного пособия на форум
\href{https://groups.google.com/forum/\#!forum/openwrt2ru}{ruOpenWrt} и в группу
\url{http://vk.com/samarahackerspace}, нам нужна обратная связь по качеству
материала, результаты тестирования на вас или ваших студентах, дополнения и
замечания.
