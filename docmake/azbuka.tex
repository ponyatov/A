\input{Замечания для соавторов ``Абзуки ARMатурщика''}

% \part{Замечания для авторов}
% 
% \section{Набор репозиториев на GitHub}
% 
% \begin{tabular}{l l}
% 
% \url{https://github.com/ponyatov/Azbuka}
% & основная репа \\
% 
% \url{https://github.com/ponyatov/bib}
% & библиографические базы данных \\
% 
% \url{https://github.com/ponyatov/scratcher}
% & журнал, используются некоторые материлы \\
% 
% \end{tabular}
% \bigskip
% 
% Для работы с проектом сделайте собственный форк основной репы,
% библиографическую базу и журнал можете клонироввать напрямую.
% Создайте каталог и склонируйте репы:
% 
% \begin{lstlisting}
% D:
% cd \
% mkdir w
% cd \w\
% git clone --depth=1 -o gh git@github.com:username/Azbuka.git
% 
% git clone --depth=1 -o gh git@github.com:ponyatov/bib.git
% git clone --depth=1 -o gh git@github.com:ponyatov/scratcher.git
% \end{lstlisting}
% 
% \section{Верстка в \latex}
% 
% Было много вопросов по выбору языка разметки, и даже предложения некоторые
% материалы просто навордятить. Но все же используется \latex, т.к. это самая
% наворченная система подготовки больших изданий, широко распространенная (в
% узких кругах), и прежде всего как имеюющая богатейший набор пакетов-расширений
% для всевозможных вывертов.
% 
% \bigskip
% \latex\ не предназначен для верстки полноцвета, журнальной верстки или
% ручного таскания блоков по листу. \latex\ изначально был заточен на подготовку
% научно-технической и учебной многостраничной литературы \textbf{с логической
% разметкой}.
% 
% \bigskip
% \textbf{Как профессиональный инструмент, \latex\ требует обучения}. Начерно
% навордятить на нем текст, накидав как попало картинок, и наляпав
% шрифтов\note{про существование стилей многие вордятники даже не слышали} не
% получиться. И это хорошо.
% 
% \bigskip
% \textbf{Но\ --- материалы на добавление принимаются в любых форматах, группой
% авторов, способных их доварить до нужного качества. Единственное ограничение:
% наличие бесплатных средств просмотра на трех основных платформах: \win, Linux и
% MacOS, или в онлайне (Google Docs, M\$ облака, и прочие сетевые болота).}
% 
% \bigskip
% Также приветствуетсся использование различных более простых языков разметки
% (SPHINX, Wiki, DocBook, .md, \ldots). Для первоначального сбора и группировки
% материала они проще для освоения, чаще всего рендер-движки для этих языков
% заточены под веб-редактиврование в т.ч. групповое, и хорошо подходят для простой
% по оформлению документации на программные пакеты, или как сборник ссылок на
% другие ресурсы.
% 
% Для включения таких материалов в основную верстку несложно написать
% \TeX-транслятор (если нет сразу готового), который создаст .tex файлы нужного
% вида с минимумом ручной доработки.
