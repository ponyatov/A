\secrel{Загрузчик syslinux}\secdown

Самый простой и удобный загрузчик для i386-систем, ставится на флешку
из под Windows, работает с FAT-разделами, поддерживает загрузку с флешек, CDROM
и по сети.

\bigskip
\url{http://www.syslinux.org/}

\secrel{Закачка}

\file{.zip}\ с бинарной сборкой \progdef{syslinux}{syslinux}:

\url{https://www.kernel.org/pub/linux/utils/boot/syslinux/syslinux-6.03.zip}
\bigskip

\progdef{memtest86+}{memtest86+}\ --- полезная утилита для тестирования \ram:

\url{http://www.memtest.org/download/5.01/memtest86+-5.01.zip}
\bigskip

Если планируете устанавливать рабочую станцию для сборки az\linux\ с флешки,
нужно скачать полные или \emph{netinst}\ установочные \file{.iso}-образы:

\emph{Установочный образ Debian Linux} i386:

\url{http://cdimage.debian.org/debian-cd/7.7.0/i386/iso-cd/debian-7.7.0-i386-netinst.iso}

\emph{Установочный образ Debian Linux} amd64:

\url{http://cdimage.debian.org/debian-cd/7.7.0/amd64/iso-cd/debian-7.7.0-amd64-netinst.iso}

\emph{Сборка HDD-инсталлятора} i386:

\url{http://http.us.debian.org/debian/dists/wheezy/main/installer-i386/current/images/hd-media/initrd.gz}

\url{http://http.us.debian.org/debian/dists/wheezy/main/installer-i386/current/images/hd-media/vmlinuz}

\emph{Сборка HDD-инсталлятора} amd64:

\url{http://http.us.debian.org/debian/dists/wheezy/main/installer-amd64/current/images/hd-media/initrd.gz}

\url{http://http.us.debian.org/debian/dists/wheezy/main/installer-amd64/current/images/hd-media/vmlinuz}

\secrel{Установка под \win\ на флешку}

Распакуйте файлы из \file{.zip}
\bigskip

\begin{tabular}{l l}
\file{/bios/win32/syslinux.exe} & консольный
инсталлятор \\
\file{/bios/com32/menu/menu.c32} & модуль текстового меню \\
\file{/bios/com32/menu/vesamenu.c32} & модуль графического меню (VESA)\\
&\emph{служебные библиотеки \prog{syslinux}}\\
\file{/bios/com32/libutil/libutil.c32} &  \\
\file{/bios/com32/lib/libcom32.c32} &  \\
\end{tabular}
\bigskip

Для установки \prog{syslinux}\ на флешку с FAT, на которую назначена буква
\file{F:}, выполните батник:

\lst{/syslinux/syslinuxusb.bat}{}{linux/syslinuxusb.bat}

\begin{tabular}{l l l}
-i & install & установить \\
-m & MBR & в MBR \\
-a & active & сделать раздел активным \\
-d & directory & в каталог \file{syslinux}
\end{tabular}
\bigskip

Если ставите Debian, распакуйте из \file{debian-netinst.iso}
в \file{F:/Debian/}
\bigskip

\begin{tabular}{l l}
\file{debian-7.7.0-amd64-netinst.iso} & \file{.iso}-образ установочного CD-ROM\\
\file{debian-7.7.0-i386-netinst.iso} & \file{.iso}-образ установочного CD-ROM\\
\file{hd-media/install.amd/vmlinuz} & ядро amd64 (x64) \\
\file{hd-media/install.amd/initrd.gz} & ramdisk с инсталляром \\
\file{hd-media/install.386/vmlinuz} & ядро i386 (x32) \\
\file{hd-media/install.386/initrd.gz} & ramdisk с инсталляром \\
\end{tabular}
\bigskip

\secrel{\file{syslinux.cfg}}

\prog{syslinux}\ настривается текстовым файлом \file{syslinux.cfg}.

\lstx{/syslinux/syslinux.cfg}{}{linux/syslinux.cfg}{syslinux}

В примере показана реализация с использованием графического VESA меню.
Для использования более надежного текстового меню замените на
\verb|UI menu.c32|.

\emph{Обратите внимание на возможность включения нестандартных видеорежимов}
используя \file{[VESA]MENU RESOLUTION}: этот финт нужен для включения графики на
ASUS EeePC 701: режим 800$\times$480 недоступен для включения через параметр
ядра \file{vga=}, поэтому приходится использовать возможности \prog{syslinux}.

\secup
