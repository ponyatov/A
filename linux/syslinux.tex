\section{Загрузчик \prog{syslinux}}

Самый простой и удобный загрузчик для i386-систем, ставится на флешку
из под Windows, работает с FAT-разделами, поддерживает загрузку с флешек, CDROM
и по сети.

\bigskip
\url{http://www.syslinux.org/}

\subsection{Установка под \win\ на флешку}

\file{.zip}\ с бинарной сборкой:

\url{https://www.kernel.org/pub/linux/utils/boot/syslinux/syslinux-6.03.zip}
\bigskip

Распакуйте файлы из \file{.zip}
\bigskip

\begin{tabular}{l l}
\file{/bios/win32/syslinux.exe} & консольный
инсталлятор \\
\file{/bios/com32/menu/menu.c32} & модуль текстового меню \\
\file{/bios/com32/menu/vesamenu.c32} & модуль графического меню (VESA)\\
&\emph{служебные библиотеки \prog{syslinux}}\\
\file{/bios/com32/libutil/libutil.c32} &  \\
\file{/bios/com32/lib/libcom32.c32} &  \\
\end{tabular}

Для установки на флешку с FAT, на которую назначена буква
\file{F:}, выполните батник:

\lst{/syslinux/syslinuxusb.bat}{}{linux/syslinuxusb.bat}

\begin{tabular}{l l l}
-i & install & установить \\
-m & MBR & в MBR \\
-a & active & сделать раздел активным \\
-d & directory & в каталог \file{syslinux}
\end{tabular}

\subsection{\file{syslinux.cfg}}

\prog{syslinux}\ настривается текстовым файлом \file{syslinux.cfg}.

\lst{/syslinux/syslinux.cfg}{}{linux/syslinux.cfg}

В примере показа реализация с использованием графического VESA меню. 
Для использования более надежного текстового меню замените на 
\verb|UI menu.c32|.

