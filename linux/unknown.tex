\secrel{Применение \linux\ для новой ``железки''}\secdown

В последние несколько лет идет взрывной рост применения встраиваемых плат на
базе процессоров архитектур ARM и MIPS, в комплекте к которым идет та или иная
сборка em\linux. В большой степени это обусловлено массовым выпуском мобильных
устройств и роутеров SOHO-сегмента. С другой стороны, производители железа
учитывают высокую популярность \linux\ в среде разработчиков.

В этом разделе рассмотрим случай, с которым вы обязательно столкнетесь: \emph{в
какой-то момент вам потребуется самостоятельно запустить и освоить новую
``железку'', в комплекте к которой идет какая-нибудь сборка em\linux}.

В \ref{azlin}\ приведено полное описание системы сборки для создания пакета
кросс-компиляции. Но этот вариант (полной пересборки прошивки) подходит вам
только в случае, если вы полностью определяете состав ПО.

Очень часто требуется сохранить прошивку и сборку \linux-системы от поставщика
или производителя оборудования, т.к. в нее могут входить части, поставляемые в
бинарном виде, vendor-specific патчи на ядро, и подобное закрытое ПО в виде так
называемых ``бинарных бл\'{о}бов''. Чаще всего этим страдают мобильные
аппаратные платформы, в т.ч. Raspberry Pi: производитель
SoC\note{[S]ystem-[o]n-[C]hip: процессор и периферия, иногда даже RAM и Flash,
объединенные в одном корпусе} предоставляет часть драйверов в виде закрытого
firmware. В клиническом случае предоставляется ядро собственной модификации и
набор утилит для управления железом только в бинарном виде без возможности
пересборки пользователем.

Другой типичный случай: вам нужно запустить на железке всего 1-2 ваших
программы. При этом \termdef{вендорная сборка}{вендорная сборка} \linux\
устройства слишком громоздка, чтобы полностью ее пересобирать: X Window, полный
мультимедийный комплект ПО и библиотек, тяжелый браузер,\ldots Или вы хотите
сохранить возможность установки сторонних пакетов и регулярного обновления
какой-нибудь популярной сборки, например Raspbian.

\bigskip

В качестве примера разберем добавление конфигурации для двух железок:

\begin{enumerate}
  \item VoCore \ref{vocore}
  \item ТионПро270 \ref{tion}
\end{enumerate}

\secup
