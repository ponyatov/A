Linux для встраиваемых систем\note{будем называть его
\termdef{em\linux}{em\linux}}\ --- популярный метод быстрого создания комплекса
ПО для больших сложных приложений, работающих на достаточно мощном железе,
особенно предполагающих интенсивное использование сетевых технологий.

За счет использования уже существующей и очень большой базы исходных текстов
ядра, библиотек и программ для \linux, бесплатно доступных в т.ч. и для
коммерческих приложений, можно на порядки сократить стоимость разработки
собственных программных компонентов, и при этом получить готовую команду
бесплатных стронних разработчиков, уже знакомых с созданием ПО для \linux.

Из недостатков можно отметить:
\begin{itemize}
  \item Отсутствие полноценной поддержки режима жесткого реального времени;
  \item Тяжелое ядро;
  \begin{itemize}
  \item Поддерживаются только мощные семейства процессоров\note{32-бит,
  необходим блок MMU};
  \item Значительные требования по объему \ram\ и общей производительности;
  \end{itemize}
  \item Дремучесть техспециалистов, контуженных ТурбоПаскалем и
Win\-dows\-ом;
\end{itemize}

Для \termdef{сборки}{сборка}\ em\linux-системы используется метод
\termdef{кросс-компиляции}{кросс-компиляция}, когда используется
\termdef{кросс-тулчейн}{кросс-тулчейн}, компилирующий весь комплект ПО для
компьютера с другой архитектурой. Типичный пример\ --- сборка ПО на ПК c
процессором Intel i7 для Raspberry Pi или планшета на процессоре
AllWinner/Tegra/\ldots.

em\linux\ очень широко применяется на рынке мобильных устройств\note{в т.ч.
является основой Android}, и устройств интенсивно использующих сетевые протоколы
(роутеры, медиацентры).

В качествe примера применения рассмотрим относительно простое приложение:
многофункциональные настенные часы с синхронизацией времени через \internet, с
будильником, медиапроигрывателем, блэкджеком и плюшками.
