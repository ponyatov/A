\secrel{Setting up SDL and Getting an Image on the Screen}

Since SDL is a third party library, you have to install it yourself. Here you'll
get a step by step guide on how to set it up.

If you have any problems try consulting the
\href{http://www.libsdl.org/faq.php}{SDL Development FAQ}.

Once you set up SDL, you can move on the second half of the tutorial and learn
to load and show images on the screen.

\secdown
\secrel{\win}

An important note for Visual Studio users: The latest version of SDL for visual
studio comes with two sets of library and binary files: x86 for 32bit, x64 for
64bit. If you're on a 64bit operating system, Visual Studio still compiles in
32bit by default. When you set the library directory, it should point to the x86
folder inside of the lib folder.

\bigskip
Установка для следующих сред разработки описана
\href{http://lazyfoo.net/SDL_tutorials/lesson01/windows/index.php}{здесь}:

\begin{enumerate}
  \item Dev C++ 4.9.9.2
  \item Code::Blocks 8.02
  \item MinGW Developer Studio 2.05
  \item Eclipse 3.1
  \item Command Line (MinGW)
  \item Visual Studio.NET 2010 Express
  \item Visual Studio.NET 2005/2008 Express
  \item Visual Studio.NET 2003
\end{enumerate}

\secrel{\linux}

Пакеты ставятся из вашего дистрибутива. 

Для включения в az\linux\ добавьте пакет \pack{sdl}\ в переменную
конфигурирования \verb|LIBS += sdl|. Сборка пакета описана в \ref{azsdl}.

\secrel{Getting an Image on the Screen}

This tutorial covers how to do Hello World SDL style.

Now that you have SDL set up, it's time to make a bare bones graphics
application that loads and displays an image on the screen.

\begin{verbatim}
// Include SDL functions and datatypes
#include "SDL/SDL.h"
\end{verbatim}

At the top of the source file we include the SDL header file so we can use the
SDL functions and data types.

Remember that some of you (like Visual Studio users) are going to include SDL
like this:

\begin{verbatim}
#include "SDL.h"
\end{verbatim}

So if the compiler is complaining that it can't find "SDL/SDL.h", then it's
either because you're including the wrong path or you forgot to put SDL.h in the
right place.

\begin{verbatim}
int main( int argc, char* args[] )
{
    //The images
    SDL_Surface* hello = NULL;
    SDL_Surface* screen = NULL;
\end{verbatim}

At the top of the main() function, two SDL\_Surface pointers are declared. An
SDL\_Surface is an image, and in this application we're going to be dealing with
two images. The surface "hello" is the image we're going to be loading and
showing. The "screen" is what is visible on the screen.

Whenever you're dealing with pointers, you should always remember to initialize
them.

Also, when using SDL, you must have your main() function declared like it is
above. You can't use void main() or anything like that.

\begin{verbatim}
    //Start SDL
    SDL_Init( SDL_INIT_EVERYTHING );

    //Set up screen
    screen = SDL_SetVideoMode( 640, 480, 32, SDL_SWSURFACE );

    //Load image
    hello = SDL_LoadBMP( "hello.bmp" );
\end{verbatim}

The first function we call in the main() function is SDL\_Init(). This call of
SDL\_Init() initializes all the SDL subsystems so we can start using SDL's
graphics functions.

Next SDL\_SetVideoMode() is called to set up a 640 pixel wide, 480 pixel high
window that has 32 bits per pixel. The last argument (SDL\_SWSURFACE) sets up
the surface in software memory. After SDL\_SetVideoMode() executes, it returns a
pointer to the window surface so we can use it.

After the window is set up, we load our image using SDL\_LoadBMP().
SDL\_LoadBMP() takes in a path to a bitmap file as an argument and returns a
pointer to the loaded SDL\_Surface. This function returns NULL if there was an
error in loading the image.

\begin{verbatim}
    //Apply image to screen
    SDL_BlitSurface( hello, NULL, screen, NULL );

    //Update Screen
    SDL_Flip( screen );

    //Pause
    SDL_Delay( 2000 );
\end{verbatim}

Now that we have our window set up and our image loaded, we want to apply the
loaded image onto the screen. We do this using SDL\_BlitSurface(). The first of
SDL\_BlitSurface() argument is the source surface. The third argument is the
destination surface. SDL\_BlitSurface() sticks the source surface onto the
destination surface. In this case, it's going to apply our loaded image onto the
screen. You'll find out what the other arguments do in later tutorials.

Now that our image is applied to screen, we need to update the screen so we can
see it. We do this using SDL\_Flip(). If you don't call SDL\_Flip(), you'll only
see an unupdated blank screen.

Now that the image is applied to the screen and it's made visible, we have to
make the window stay visible so it doesn't just flash for a split second and
disappear. We'll make the window stay by calling SDL\_Delay(). Here we delay the
window for 2000 milliseconds (2 seconds). You'll learn a better way to make the
window stay in place in tutorial 4.

\begin{verbatim}
    //Free the loaded image
    SDL_FreeSurface( hello );

    //Quit SDL
    SDL_Quit();

    return 0;
}
\end{verbatim}

Now that we're not going to use the loaded image anymore in our program, we need
to remove it from memory. You can't just use delete, you have to use
SDL\_FreeSurface() to remove the image from memory. At the end of our program,
we call SDL\_Quit() to shut down SDL.

You may be wondering why we never deleted the screen surface. Don't worry,
SDL\_Quit() cleans it up for you.

Congratulations, you've just made your first graphics application.

\secrel{Troubleshooting your SDL application}

If the compiler complains that it can't find 'SDL/SDL.h', it means you forgot to
set up your header files. Your compiler/IDE should be looking for the SDL header
files, so make sure that it's configured to look in the SDL include folder.

If you're using Visual Studio and the compiler complains 'SDL/SDL.h': No such
file or directory, go to the top of the source code and make sure it says
\verb|#include "SDL.h"|.

If your program compiles, but linker complains it can't find some lib files,
make sure your compiler/IDE is looking in the SDL lib folder. If your linker
complains about an undefined references to a bunch of SDL functions, make sure
you linked against SDL in the linker.

If your linker complains about entry points, make sure that you have the main
function declared the right way and that you only have one main function
combined in your source code.

If the program compiles, links, and builds, but when you try to run it it
complains that it can't find SDL.dll, make sure you put SDL.dll in the same
directory as the compiled executable. Visual Studio users will need to put the
dll file in the same directory as your vcproj file. Windows users can also put
the dll inside of the system32 directory.

If you run the program and the images don't show up or the window flashes for a
second and you find in stderr.txt:

\begin{verbatim}
Fatal signal: Segmentation Fault (SDL Parachute Deployed)
\end{verbatim}

It's because the program tried to access memory it wasn't supposed to. Odds are
its because it tried to access NULL when SDL\_BlitSurface() was called. This
means you need to make sure the bitmap files are in the same directory as the
program. Visual Studio users will need to put the bitmap file in the same
directory as your vcproj file.

Also if you're using Visual Studio and you get the error "The application failed
to start because the application configuration is incorrect. Reinstalling the
application may fix this problem.", it's because you don't have the service pack
update installed. Do not forget to have the latest version of your compiler/IDE
with the service pack update for your compiler/IDE or SDL will not work with
Visual Studio.

Some Linux users will run and get a blank screen. Try running the program from
the command line.

If you had to create a project to compile an SDL program, remember that you will
need to create a project for every SDL program you create. Or, better yet, you
can reuse the SDL project you made the first time.
Download the media and source code for this tutorial here.

\secup
