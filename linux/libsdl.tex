\chapter{Библиотека libSDL}\secdown

Библиотека \href{http://www.libsdl.org/}{SDL} предоставляет такие средства, как
быстрый вывод 2D-графики, обработку ввода, проигрывание звука, вывод 3D через
OpenGL и другие операции, причем делает это кросплатформенно. Список платформ
обширный: Linux, Windows, Windows CE, BeOS, Mac OS X, FreeBSD, NetBSD, OpenBSD,
BSD/OS, Solaris, IRIX и QNX\ --- и вдобавок есть неофициальные порты на другие
системы.

Сама библиотека написана на Си и поддерживает \cpp, однако есть биндинги к
большинству популярных языков. Автор \prog{libsdl} был нанят компанией Valve,
программные продукты которой активно используют библиотеку. К тому же, теперь
библиотека выходит под лицензией zlib, а не LGPLv2, как было раньше, и SDL 2.0
можно использовать в любых своих приложениях, в т.ч. коммерческих Скорее всего
сделано это было для того, чтобы Valve смогла включить ее в Steam для Linux.

\bigskip
Использование SDL позволяет писать графические и мультимедийные приложения для
em\linux, не включая в систему достаточно тяжелую X Window System.

\bigskip
В книге рассмотрена версия 1.2.15, последняя версия из ветки \prog{SDL1}, т.к.
она используется в сборке az\linux\ \ref{azlin}. Сейчаc активно развивается
ветка \prog{SDL2}, ее особенности кратко рассмотрены в разделе \ref{sdl2}.

\secrel{Инициализация и завершение SDL-программы}

\begin{verbatim}
int main() {
	...
	SDL_Init(SDL_INIT_xxx));
	...
	SDL_Quit();
}
\end{verbatim}

\begin{description}
\item[SDL\_INIT\_EVERYTHING] все подсистемы, не работает с отключенным звуком
\item[SDL\_INIT\_VIDEO] только графическая подсистема
\end{description}

\secrel{LazyFoo tutorial}\secdown

\cp{http://lazyfoo.net/SDL\_tutorials/}

\secrel{Setting up SDL and Getting an Image on the Screen}

Since SDL is a third party library, you have to install it yourself. Here you'll
get a step by step guide on how to set it up.

If you have any problems try consulting the
\href{http://www.libsdl.org/faq.php}{SDL Development FAQ}.

Once you set up SDL, you can move on the second half of the tutorial and learn
to load and show images on the screen.

\secdown
\secrel{\win}

An important note for Visual Studio users: The latest version of SDL for visual
studio comes with two sets of library and binary files: x86 for 32bit, x64 for
64bit. If you're on a 64bit operating system, Visual Studio still compiles in
32bit by default. When you set the library directory, it should point to the x86
folder inside of the lib folder.

\bigskip
Установка для следующих сред разработки описана
\href{http://lazyfoo.net/SDL_tutorials/lesson01/windows/index.php}{здесь}:

\begin{enumerate}
  \item Dev C++ 4.9.9.2
  \item Code::Blocks 8.02
  \item MinGW Developer Studio 2.05
  \item Eclipse 3.1
  \item Command Line (MinGW)
  \item Visual Studio.NET 2010 Express
  \item Visual Studio.NET 2005/2008 Express
  \item Visual Studio.NET 2003
\end{enumerate}

\secrel{\linux}

Пакеты ставятся из вашего дистрибутива. 

Для включения в az\linux\ добавьте пакет \pack{sdl}\ в переменную
конфигурирования \verb|LIBS += sdl|. Сборка пакета описана в \ref{azsdl}.

\secrel{Getting an Image on the Screen}

This tutorial covers how to do Hello World SDL style.

Now that you have SDL set up, it's time to make a bare bones graphics
application that loads and displays an image on the screen.

\begin{verbatim}
// Include SDL functions and datatypes
#include "SDL/SDL.h"
\end{verbatim}

At the top of the source file we include the SDL header file so we can use the
SDL functions and data types.

Remember that some of you (like Visual Studio users) are going to include SDL
like this:

\begin{verbatim}
#include "SDL.h"
\end{verbatim}

So if the compiler is complaining that it can't find "SDL/SDL.h", then it's
either because you're including the wrong path or you forgot to put SDL.h in the
right place.

\begin{verbatim}
int main( int argc, char* args[] )
{
    //The images
    SDL_Surface* hello = NULL;
    SDL_Surface* screen = NULL;
\end{verbatim}

At the top of the main() function, two SDL\_Surface pointers are declared. An
SDL\_Surface is an image, and in this application we're going to be dealing with
two images. The surface "hello" is the image we're going to be loading and
showing. The "screen" is what is visible on the screen.

Whenever you're dealing with pointers, you should always remember to initialize
them.

Also, when using SDL, you must have your main() function declared like it is
above. You can't use void main() or anything like that.

\begin{verbatim}
    //Start SDL
    SDL_Init( SDL_INIT_EVERYTHING );

    //Set up screen
    screen = SDL_SetVideoMode( 640, 480, 32, SDL_SWSURFACE );

    //Load image
    hello = SDL_LoadBMP( "hello.bmp" );
\end{verbatim}

The first function we call in the main() function is SDL\_Init(). This call of
SDL\_Init() initializes all the SDL subsystems so we can start using SDL's
graphics functions.

Next SDL\_SetVideoMode() is called to set up a 640 pixel wide, 480 pixel high
window that has 32 bits per pixel. The last argument (SDL\_SWSURFACE) sets up
the surface in software memory. After SDL\_SetVideoMode() executes, it returns a
pointer to the window surface so we can use it.

After the window is set up, we load our image using SDL\_LoadBMP().
SDL\_LoadBMP() takes in a path to a bitmap file as an argument and returns a
pointer to the loaded SDL\_Surface. This function returns NULL if there was an
error in loading the image.

\begin{verbatim}
    //Apply image to screen
    SDL_BlitSurface( hello, NULL, screen, NULL );

    //Update Screen
    SDL_Flip( screen );

    //Pause
    SDL_Delay( 2000 );
\end{verbatim}

Now that we have our window set up and our image loaded, we want to apply the
loaded image onto the screen. We do this using SDL\_BlitSurface(). The first of
SDL\_BlitSurface() argument is the source surface. The third argument is the
destination surface. SDL\_BlitSurface() sticks the source surface onto the
destination surface. In this case, it's going to apply our loaded image onto the
screen. You'll find out what the other arguments do in later tutorials.

Now that our image is applied to screen, we need to update the screen so we can
see it. We do this using SDL\_Flip(). If you don't call SDL\_Flip(), you'll only
see an unupdated blank screen.

Now that the image is applied to the screen and it's made visible, we have to
make the window stay visible so it doesn't just flash for a split second and
disappear. We'll make the window stay by calling SDL\_Delay(). Here we delay the
window for 2000 milliseconds (2 seconds). You'll learn a better way to make the
window stay in place in tutorial 4.

\begin{verbatim}
    //Free the loaded image
    SDL_FreeSurface( hello );

    //Quit SDL
    SDL_Quit();

    return 0;
}
\end{verbatim}

Now that we're not going to use the loaded image anymore in our program, we need
to remove it from memory. You can't just use delete, you have to use
SDL\_FreeSurface() to remove the image from memory. At the end of our program,
we call SDL\_Quit() to shut down SDL.

You may be wondering why we never deleted the screen surface. Don't worry,
SDL\_Quit() cleans it up for you.

Congratulations, you've just made your first graphics application.

\secrel{Troubleshooting your SDL application}

If the compiler complains that it can't find 'SDL/SDL.h', it means you forgot to
set up your header files. Your compiler/IDE should be looking for the SDL header
files, so make sure that it's configured to look in the SDL include folder.

If you're using Visual Studio and the compiler complains 'SDL/SDL.h': No such
file or directory, go to the top of the source code and make sure it says
\verb|#include "SDL.h"|.

If your program compiles, but linker complains it can't find some lib files,
make sure your compiler/IDE is looking in the SDL lib folder. If your linker
complains about an undefined references to a bunch of SDL functions, make sure
you linked against SDL in the linker.

If your linker complains about entry points, make sure that you have the main
function declared the right way and that you only have one main function
combined in your source code.

If the program compiles, links, and builds, but when you try to run it it
complains that it can't find SDL.dll, make sure you put SDL.dll in the same
directory as the compiled executable. Visual Studio users will need to put the
dll file in the same directory as your vcproj file. Windows users can also put
the dll inside of the system32 directory.

If you run the program and the images don't show up or the window flashes for a
second and you find in stderr.txt:

\begin{verbatim}
Fatal signal: Segmentation Fault (SDL Parachute Deployed)
\end{verbatim}

It's because the program tried to access memory it wasn't supposed to. Odds are
its because it tried to access NULL when SDL\_BlitSurface() was called. This
means you need to make sure the bitmap files are in the same directory as the
program. Visual Studio users will need to put the bitmap file in the same
directory as your vcproj file.

Also if you're using Visual Studio and you get the error "The application failed
to start because the application configuration is incorrect. Reinstalling the
application may fix this problem.", it's because you don't have the service pack
update installed. Do not forget to have the latest version of your compiler/IDE
with the service pack update for your compiler/IDE or SDL will not work with
Visual Studio.

Some Linux users will run and get a blank screen. Try running the program from
the command line.

If you had to create a project to compile an SDL program, remember that you will
need to create a project for every SDL program you create. Or, better yet, you
can reuse the SDL project you made the first time.
Download the media and source code for this tutorial here.

\secup

\secrel{Optimized Surface Loading and Blitting}

Now that you got an image on the screen in part 2 of the last tutorial, it's
time do your surface loading and blitting in a more efficient way.

\begin{verbatim}
//The headers
#include "SDL/SDL.h"
#include <string>
\end{verbatim}

Here are our headers for this program.

SDL.h is included because obviously we're going to need SDL's functions.

The string header is used because ... eh I just like 
\verb|std::string over char*|

\begin{verbatim}
//The attributes of the screen
const int SCREEN_WIDTH = 640;
const int SCREEN_HEIGHT = 480;
const int SCREEN_BPP = 32;
\end{verbatim}

Here we have the various attributes of the screen.

I'm pretty sure you can figure out what SCREEN\_WIDTH and SCREEN\_HEIGHT are.
SCREEN\_BPP is the bits per-pixel. In all of the tutorials, 32-bit color will be
used.

\begin{verbatim}
//The surfaces that will be used
SDL_Surface *message = NULL;
SDL_Surface *background = NULL;
SDL_Surface *screen = NULL;
\end{verbatim}

These are the three images that are going to be used.

"background"\ is obviously going to be the background image, "message"\ is the
bitmap that says "Hello"\ and "screen"\ is obviously the screen.

Remember: its a good idea to always set your pointers to NULL if they're not
pointing to anything.

\begin{verbatim}
SDL_Surface *load_image( std::string filename )
{
    //Temporary storage for the image that's loaded
    SDL_Surface* loadedImage = NULL;

    //The optimized image that will be used
    SDL_Surface* optimizedImage = NULL;
\end{verbatim}

Here we have our image loading function.

What this function does is load the image, then returns a pointer to the
optimized version of the loaded image.

The argument "filename"\ is the path of the image to be loaded. "loadedImage"\
is the surface we get when the image is loaded. "optimizedImage"\ is the surface
that is going to be used.

\begin{verbatim}
    //Load the image
    loadedImage = SDL_LoadBMP( filename.c_str() );
\end{verbatim}

First the image is loaded using SDL\_LoadBMP().

But it shouldn't be used immediately because the bitmap is 24-bit. The screen is
32-bit and it's not a good idea to blit a surface onto another surface that is a
different format because SDL will have to change the format on the fly which
causes slow down.

\begin{verbatim}
    //If nothing went wrong in loading the image
    if( loadedImage != NULL )
    {
        //Create an optimized image
        optimizedImage = SDL_DisplayFormat( loadedImage );

        //Free the old image
        SDL_FreeSurface( loadedImage );
    }
\end{verbatim}

Next we check if the image was loaded properly. If there was an error,
loadedImage will be NULL.

If the image loaded fine, SDL\_DisplayFormat() is called which creates a new
version of "loadedImage"\ in the same format as the screen. The reason we do
this is because when you try to stick one surface onto another one of a different
format, SDL converts the surface so they're the same format.

Creating the converted surface every time you blit wastes processing power which
costs you speed. Because we convert the surface when we load it, when you want
to apply the surface to the screen, the surface is already the same format as
the screen. Now SDL won't have to convert it on the fly.

So now we have 2 surfaces, the old loaded image and the new optimized image.

SDL\_DisplayFormat() created a new optimized surface but didn't get rid of the
old one.

So we call SDL\_FreeSurface() to get rid of the old loaded image.

\begin{verbatim}
    //Return the optimized image
    return optimizedImage;
}
\end{verbatim}

Then the newly made optimized version of the loaded image is returned.

\begin{verbatim}
void apply_surface( int x, int y, SDL_Surface* source, SDL_Surface* destination )
{
    //Make a temporary rectangle to hold the offsets
    SDL_Rect offset;

    //Give the offsets to the rectangle
    offset.x = x;
    offset.y = y;
\end{verbatim}

Here we have our surface blitting function.

It takes in the coordinates of where you want to blit the surface, the surface
you're going to blit and the surface you're going to blit it to.

First we take the offsets and put them inside an SDL\_Rect. We do this because
SDL\_BlitSurface() only accepts the offsets inside of an SDL\_Rect.

An SDL\_Rect is a data type that represents a rectangle. It has four members
representing the X and Y offsets, the width and the height of a rectangle. Here
we're only concerned about x and y data members.

\begin{verbatim}
    //Blit the surface
    SDL_BlitSurface( source, NULL, destination, &offset );
}
\end{verbatim}

Now we actually blit the surface using SDL\_BlitSurface().

The first argument is the surface we're using.

Don't worry about the second argument, we'll just set it to NULL for now.

The third argument is the surface we're going to blit on to.

The fourth argument holds the offsets to where on the destination the source is
going to be applied.

\begin{verbatim}
int main( int argc, char* args[] )
{
\end{verbatim}

Now we start the main function.

When using SDL, you should always use:

\begin{verbatim}
int main( int argc, char* args[] )
\end{verbatim}

or

\begin{verbatim}
int main( int argc, char** args )
\end{verbatim}

Using int main(), void main(), or any other kind won't work.

\begin{verbatim}
    //Initialize all SDL subsystems
    if( SDL_Init( SDL_INIT_EVERYTHING ) == -1 )
    {
        return 1;
    }
\end{verbatim}

Here we start up SDL using SDL\_Init().

We give SDL\_Init() SDL\_INIT\_EVERYTHING, which starts up every SDL subsystem.
SDL subsystems are things like the video, audio, timers, etc that are the
individual engine components used to make a game.

We're not going to use every subsystem but it's not going to hurt us if they're
initialized anyway.

If SDL can't initialize, it returns -1. In this case we handle the error by
returning 1, which will end the program.

\begin{verbatim}
    //Set up the screen
    screen = SDL_SetVideoMode( SCREEN_WIDTH, SCREEN_HEIGHT, SCREEN_BPP, SDL_SWSURFACE );
\end{verbatim}

Now it's time to make our window and get a pointer to the window's surface so
we can blit images to the screen.

You already know what the first 3 arguments do. The fourth argument creates the
screen surface in system memory.

\begin{verbatim}
    //If there was an error in setting up the screen
    if( screen == NULL )
    {
        return 1;
    }
\end{verbatim}

If there was a problem in making the screen pop up, screen will be set to NULL.

\begin{verbatim}
    //Set the window caption
    SDL_WM_SetCaption( "Hello World", NULL );
\end{verbatim}

Here the caption is set to "Hello World".

The caption is this part of the window:

\begin{verbatim}
    //Load the images
    message = load_image( "hello.bmp" );
    background = load_image( "background.bmp" );
\end{verbatim}

Now the images are loaded using the image loading function we made.

\begin{verbatim}
    //Apply the background to the screen
    apply_surface( 0, 0, background, screen );
\end{verbatim}

Now it's time to apply the background with the function we made.

Before we blitted the background, the screen looked like this:

But now that we blitted the background image, the screen looks like this in
memory:

When you blit, you copy the pixels from one surface onto another. So now the
screen has our background image in the top left corner, but we want to fill up
the entire screen. Does that mean we have to load the background image 3 more
times?

\begin{verbatim}
    apply_surface( 320, 0, background, screen );
    apply_surface( 0, 240, background, screen );
    apply_surface( 320, 240, background, screen );
\end{verbatim}

Nope. We can just blit the same surface 3 more times.

\begin{verbatim}
    //Apply the message to the screen
    apply_surface( 180, 140, message, screen );
\end{verbatim}

Now we're going to apply the message surface onto the screen at x offset 180 and
y offset 140.

The thing is SDL coordinate system doesn't work like this:

SDL's coordinate system works like this:

So the origin (0,0) is at the top left corner instead of the bottom left.

So when you blit the message surface, it's going to blit it 180 pixels right,
and 140 pixels down from the origin in the top left corner:

SDL's coordinate system is awkward at first but you'll get used to it.

\begin{verbatim}
    //Update the screen
    if( SDL_Flip( screen ) == -1 )
    {
        return 1;
    }
\end{verbatim}

Even though we have applied our surfaces, the screen we see is still blank.

Now we have to update the screen using SDL\_Flip() so that the screen surface we
have in memory matches the one shown on the screen.

If there's an error it will return -1.

\begin{verbatim}
    //Wait 2 seconds
    SDL_Delay( 2000 );
\end{verbatim}

We call SDL\_Delay() so that the window doesn't just flash on the screen for a
split second. SDL\_Delay() accepts time in milliseconds, or 1/1000 of a second.

So the window will stay up for 2000/1000 of a second or 2 seconds.

\begin{verbatim}
    //Free the surfaces
    SDL_FreeSurface( message );
    SDL_FreeSurface( background );

    //Quit SDL
    SDL_Quit();

    //Return
    return 0;
}
\end{verbatim}

Now we do the end of the program clean up.

SDL\_FreeSurface() is used to get rid of the surfaces we loaded since we're not
using them anymore. If we don't free the memory we used, we will cause a memory
leak.

Then SDL\_Quit() is called to quit SDL. Then we return 0, ending the program.

You may be asking yourself "why aren't we freeing the screen surface?". Don't
worry. SDL\_Quit() will take care of that for us. If you run the program and the
images don't show up or the window flashes for a second and you find in
stderr.txt:

\begin{verbatim}
Fatal signal: Segmentation Fault (SDL Parachute Deployed)
\end{verbatim}

It's because the program tried to access memory it wasn't supposed to. Odds are
it's because it tried to access NULL when apply\_surface() was called. This
means you need to make sure the bitmap files are in the same directory as the
program.

If the window pops up and the image doesn't show up, again make sure the bitmaps
are in the same folder as the program or in the project directory.

If you're using Visual Studio and the compiler complains about 'SDL/SDL.h': No
such file or directory, go to the top of the source code and make sure it says
\verb|#include "SDL.h"|.

Also if you're using Visual Studio and you get the error "The application failed
to start because the application configuration is incorrect. Reinstalling the
application may fix this problem.", it's because you don't have the service pack
update installed. Do not forget to have the latest version of your compiler/IDE
with the service pack update for your compiler/IDE or SDL will not work with
Visual Studio.

\bigskip
Download the media and source code for this tutorial here.

\secrel{Extension Libraries and Loading Other Image Formats}

SDL only supports .bmp files natively, but using the SDL\_image extension
library, you'll be able to load BMP, PNM, XPM, LBM, PCX, GIF, JPEG, TGA and PNG
files.

Extension libraries allow you to use features that basic SDL doesn't support
natively. Setting up an SDL extension library isn't hard at all, I'd even say
it's easier to set up than basic SDL. This tutorial will teach you how to use
the SDL\_image extension library.

\secdown
\secrel{\win}

\secrel{\linux}
\secup

\secrel{Event Driven Programming}

Up until this point you're probably used to command driven programs using cin
and cout. This tutorial will teach you how to check for events and handle
events.

An event is simply something that happens. It could be a key press, movement of
the mouse, resizing the window or in this case when the user wants to X out the
window.

\begin{verbatim}
//The headers
#include "SDL/SDL.h"
#include "SDL/SDL_image.h"
#include <string>

//Screen attributes
const int SCREEN_WIDTH = 640;
const int SCREEN_HEIGHT = 480;
const int SCREEN_BPP = 32;

//The surfaces
SDL_Surface *image = NULL;
SDL_Surface *screen = NULL;
\end{verbatim}

Here we have the same story as before, we have our headers, constants and
surfaces.

\begin{verbatim}
//The event structure that will be used
SDL_Event event;
\end{verbatim}

Now this is new. A SDL\_Event structure stores event data for us to handle.

\begin{verbatim}
SDL_Surface *load_image( std::string filename ) 
{
    //The image that's loaded
    SDL_Surface* loadedImage = NULL;
    
    //The optimized image that will be used
    SDL_Surface* optimizedImage = NULL;
    
    //Load the image
    loadedImage = IMG_Load( filename.c_str() );
    
    //If the image loaded
    if( loadedImage != NULL )
    {
        //Create an optimized image
        optimizedImage = SDL_DisplayFormat( loadedImage );
        
        //Free the old image
        SDL_FreeSurface( loadedImage );
    }
    
    //Return the optimized image
    return optimizedImage;
}

void apply_surface( int x, int y, SDL_Surface* source, SDL_Surface* destination )
{
    //Temporary rectangle to hold the offsets
    SDL_Rect offset;
    
    //Get the offsets
    offset.x = x;
    offset.y = y;
    
    //Blit the surface
    SDL_BlitSurface( source, NULL, destination, &offset );
}
\end{verbatim}

Here we have our surface loading and blitting functions. Nothing has changed
from the previous tutorial.

\begin{verbatim}
bool init()
{
    //Initialize all SDL subsystems
    if( SDL_Init( SDL_INIT_EVERYTHING ) == -1 )
    {
        return false;
    }

    //Set up the screen
    screen = SDL_SetVideoMode( SCREEN_WIDTH, SCREEN_HEIGHT, SCREEN_BPP, SDL_SWSURFACE );

    //If there was an error in setting up the screen
    if( screen == NULL )
    {
        return false;
    }

    //Set the window caption
    SDL_WM_SetCaption( "Event test", NULL );

    //If everything initialized fine
    return true;
}
\end{verbatim}

Here is the initialization function. This function starts up SDL, sets up the
window, sets the caption and returns false if there are any errors.

\begin{verbatim}
bool load_files()
{
    //Load the image
    image = load_image( "x.png" );

    //If there was an error in loading the image
    if( image == NULL )
    {
        return false;
    }

    //If everything loaded fine
    return true;
}
\end{verbatim}

Here is the file loading function. It loads the images, and returns false if
there were any problems.

\begin{verbatim}
void clean_up()
{
    //Free the image
    SDL_FreeSurface( image );

    //Quit SDL
    SDL_Quit();
}
\end{verbatim}

Here we have the end of the program clean up function. It frees up the surface
and quits SDL.

\begin{verbatim}
% int main( int argc, char* args[] )
% {
%     //Make sure the program waits for a quit
%     bool quit = false;
\end{verbatim}

Now we enter the main function.

Here we have the quit variable which keeps track of whether the user wants to
quit. Since the program just started we set it to false or the program will end
immediately.

\begin{verbatim}
%     //Initialize
%     if( init() == false )
%     {
%         return 1;    
%     }
%     
%     //Load the files
%     if( load_files() == false )
%     {
%         return 1;    
%     }
\end{verbatim}

Now we call the initialization and file loading functions we made earlier.

\begin{verbatim}
%     //Apply the surface to the screen
%     apply_surface( 0, 0, image, screen );
%     
%     //Update the screen
%     if( SDL_Flip( screen ) == -1 )
%     {
%         return 1;    
%     }
\end{verbatim}

Then we show the image on the screen.

\begin{verbatim}
%     //While the user hasn't quit
%     while( quit == false )
%     {
\end{verbatim}

Now we start the main loop. This loop will keep going until the user sets quit
to true.

\begin{verbatim}
%         //While there's an event to handle
%         while( SDL_PollEvent( &event ) )
%         {
\end{verbatim}

In SDL whenever an event happens, it is put on the event queue. The event queue
holds the event data for every event that happens.

So if you were to press a mouse button, move the mouse around, then press a
keyboard key, the event queue would look something like this:

What SDL\_PollEvent() does is take an event from the queue and sticks its data
in our event structure:

What this code does is keep getting event data while there's events on the
queue.

\begin{verbatim}
%             //If the user has Xed out the window
%             if( event.type == SDL_QUIT )
%             {
%                 //Quit the program
%                 quit = true;
%             }    
%         }
%     }
\end{verbatim}

When the user Xs out the window, the event type will be SDL\_QUIT.

But when the user does that it does not end the program, all it does inform us
the user wants to exit the program.

Since we want the program to end when the user Xs the window, we set quit to
true and it will break the loop we are in.

\begin{verbatim}
%     //Free the surface and quit SDL
%     clean_up();
%         
%     return 0;    
% }
\end{verbatim}

Finally, we do the end of the program clean up.

There are other ways to handle events like SDL\_WaitEvent() and
SDL\_PeepEvents(). You can find out more about them in the SDL documentation.
Download the media and source code for this tutorial here.

On a side note, now would also be a good time to learn to use the SDL error
functions. I don't have a tutorial on them, but I touch on them in article 5.
The SDL documentation should explain SDL\_GetError(), and the SDL\_image
documentation should explain IMG\_GetError(). SDL\_ttf and SDL\_mixer also have
their error functions so remember to look those up in their documentations.

\input{linux/sdl/lf05}
\secrel{Clip Blitting and Sprite Sheets}

\input{linux/sdl/lf07}
\input{linux/sdl/lf08}
\input{linux/sdl/lf09}
\input{linux/sdl/lf10}
\input{linux/sdl/lf11}
\input{linux/sdl/lf12}
\input{linux/sdl/lf13}
\input{linux/sdl/lf14}
\input{linux/sdl/lf15}
\input{linux/sdl/lf16}
\input{linux/sdl/lf17}
\secrel{Per-pixel Collision Detection}

\input{linux/sdl/lf19}
\input{linux/sdl/lf20}
\input{linux/sdl/lf21}
\input{linux/sdl/lf22}
\input{linux/sdl/lf23}
\input{linux/sdl/lf24}
\input{linux/sdl/lf25}
\secrel{Resizable Windows and Window Events}

\input{linux/sdl/lf27}
\input{linux/sdl/lf28}
\input{linux/sdl/lf29}
\input{linux/sdl/lf30}
\secrel{Pixel Manipulation and Surface Flipping}

\input{linux/sdl/lf32}
\input{linux/sdl/lf33}
\input{linux/sdl/lf34}
\input{linux/sdl/lf35}
\secrel{Using OpenGL with SDL}


\secup

\secrel{Примеры программ}\secdown

\clearpage\secrel{Графический Hello World}

\lstx{user/sdl\_hello.c}{}{azlin/user/sdl_hello.c}{c}

\clearpage\secrel{Вывод случайных прямоугольников}

\lstx{user/sdl\_rect.c}{}{azlin/user/sdl_rect.c}{c}

\secup

\secrel{Версия SDL2}\label{sdl2}

\secup
