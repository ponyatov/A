\chapter{Библиотека libSDL}\secdown

Библиотека \href{http://www.libsdl.org/}{SDL} предоставляет такие средства, как
быстрый вывод 2D-графики, обработку ввода, проигрывание звука, вывод 3D через
OpenGL и другие операции, причем делает это кросплатформенно. Список платформ
обширный: Linux, Windows, Windows CE, BeOS, Mac OS X, FreeBSD, NetBSD, OpenBSD,
BSD/OS, Solaris, IRIX и QNX\ --- и вдобавок есть неофициальные порты на другие
системы.

Сама библиотека написана на Си и поддерживает \cpp, однако есть биндинги к
большинству популярных языков. Автор \prog{libsdl} был нанят компанией Valve,
программные продукты которой активно используют библиотеку. К тому же, теперь
библиотека выходит под лицензией zlib, а не LGPLv2, как было раньше, и SDL 2.0
можно использовать в любых своих приложениях, в т.ч. коммерческих Скорее всего
сделано это было для того, чтобы Valve смогла включить ее в Steam для Linux.

\bigskip
Использование SDL позволяет писать графические и мультимедийные приложения для
em\linux, не включая в систему достаточно тяжелую X Window System.

\bigskip
В книге рассмотрена версия 1.2.15, последняя версия из ветки \prog{SDL1}, т.к.
она используется в сборке az\linux\ \ref{azlin}. Сейчаc активно развивается
ветка \prog{SDL2}, ее особенности кратко рассмотрены в разделе \ref{sdl2}.

\secrel{Инициализация и завершение SDL-программы}

\begin{verbatim}
int main() {
	...
	SDL_Init(SDL_INIT_xxx));
	...
	SDL_Quit();
}
\end{verbatim}

\begin{description}
\item[SDL\_INIT\_EVERYTHING] все подсистемы, не работает с отключенным звуком
\item[SDL\_INIT\_VIDEO] только графическая подсистема
\end{description}

\secrel{Примеры программ}

\lstx{user/sdl\_hello.c}{}{azlin/user/sdl_hello.c}{c}

\lstx{user/sdl\_rect.c}{}{azlin/user/sdl_rect.c}{c}

\secrel{Версия SDL2}\label{sdl2}

\secup
