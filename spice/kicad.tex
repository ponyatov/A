
\secrel{Настройка KiCAD для SPICE-моделирования}\secdown\label{kispice}

\secrel{Библиотеки компонентов со SPICE-элементами}

\begin{itemize}
\item Библиотека базовых SPICE-компонентов поствляется с KiCAD. Эта
библиотека\ --- хороший вариант для начальных экспериментов.
Библиотека не подключена по умолчанию, вы должны сделать это вручную сами через
пенеджер библиотек. На Debian Linux это файл\\
\file{/usr/share/kicad/library/pspice.lib}\note{PSpice\ --- популярная
коммерческая версия SPICE}

\item Mithat Konar \email{webs@mithatkonar.com} разрабатывает (очень медленно)
\href{https://bitbucket.org/mithat/kicad-spice-library}{собственую
библиотеку} с некоторыми модификациями.

\item В комплекте с этой книгой поставляются библиотеки, адаптированные для
SPICE.

\end{itemize}

\secrel{Настройка проекта}

\begin{enumerate}
\item Создайте новый проект как обычно.
\item Откройте \prog{Eeschema} и удалите все библиотеки, подключаемые по
умолчанию.
\item Вручную добавьте одную из SPICE-библиотек, или набор библиотек для
этой книги. ОБратите внимание, что SPICE-библитека из поставки \prog{KiCAD}
по умолчанию не подключается к пректу.
\item Укажите расчетный SPICE-движок, который вы хотите использовать:

\menu{\prog{eeschema}>Меню>Инструменты>Сформировать список цепей>Spice}

\menu{\checkbox\ Формат по умолчанию}

\menu{\uncheckbox\ Префикс обозначений}

\menu{\checkbox\ Использовать имена цепей}

\menu{вкладка Spice>Команда симулятора:>\prog{xterm -e ngspice}}

\menu{Список цепей}

\end{enumerate}

\secrel{Как это работает}

\begin{enumerate}
  \item
Укажите режимы сиуляции, которые вы хотите выполнить, и генерацию вывода,
который хотите отобразить, добавив на схему текстовый блок (т.е.
``комментарий'') c необходимыми директивами в синтаксисе
\href{http://newton.ex.ac.uk/teaching/cdhw/Electronics2/userguide/sec5.html}{SPICE и Nutmeg}
с некоторыми добавками. Например, для выполнения \term{расчета по постоянному
току} и вывода сигнала в точке \verb|vout|, добавьте блок:
\begin{lstlisting}
+PSPICE
.control
ac dec 66 1kHz 120kHz
plot vdb(vout)
set units = degrees
plot vp(vout)
.endc
\end{lstlisting}
\begin{itemize}
  \item
Первая строка ``+PSPICE ''\ указывает kicadу добавить текст \emph{в конец}
сформированного \file{.cir}-файла. \emph{В текущей версии KiCAD есть баг,
который требует обязательного пробела после +SPICE}.
  \item
Соответственно строка ``-PSPICE ''\ добавляет текст \emph{в начало} .cir-файла.
  \item
Для поборников OpenSource, не желающих видеть ссылка на коммерческий PSpice,
предусмотрены директивы-синонимы $\pm$``GNUCAP ''. Я думаю это то же самое что и
$\pm$``PSPICE ``, но не уверен на 100\%, проверьте в документации.
  \item
Да, вам потребуется немного изучить синтакис
\href{http://newton.ex.ac.uk/teaching/cdhw/Electronics2/userguide/sec5.html}{SPICE
and Nutmeg}. Это нетрудно.
\end{itemize}
  \item
Запустите симуляцию:

\menu{\prog{eeschema}>Меню>Инструменты>Сформировать список цепей>Spice}

\menu{Запустить симулятор}

\end{enumerate}

\secup
