\chapter{FreeCAD}

\includegraphics[height=0.5\textheight]{logo/FreeCAD.png}

\cp{https://ru.wikipedia.org/wiki/FreeCAD_(Juergen_Riegel\%27s)}

В среде специалистов ряда отраслей известна проблема создания полноценной САПР в
рамках OpenSource, и хотя FreeCAD ещё не является кандидатом на такую
«полноценность», этот продукт может рассматриваться как одна из попыток создания
базы для решения этой проблемы. Разработчик FreeCAD Юрген Ригель, работающий в
корпорации DaimlerChrysler, позиционирует свою программу как первый бесплатный
инструмент проектирования механики (сравнивая свой продукт с такими развитыми
проприетарными системами как CATIA версий 4 и 5, SolidWorks), созданный на
основе библиотеки \textbf{Open CASCADE}. Цель программы\ --- предоставить
базовый инструментарий этой библиотеки в интерактивном режиме.

Следует отметить, что имеет место ещё один программный продукт имеющий название
freeCAD, его разработчик\ --- Aik-Siong Koh, и он не связан с FreeCAD’ом Юргена
Ригеля.

\section{Установка под \win}

\menu{\winr>\url{http://www.freecadweb.org/}>Download>\win>\href{http://sourceforge.net/projects/free-cad/files/FreeCAD Windows/FreeCAD 0.14/}{\file{FreeCAD
0.14}}>\file{\ldots\_setup.exe}}

\section{Чертеж}

\section{Эскиз}

\section{Деталь}

\section{Сборка}

\section{Автогенерация конструкторской докуметации}

\section{Скрипты и пользовательские расширения}

