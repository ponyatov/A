\secrel{USB}\secdown

Многообразные порты, унаследованные от компьютеров IBM PC и PS/2, уходят в
прошлое. Будущее, да и настоящее, принадлежит универсальным скоростным портам
USB. Об удобствах, которые USB предоставляет простым пользователям,
распространяться не приходится. Единый интерфейс для всех устройств, обладающий
возможностями Plug’n’Play и продвинутого управления питанием\ --- именно то, что
нужно пользователям, для которых компьютер\ --- часть бытовой техники.
Другое дело\ --- индивидуальные разработчики различных устройств и просто
хакеры. Для этих категорий переход на USB представляет определенные сложности.
Проблема заключается в том, что USB\ --- интеллектуальный интерфейс.

Любое устройство, предназначенное для подключения к компьютеру через USB, должно
поддерживать хотя бы небольшую часть спецификации протокола USB: уметь
«представиться»\note{предоставить информацию о себе и своих возможностях} и
адекватно реагировать на стандартные сообщения USB, посылаемые компьютером. В
результате, даже устройство, все функции которого ограничиваются включением и
выключением светодиода по сигналу с компьютера, при подключении через USB
требует наличия микросхемы, которая умеет «разговаривать» с хостом.

Однако и для
разработчиков собственных устройств переход на USB несет определенные
преимущества. Прежде всего, упрощается процесс написания драйверов. Поскольку
для общения с компьютером все USB устройства используют единый протокол, причем
протокол этот абстрагирован от таких аппаратно-зависимых вещей как отображенные
в память порты и прерывания, возникает возможность не писать свой собственный
драйвер уровня ядра для каждого устройства. Вместо этого целые группы устройств
могут использовать один и тот же драйвер уровня ядра, а специфичный код,
учитывающий особенности конкретного устройства, может быть размещен на
пользовательском уровне. При этом драйвер уровня ядра берет на себя такие
функции как управление питанием устройства\note{весьма нетривиальная задача с
учетом того, что сам компьютер может переключаться между несколькими
энергосберегающими режимами}, оставляя нам самое интересное\ --- управление
функциями устройства.

Перенос кода управления устройством в \term{userspace} не только упрощает
отладку\note{при падении приложения, скорее всего, не придется перезагружать
машину}, но и позволят писать процедуры управления устройством на самых разных
языках программирования, а не только на Си, как это делают те, кто пишет
драйверы уровня ядра. Более того, пользовательская часть драйвера, не
взаимодействующая напрямую с механизмами ядра операционной системы, может быть
сделана кросс-платформенной, что мы и имеем в случае таких инструментов как
\prog{libusb}. Благодаря \prog{libusb} даже многие устройства промышленного
уровня могут обходиться без собственных драйверов на уровне ядра для каждой ОС,
и иметь одну кодовую базу, которую проще модифицировать и сопровождать.
Что уж говорить о любительских устройствах?

\secrel{Стек протоколов USB}

Протокол USB похож на стек сетевых протоколов, основанных на TCP/IP, точнее за
основу принималась сетевая модель стека сетевых протоколов OSI/ISO\note{ISO/IEC
7498-1, ГОСТ Р ИСО/МЭК 7498-1-99}:

\bigskip\noindent
\begin{tabular}{l l l l p{0.43\textwidth}}
& Уровень & Layer & Тип данных & Функции \\
\hline
L1 & Физический & Physical & Биты & Работа со средой передачи, сигналами и
двоичными данными \\
L2 & Канальный & Data link & Кадры & Физическая адресация \\
L3 & Сетевой & Network & Пакеты/Датаграммы & Определение маршрута и логическая
аресация \\
L4 & Транспортный & Transport & Сегменты & Прямая связь между конечными пунктами
и надежность \\
L5 & Сеансовый & Session & Сеансы & Управление сеансом связи \\
L6 & Представления & Presentation & Поток & Представление и шифрование данных \\
L7 & Прикладной & Application & Данные & Доступ к сетевым службам \\ 
\end{tabular}


На самом нижнем логическом уровне\note{спецификации физического уровня мы пока
не рассматриваем} устройства обмениваются пакетами данных\note{со встроенными
механизмами коррекции ошибок, подтверждения получения и т.д}. Из пакетов
формируются запросы, которые устройства посылают друг другу. Запросы составляют
блоки запросов [U]SB [R]equest [B]lock, \termdef{URB}{USB!URB}.

Протокол USB является ``хостоцентричным"\ --- процесс передачи данных всегда
инициируется хостом (то есть, компьютером). Если у периферийного устройства
появились данные для передачи хосту, оно должно ожидать запроса хоста на
передачу данных. Соответственно мы имеем следующие типы USB устройтсв:

\begin{description}
  \item[host] \termdef{хост-контроллер}{USB!хост-контроллер} выполянет функции
  головного узла: инициирует передачу данных, управляет питанием
  \item[client] 
  \item[OTG] \termdef{[O]n-[T]he-[G]o}{USB!On-The-Go}, \termdef{OTG}{USB!OTG}\
  --- расширение спецификации USB 2.0, предназначенное для лёгкого соединения
  периферийных USB-устройств друг с другом без необходимости подключения к
  хосту. Например, цифровой фотоаппарат можно подключать к фотопринтеру
  напрямую, если они оба поддерживают стандарт USB OTG. К моделям КПК и
  коммуникаторов, поддерживающих USB OTG, можно подключать некоторые
  USB-устройства. Обычно это флэш-накопители, цифровые фотоаппараты, клавиатуры,
  мыши и другие устройства, не требующие дополнительных драйверов.
  При подключении через USB OTG ранг устройства (ведущий или ведомый)
  определяется наличием или, соответственно, отсутствием перемычки между
  контактами 4 (ID) и 5 (Ground) в штекере соединительного кабеля. В USB OTG
  кабеле такая перемычка устанавливается лишь в одном из двух разъёмов.
\end{description}

\secrel{libUSB}\secdown

\secrel{драйвер для устройства USB}

\cp{http://symmetrica.net/usb/usb1.htm}



\secup
\secrel{Поддержка USB в \linux}\secdown
\secrel{Опции ядра}\secdown
\secrel{режимы host/client/otg и хост-контроллеры xHCI}
\secrel{data storage: носители данных}
\secrel{hid: клавиатура, мышь, джойстик}
\secrel{USB-периферия: сеть, звук,\dots}
\secup
\secrel{Настройка hotplug и автомонтирования USB носителей}
\secrel{Сборка и настройка libusb}
\secrel{Примеры программ низкоуровневого ввода/вывода}
\secup


\secup
