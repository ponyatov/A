% % 3.4 The Board Editor . 33

\secrel{The Board Editor}

% The board editor is opened using a button in the toolbar, find this button and answer yes to the
% question about creating the board.
% The new window has a pile of parts and an area upon which to place them.
% WARNING: once you have started to create a board always have both the board and
% schematic open at the same time, never work on one without the other open or you will get
% horrible errors which will require you to delete the .brd file and restart the board from
% scratch.
% 
% 33
% 
% 3.4.1 Airwires
% The wires from the schematic have become connections called airwires, these wires will shortly
% become tracks on the PCB.
% These connections can look very messy at times and at this stage it is called a RATSNEST.
% 
% 3.4.2 Moving Components
% 
% Move the components into the highlighted area. In the demo version you cannot place parts
% outside this area. Keep the components in the lower left corner near the origin (cross).
% Reduce the size of the highlighted area you are using for the components. Then zoom to fit.
% Progressively arrange the components so that there is the minimum number of crossovers.
% As you place components press the Ratsnest button often to reorganize the Airwires. Eventually
% your picture will look like the one on the right.
% Good PCB design is more about placement of components than routing, so spending most
% of your time (80%) doing this step is crucial to success.
% You want to make track lengths as short as possible
% 
% 34
% 
% 3.4.3 Hiding/Showing Layers
% The DISPLAY button in the TOOLBOX is used to turn on and off different sets of screen
% information. Turn off the names, and values while you are placing components. This will keep the
% screen easier to read. Turn off the layer by selecting the display button and in the popup window
% pressing the number of the layer you no longer want to see.
% 
% 35
% 
% 3.4.4 Routing Tracks
% Now is the time to replace the airwires with actual PCB
% tracks. Tracks need to connect all the correct parts of the
% circuit together without connecting together other parts.
% This means that tracks cannot go over the top of one
% another, nor can they go through the middle of
% components!
% Go to the Toolbar, Select the ROUTE button
% On the Toolbar make sure the Bottom layer is selected
% (blue) and that the track width is 0.04. Left click on a
% component.
% Note that around your circuit all of the pads on the same
% net will be highlighted. Route the track by moving the
% mouse and left clicking on corner points for your track as
% you go. YOU ONLY WANT TO CONNECT THE PADS
% ON THE SAME NET, DON'T CONNECT ANY OTHERS
% OR YOUR CIRCUIT WILL NOT WORK. Double click on a
% pad to finish laying down the track.
% 
% Track layout Rules
% 1.
% 
% 2.
% 3.
% 
% Place tracks so that no track touches the leg of
% a component that it is not connected to on the
% schematic
% No track may touch another track that it is not
% connected to on the schematic
% Tracks may go underneath the body of a
% component as long as they meet the above
% rules
% 
% 3.4.5 Ripping up Tracks
% Ripping up a track is removing the track you have laid
% down and putting the airwire back in place. This will be
% necessary as you go to solve problems where it is not
% possible to route the tracks. You may even want to rip up
% all the tracks and move components around as you go.

% 36
