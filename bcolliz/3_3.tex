
\secrel{The Schematic Editor}\secdown

The first part of the process in creating a PCB is drawing the schematic.

\begin{enumerate}
  \item 
Parts will be added from libraries
  \item 
and joined together using ‘nets’ to make the circuit
\end{enumerate}

\secrel{The Toolbox}

As you point to the tools in the TOOLBOX their names will appear in a popup and
also their description will appear in the status bar at the bottom of the window

Find the following tools:

\begin{itemize}
  \item 
ADD A PART
  \item 
MOVE AN OBJECT
  \item 
DELETE AN OBJECT
  \item 
DEFINE THE NAME OF AN OBJECT
  \item 
DEFINE THE VALUE OF AN OBJECT
  \item 
DRAW NETS (connections)
  \item 
ERC (electrical rule check)
\end{itemize}
 
\secrel{Using parts libraries}

Selecting parts libraries to use.

Parts are stored within libraries and there are a large number of libraries in
Eagle.

It is not hard to create your own library and modify the parts within it. The
cls.lbr has many already modified components within it. If Eagle is not setup to
use the cls library you will need to do it now.

\begin{enumerate}
\item From your internet browser save the file cls.lbr into your Eagle folder.
\item In Eagle's control panel from the menu select options then directories
\item In the new window that appears make sure the directories for the libraries
 are highlighted
\item Click on browse and find your Eagle.directory
\item Next highlight the directories for Projects
\item Click on browse and find your Eagle directory again.
\item Choose OK.
\item You might need to close EAGLE and restart it to make sure it reads the
 libraries ok.
\item To use a library right click on it from within the Control Panel
\item Make sure Use is highlighted. It will have a green dot next to it if it is
 selected
\item At this time right click on the other lbr folder and select Use none.
\end{enumerate}

NOTE THE IMPORTANCE OF THE GREEN DOT NEXT TO THE LIBRARY,

if its not there you will not see the library in the schematic editor!

\secrel{Using Components from within libraries}

From your schematic Click the ADD button in the toolbox

A new window will open (it may take a while)

\begin{itemize}
\item Find the CLS library
\item Open it by double clicking on it or by
\item clicking the + sign
\item Open the R-EU\_ section (ResistorEuropean)
\item Here you will find the 0204/10 resistor.
\item Select it and then click OK
\end{itemize}

Add 2 more resistors of the same type.

Add all of the following parts

\begin{tabular}{l l l}

 LIBRARY
& PART
& Qty
\\
cls
&REU-0204/10
&3
\\
cls
&LDR
&1
\\
cls
&2,54/0,8 (wirepads)
&2
\\
cls
&led 5MM
&1
\\
cls
&1N4148 D41-10
&1
\\
cls
&2N7000
&1
\\
cls
&GND
&3
\end{tabular}

A wirepad allows us to connect wires to the PCB (such as wires to switches and
batteries)

\secrel{Different component packages}

There are several different types of resistors; they all have the same symbol
however resistors come in different physicalpackages so we must choose an
appropriate one.The 0204/7 is suitable for us but any of the 4 smallest ones
would be OK.

\secrel{Moving parts}

Move the parts around within the schematic editor so that they are arranged as
per the schematic below. Keep the component identifiers (numbers like R1, R2,
R3) in the same places as those below.

\secrel{Wiring parts together}

These form the electrical connections that makeup the circuit. Select the net
button from the toolbox.

\bigskip
Left click on the very end of a component and draw in a straight line either up,
down, left or right.

\bigskip
Left click again to stop at a point and draw before drawing in another
direction.

\bigskip
Double left click at another component to finish the wire.

\secrel{Zoom Controls}

There are a number of zoom controls that can be used to help you work in your
circuit.

Find these on the toolbar and identify what each does.

Nets

Nets are the wire connections between the components, each has a unique name.

Find the info button in the toolbox and check the names and details of the
components and nets/wires.

When you want to connect a new net to an existing net, Eagle will prompt you as
to which name to give the combined net.

If one of the nets has a proper name i.e. VCC, V+,V-, ground... use that name,
otherwise choose the net with the smallest number

\secrel{Junctions}

Junctions are the dots at joins in the circuit, they are there to make sure that
the wires are electrically connected. Generally you will NOT need to add these
to your circuit as the net tool puts them in place automatically

\secrel{ERC}

The ERC button causes Eagle to test the schematic for electrical errors.

Errors such as pins overlapping, and components unconnected are very common.

The ERC gives a position on the circuit as to where the error is; often zooming
in on that point and moving components around will help identify the error.

\emph{You must correct all errors before going on}.

\secup
