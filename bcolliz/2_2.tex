\secrel{Определение сопротивления резистора по цветовому коду}

Когда берете резистор, проверьте его \termdef{номинал}{номинал} (значение). В
наших схемах каждый резистор имеет свою цель, и значение выбирается в
зависимости от того, хотим ли мы б\'{о}льший или меньший ток в этой части цепи.
Чем выше \term{номинал} резистора, тем меньше ток. Чем ниже номинал резистора,
тем выше ток. Для \termdef{выводн\'{ы}х}{выводн\'{о}й корпус} резисторов,
которые мы будем использовать на макетках, маркировка наносится на корпус в
виде набора цветных полосок:

\bigskip
Цветовая маркировка на 5 цветных полосок:
\begin{tabular}{|l|l|l|l|l|}
\hline
 цифра & цифра & цифра & множитель & точность \\
\hline
\end{tabular}

\bigskip Цифра:
\begin{tabular}{l l l l l l l l l l}
0&1&2&3&4&5&6&7&8&9\\
\textcolor{Black}{$\blacksquare$} &
\textcolor{Brown}{$\blacksquare$} &
\textcolor{Red}{$\blacksquare$} &
\textcolor{Orange}{$\blacksquare$} &
\textcolor{Yellow}{$\blacksquare$} &
\textcolor{Green}{$\blacksquare$} &
\textcolor{Blue}{$\blacksquare$} &
\textcolor{Magenta}{$\blacksquare$} &
\textcolor{Grey}{$\blacksquare$} &
$\square$ \\
\end{tabular}

\bigskip Множитель:
\begin{tabular}{l l l l l l l l l l}
\textcolor{Black}{$\blacksquare$} & $10^{0}=1$ &
\textcolor{Brown}{$\blacksquare$} & $10^{1}=10$ &
\textcolor{Red}{$\blacksquare$} & $10^{2}=100$ &
\textcolor{Orange}{$\blacksquare$} & $10^{3}=1000$ \\
\textcolor{Yellow}{$\blacksquare$} & $10^{4}=10000$ &
\textcolor{Green}{$\blacksquare$} & $10^{5}=100000$ &
\textcolor{Blue}{$\blacksquare$} & $10^{6}=1000000$ \\
\textcolor{Gold}{$\blacksquare$} & $10^{-1}=0.1$ &
\textcolor{Silver}{$\blacksquare$} & $10^{-2}=0.01$ \\
\end{tabular}

\bigskip Точность:
\begin{tabular}{l l l l l l l l l l l}
\textcolor{Brown}{$\blacksquare$} & $\pm 1\%$ &
\textcolor{Red}{$\blacksquare$} & $\pm 2\%$ &
\textcolor{Gold}{$\blacksquare$} & $\pm 5\%$ &
\textcolor{Silver}{$\blacksquare$} & $\pm 10\%$ \\
\end{tabular}

\bigskip
\begin{tabular}{l l l l l l l l l l l l l}
-[&
\textcolor{Brown}{$\blacksquare$}&
\textcolor{Black}{$\blacksquare$}&
\textcolor{Black}{$\blacksquare$}&
\textcolor{Yellow}{$\blacksquare$}&
\textcolor{Brown}{$\blacksquare$}&
]- 
& $100\times 10^{4}\pm 1\%$ & 1M & 1 миллион Ом & 1M $\Omega$ & 1 000 000 Ом \\
&1&0&0&4&1\\
-[&
\textcolor{Brown}{$\blacksquare$}&
\textcolor{Black}{$\blacksquare$}&
\textcolor{Black}{$\blacksquare$}&
\textcolor{Red}{$\blacksquare$}&
\textcolor{Red}{$\blacksquare$}&
]- 
& $100\times 10^{2}\pm 2\%$ & 10k & 10 тысяч ом & 10,000 Ом & 10k $\Omega$ \\
&1&0&0&2&2\\
-[&
\textcolor{Brown}{$\blacksquare$}&
\textcolor{Black}{$\blacksquare$}&
\textcolor{Black}{$\blacksquare$}&
\textcolor{Brown}{$\blacksquare$}&
\textcolor{Gold}{$\blacksquare$}&
]- 
& $100\times 10^{1}\pm 5\%$ & 1k & 1 тысяча ом & 1000 Ом & 1k $\Omega$ \\
&1&0&0&1&5\\
-[&
\textcolor{Orange}{$\blacksquare$}&
$\square$&
$\blacksquare$&
\textcolor{Black}{$\blacksquare$}&
\textcolor{Brown}{$\blacksquare$}&
]- 
& $390\times 10^{0}\pm 1\%$ & 390R & 390 Ом & 390$\Omega$ \\
&3&9&0&0&1\\
-[&
\textcolor{Brown}{$\blacksquare$}&
\textcolor{Black}{$\blacksquare$}&
\textcolor{Black}{$\blacksquare$}&
\textcolor{Black}{$\blacksquare$}&
\textcolor{Brown}{$\blacksquare$}&
]- 
& $100\times 10^{0}\pm 1\%$ & 100R & 1000 Ом & 100$\Omega$ \\
&1&0&0&0&1\\
-[&
\textcolor{Yellow}{$\blacksquare$}&
\textcolor{Magenta}{$\blacksquare$}&
\textcolor{Black}{$\blacksquare$}&
\textcolor{Gold}{$\blacksquare$}&
\textcolor{Brown}{$\blacksquare$}&
]- 
& $470\times 10^{-1}\pm 1\%$ & 47R & 47 Ом & 47$\Omega$ \\
&4&7&0&-1&1\\
\end{tabular}
