\secrel{Станочное оборудование}\label{stanki}
\secdown

\input{tabletop/tabletop}

\secrel{Самодельная оснастка и станки}
\secdown

\secrel{Сверлилка из китайского шпинделя}

\secrel{Ручной резьбонарезной стенд}

\secrel{Намоточный станок}

\secup


\secrel{Промышленные станки}\secdown

% Самый распространенные станки\ --- \term{сверлильные}, т.к. имеют самую простую
% конструкцию, и минимальную стоимость. Предназначены для самой частой операции:
% изготовления перпендикулярных круглых дырок в различных материалах, топовые
% модели имеют также функцию нарезения резьбы.
% 
% \bigskip
% 
% Наиболее многочисленную группу металлорежущих станков составляют \term{токарные
% станки}, используются в механических, инструментальных и ремонтных цехах
% заводов, а также в ремонтных мастерских в основном для обработки деталей,
% имеющих форму тел вращения. При использовании соответствующей оснастки позволяют
% растачивать отверстия в призматических (прямоугольных) деталях, и фрезеровать
% небольшие детали. Самый ходовой тип детали\ --- тело вращения с наружними и
% внутренними резьбами: валики, втулки, оси, болты, винты, шпильки, кольца, шайбы
% и т.д.
% 
% К основным размерам, характеризующим токарный станок, относятся 
% \begin{itemize}
%   \item наибольший допустимый диаметр обрабатываемой заготовки, 
%   \item высота \term{центров} над станиной и 
%   \item расстояние между центрами.
%   \item 
% Часто обращают внимание на диаметр \emph{проходного отверстия шпинделя},
% определяющий максимальный диаметр \term{длинномерных заготовок}, что важно при
% изготовлении мелких партий деталей или нарезке резьб на трубах.
% \end{itemize}
% 
% \bigskip
% Значительную часть среди металлорежущих станков составляют \term{фрезерные
% станки}. Наибольшее распространение имеют консольно-фрезерные.
% Предназначены для выполнения различных фрезерных работ цилиндрическими,
% дисковыми, фасонными и другими \term{фрезами}, можно фрезеровать плоскости,
% пазы, фасонные поверхности, и т.д. Кроме этого, универсальные
% консольно-фрезерные станки c поворотным столом или делительной головкой
% позволяют фрезеровать различного рода винтовые канавки и зубья зубчатых колес.
% 
% Основными размерами фрезерных станков, по которым можно определить возможность
% установки и обработки конкретных заготовок с определенными габаритами, являются
% размеры рабочей поверхности стола (длина и ширина) и \emph{рабочий ход
% стола}/\term{рабочая зона} в продольном, поперечном и вертикальном направлениях.

\secrel{Маркировка моделей станков производства СССР}

% \begin{tabular}{p{0.3\textwidth} p{0.6\textwidth}}
% \includegraphics[height=0.3\textheight]{stanki/chugunok.jpg}
% &
% станок-''чугунок'', простой, дубовый, надежный (потому что ненадежные давно
% сломались), дешевый, но требует помещение с силовым полом, 3х-фазное питание, и
% кучу времени на поиск запчастей для восстановления по металлобазам и развалам. В
% диком виде пока что встречается в школах и других типа учебных заведениях, т.к.
% висит на балансе, но не эксплуатируется, и не обслуживается, потому что некем.
% Отличается дешевизной (ржавого) инструмента и (еще более ржавой) оснастки, и
% некоторым гемором с поиском запчастей.
% \\
% \end{tabular}
% \clearpage
% 
% 
% Для станков, выпускавшихся в СССР, принята единая система классификации и
% условных обозначений, основанная на присвоении каждому станку особого шифра
% (номера). Cтанки каждой группы подразделяются на девять \emph{типов}.
% Внутри каждого типа металлорежущие станки могут отличаться друг от друга
% конструктивными особенностями. Эти особенности, а также некоторые другие
% характеристики и отражаются в шифре (номере) станка.
% 
% \begin{verbatim}
% <группа>[<буква1>]<тип><характеризующий размер>[<буква2>][M][Фn]
% \end{verbatim}
% 
% Кроме цифр, в условные обозначения модели станка часто входят буквы. Если
% \emph{буква1} стоит между первой и второй цифрами, то это означает, что
% конструкция станка подверглась усовершенствованию по сравнению с прежней
% моделью. 
% 
% Если \emph{буква2} стоит в конце номера станка, то это говорит об
% изменении основной, «базовой» модели станка. Часто \emph{буква2} задает класс
% точности:
% 
% \begin{description}
% \item[Н] нормальная (часто не указывается)
% \item[П] повышенная
% \item[В] высокая
% \item[А] особо высокая
% \item[С] особо-точный станок
% \end{description}
% 
% Для станков с ЧПУ:
% \begin{description}
% \item[М] наличие магазина инструментов,
% \item[Ф1] станки с цифровой индикацией и преднабором координат,
% \item[Ф2] с позиционными и прямоугольными системами,
% \item[Ф3] с контурными системами,
% \item[Ф4] с универсальной системой для позиционной и контурной обработки. Эти
% шифры пишутся в конце номера модели.
% \end{description}
% 
% \bigskip
% \noindent\emph{группа}/\emph{тип}:
% \begin{enumerate}[label={\arabic*}]
%   \item токарные;
%   \begin{enumerate}[label={\arabic*}]
%     \item 
%     \item 
%     \item револьверные;
%     \item 
%     \item 
%     \item токарно-винторезные;
%   \end{enumerate}
%   \item сверлильные и расточные;
%   \begin{enumerate}[label={\arabic*}]
%     \item вертикально-сверлильные,
%     \item одношпиндельные полуавтоматы,
%     \item многошпиндельные полуавтоматы,
%     \item координатно-расточные,
%     \item радиально-сверлильные,
%     \item горизонтально-расточные,
%     \item алмазно-расточные,
%     \item горизонтально-сверлильные,
%     \item разные сверлильные.  
%   \end{enumerate}
%   \item шлифовальные, полировальные, доводочные и заточные;
%   \item специальные;
%   \item зубо- и резьбообрабатывающие;
%   \item фрезерные;
%   \item разрезные;
%   \item строгальные, долбежные, протяжные;
%   \item прочие
% \end{enumerate}
% 
% \bigskip
% \term{Характеризующий размер}:
% \begin{description}
% \item[токарные] \hfill \\
% высота оси шпинделя над станиной, \\
% задает \emph{максимально возможный радиус} обрабатываемой \emph{заготовки}
% \end{description}
% 
% \bigskip
% Пример: 1А616: станок токарно(1)-винторезный(6), модификация(А), высота
% центров над станиной (16)0 мм.

\clearpage
\secrel{\odina: станок токарно-винторезный}\label{latheodina}\secdown

\includegraphics[height=0.5\textheight]{stanki/1A616.jpg}

\secrel{Назначение и области применения}

\secrel{Распаковка и транспортировка}

\secrel{Фундамент станка, монтаж и установка}

\secrel{Подготовка станка к первоначальному пуску}

\secrel{Паспортные данные}

\secrel{Описание основных узлов}

\secrel{Смазка}

\secrel{Первоначальный пуск}

\secrel{Указания по технике безопасности}

\secrel{Настройка}

\secrel{Регулирование}

\secrel{Ведомость комплектации}

\secup


\secup

\secup
