\secrel{2.10 Добавление транзистора в схему 20}

2N7000

\bigskip
\termdef{Транзистор}{транзистор}\ --- электронный элемент, выполняющий функцию
\termdef{усиления}{усиление}: \emph{управление сильным сигналом под контролем
слабого сигнала}.

\bigskip
Частный случай транзистора\ --- \termdef{Полевой транзистор}{полевой транзистор} 
(\termdef{FET}{FET}, [F]ield [E]ffect [T]ransistor). Управление выполняется
слабым входным \emph{напряжением}, которое управляет выходным током, изменяя
проводимость (сопротивление) части транзистора в выходной цепи. 

Главная особенность полевого транзистора\ --- во входной цепи, по которой
подается управляющее напряжение, течет почти нулевой \termdef{входной
ток}{входной ток}, т.е. \emph{полевой транзистор имеет бесконечное
\termdef{входное сопротивление}{входное сопротивление}}. Эта особенность важна
при подключении источников слабого сигнала (датчиков), генерирующих слабое
сигнальное напряжение, но не способных выдать сколь нибудь заметный ток\ ---
электро-динамический микрофон, индуктивный звукосниматель электрогитары и т.п.
Сигнал от таких датчиков усиливается \termdef{предварительным
усилителем}{предварительный усилитель} на полевом транзисторе, а затем усиленный
сигнал подается в остальную часть схемы для дальнейшей обработки.
