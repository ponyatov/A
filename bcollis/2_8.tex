\secrel{Важные понятия схемотехники}

Кратко повторим:
\bigskip

Схема состоит из нескольких компонентов и источника питания, соединенных
проводами.

\emph{Поток электронов} (часто называют \termdef{носители заряда}{носители
заряда}) течет в цепи; однако если нет \emph{полной (замкнутой) цепи}, электроны
не могут течь.

\emph{Напряжение} (U) является мерой энергии в цепи, оно используется в качестве
меры энергии, подаваемой от батареи или энергии (напряжения) через часть схемы.

\emph{Ток (I) представляет собой поток электронов} от батареи по контуру и
обратно к батарее. Ток измеряется в амперах (обычно мы будем использовать
миллиампер или мА). Обратите внимание, что \emph{электрический ток это не ток
электронов или ток зарядов}. Так же, как течение реки не эквивалентно потоку
воды.

\emph{Сопротивление} работает \emph{на уменьшение тока}, резисторы в схеме
оказывают сопротивление току.

\emph{Проводники}, такие как провода, соединяющие компоненты вместе, не имеют
(теоретически) никакого сопротивления току.

\bigskip
Действительно важное понятие, для ясного понимания:

Напряжение прикладывается параллельно компонентам, а ток течет через компоненты.
