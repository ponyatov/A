% % 3.3 The Schematic Editor  28

\secrel{The Schematic Editor}

% The first part of the process in creating a PCB is drawing the schematic.
% 1. Parts will be added from libraries
% 2. and joined together using ‘nets’ to make the circuit
% 
% 3.3.1 The Toolbox
% As you point to the tools in the TOOLBOX their names will appear in a popup and also their
% description will appear in the status bar at the bottom of the window
% Find the following tools
% 
% 
% ADD A PART
% 
% 
% 
% MOVE AN OBJECT
% 
% 
% 
% DELETE AN OBJECT
% 
% 
% 
% DEFINE THE NAME OF AN OBJECT
% 
% 
% 
% DEFINE THE VALUE OF AN OBJECT
% 
% 
% 
% DRAW NETS (connections)
% 
% 
% 
% ERC (electrical rule check)
% 
% 28
% 
% 3.3.2 Using parts libraries
% Selecting parts libraries to use.
% Parts are stored within libraries and there are a large number of libraries in Eagle.
% It is not hard to create your own library and modify the parts within it. The cls.lbr has many already
% modified components within it. If Eagle is not setup to use the cls library you will need to do it
% now.
% 1.
% 2.
% 3.
% 4.
% 5.
% 6.
% 7.
% 8.
% 9.
% 10.
% 11.
% 
% From your internet browser save the file cls.lbr into your Eagle folder.
% In Eagle's control panel from the menu select options then directories
% In the new window that appears make sure the directories for the libraries are highlighted
% Click on browse and find your Eagle.directory
% Next highlight the directories for Projects
% Click on browse and find your Eagle directory again.
% Choose OK.
% You might need to close EAGLE and restart it to make sure it reads the libraries ok.
% To use a library right click on it from within the Control Panel
% Make sure Use is highlighted. It will have a green dot next to it if it is selected
% At this time right click on the other lbr folder and select Use none.
% 
% NOTE THE IMPORTANCE OF THE GREEN DOT NEXT TO THE LIBRARY,
% if its not there you will not see the library in the schematic editor!
% 
% 29
% 
% 3.3.3 Using Components from within libraries.
% From your schematic Click the ADD button
% in the toolbox
% A new window will open (it may take a
% while)
% 
% 
% 
% 
% 
% 
% Find the CLS library
% Open it by double clicking on it or by
% clicking the + sign
% Open the R-EU_ section (ResistorEuropean)
% Here you will find the 0204/10
% resistor.
% Select it and then click OK
% 
% Add 2 more resistors of the same type.
% Add all of the following parts
% LIBRARY
% PART
% Qty
% cls
% REU-0204/10
% 3
% cls
% LDR
% 1
% cls
% 2,54/0,8 (wirepads)
% 2
% cls
% led 5MM
% 1
% cls
% 1N4148 D41-10
% 1
% cls
% 2N7000
% 1
% cls
% GND
% 3
% A wirepad allows us to connect wires to the PCB (such as wires to switches and batteries)
% 
% 3.3.4 Different component
% packages
% There are several different types of
% resistors; they all have the same symbol
% however resistors come in different
% physicalpackages so we must choose an
% appropriate one.The 0204/7 is suitable
% for us but any of the 4 smallest ones
% would be OK.
% 
% 30
% 
% Moving parts
% Move the parts around within the schematic editor so that they are arranged as per the schematic
% below. Keep the component identifiers (numbers like R1, R2, R3) in the same places as those
% below.
% 
% 3.3.5 Wiring parts together
% 
% These form the electrical connections that makeup the
% circuit. Select the net button from the toolbox.
% Left click on the very end of a component and draw in a
% straight line either up, down, left or right.
% Left click again to stop at a point and draw before drawing in
% another direction.
% Double left click at another component to finish the wire.
% 
% 31
% 
% 3.3.6 Zoom Controls
% There are a number of zoom controls that can be used to help you work in your circuit.
% 
% Find these on the toolbar and identify what each does.
% Nets
% Nets are the wire connections between the components, each has a
% unique name.
% Find the info button in the toolbox and check the names and details of the
% components and nets/wires.
% When you want to connect a new net to an existing net, Eagle will prompt
% you as to which name to give the combined net.
% If one of the nets has a proper name i.e. VCC, V+,V-, ground... use that
% name, otherwise choose the net with the smallest number
% 
% 3.3.7 Junctions
% Junctions are the dots at joins in the circuit, they are there to make sure that the wires are
% electrically connected. Generally you will NOT need to add these to your circuit as the net tool puts
% them in place automatically
% 
% 32
% 
% 3.3.8 ERC
% The ERC button causes Eagle to test the schematic for electrical errors.
% Errors such as pins overlapping, and components unconnected are very common.
% The ERC gives a position on the circuit as to where the error is; often zooming in on that point and
% moving components around will help identify the error.
% You must correct all errors before going on.
