\chapter{1 Введение в практическую электронику 13}

\section{1.1 Ваше обучение по специальности <<Технология>> 14}

Эта книга\note{оригинал: B.Collis The Introduction to Practical 
Electronics\ldots}\ имеет слеующий ряд основных направлений:

\begin{itemize}
  \item Распознавание электронных компонентов и их правильное использование
  \item Наработка цельного набора компетенций в базовой электронике
  \item Использование макетных плат
  \item Навыки ручной пайки
  \item Использование закона Ома для выбора токоограничивающих резисторов
  \item Делитель напряжения
  \item Использование EDA CAD\note{\keys{E}\,lectronic \keys{D}\,esign
  \keys{A}\,utomation, САПР автоматизации проектирования электроники}\ для
  разработки и подготовки производства печатных плат
  \item Программирование микроконтроллеров и их сопряжение
  \item Транзистор в ключевом режиме
  \item Теория источников питания
  \item Принципы и схемы электропривода
  \item Моделирование решений через тестирование и испытания
  \item Следование кодексу практики
  \item Безопасные приемы работы
\end{itemize}

\section{1.2 Ключевые компетенции из Ново-Зеландской учебной программы 14}

