\chapter{1 Введение в практическую электронику 13}

Эта книга\note{оригинал: B.Collis The Introduction to Practical 
Electronics\ldots}\ имеет слеующий ряд основных направлений:

\begin{itemize}
  \item Распознавание электронных компонентов и их правильное использование
  \item Наработка цельного набора компетенций в базовой электронике
  \item Использование макетных плат
  \item Навыки ручной пайки
  \item Использование закона Ома для выбора токоограничивающих резисторов
  \item Делитель напряжения
  \item Использование EDA CAD\note{\keys{E}\,lectronic \keys{D}\,esign
  \keys{A}\,utomation, САПР автоматизации проектирования электроники}\ для
  разработки и подготовки производства печатных плат
  \item Программирование микроконтроллеров и их сопряжение
  \item Транзистор в ключевом режиме
  \item Теория источников питания
  \item Принципы и схемы электропривода
  \item Моделирование решений через тестирование и испытания
  \item Следование кодексу практики
  \item Безопасные приемы работы
\end{itemize}

\section{Ваше обучение по специальности <<Технология>>}

\begin{itemize}

\item \textbf{Технологическая практика}

\begin{itemize}

\item\textbf{Быть четким}: разработка четких спецификаций для ваших
технологических проектов.

\item\textbf{Планирование}: думать прежде чем делать, и использовать во время
работы наброски типа блок-схем, принципиальных схем, чертежей разводки плат,
диаграмм и эскизов.

\item\textbf{Работа на результат}: испытания, тестирование и сборка электронных
схем, проектирование и изготовление печатных плат, написание программ для
микроконтроллеров.

\end{itemize}

\item \textbf{Технологические знания}

\begin{itemize}

\item\textbf{Технологическое моделирование}: прежде чем строить электронное
устройство, важно понять как оно работает сначала путем моделирования и/или
тестирования аппаратного и программного обеспечения.

\item\textbf{Технологические продукты}: знания о компонентах и ​​их
характеристиках.

\item\textbf{Технологические системы}: электронное устройство является более,
чем набором компонентов, это функционирующая система с входами, выходами и
контролирующим процессом.

\end{itemize}

\item \textbf{Природа технологии}

\begin{itemize}

\item\textbf{Характеристики технологических достижений}: знания об электронных
компонентах, особенно микроконтроллерах как основа современных технологий.

\item\textbf{Роль технологии}: электронные устройства в настоящее время играют
центральную роль в инфраструктуре нашего современного общества; мы их хозяева,
как они изменили нашу жизнь?

\end{itemize}

\end{itemize}

\section{Ключевые компетенции Ново-Зеландской программы}

\begin{itemize}

\item\textbf{Размышление}: для меня задачей технологии является все что касается
размышления. Моя цель: заставить студентов понимать технологии, использованные в
электронных устройствах. Для достижения этой цели студенты должны активно
взаимодействовать с их работой на самом раннем этапе, чтобы они могли построить
свое собственное понимание и пойти дальше, чтобы стать хорошими решалами
проблем. В начале своего обучения в электронике это требует от студентов
понимания инструкций, которые им дают и поиск ясности, когда они не понимают их.
Для этого многие новые и различные элементы знаний рассматриваются на занятиях,
и студентам выдаются задания на решение проблем чтобы помочь им мыслить
логически. Копирование чужого ответа является ошибочным, но приветствуется
совместная работа. В основе обучения лежит построение правильных концептуальных
моделей и анализ в контексте "большой картины".

\item\textbf{Relating to others} – working together in pairs and groups is as
essential in the classroom as it is in any other situation in life; we all have to share and negotiate resources and equipment
with others; it is essential therefore to actively communicate with each other and assist one
other.

\item\textbf{Using language symbols and texts} – At the heart of our subject is
the language we use for communicating electronic circuits, concepts, algorithms and computer programming syntax; so
the ability to recognise and using symbols and diagrams correctly for the work we do is vital.

\item\textbf{Managing self} – This is about students taking personal
responsibility for their own learning; it is about challenging students who expect to read answers in a book or have a teacher tell them
what to do. It means that students need to engage with the material in front of them.
Sometimes the answers will come easily, sometimes they will not; often our subject involves a
lot of trial and error (mostly error). Students should know that it is in the tough times that the
most is learnt. And not to give up keep searching for understanding.

\item\textbf{Participating and contributing} – We live in a world that is
incredibly dependent upon technology especially electronics, students need to develop an awareness of the importance
of this area of human creativity to our daily lives and to recognise that our projects have a
social function as well as a technical one.

\end{itemize}
