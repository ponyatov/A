\chapter{1 Введение в практическую электронику 13}

Эта книга\note{оригинал: B.Collis The Introduction to Practical 
Electronics\ldots}\ имеет слеующий ряд основных направлений:

\begin{itemize}
  \item Распознавание электронных компонентов и их правильное использование
  \item Наработка цельного набора компетенций в базовой электронике
  \item Использование макетных плат
  \item Навыки ручной пайки
  \item Использование закона Ома для выбора токоограничивающих резисторов
  \item Делитель напряжения
  \item Использование EDA CAD\note{\keys{E}\,lectronic \keys{D}\,esign
  \keys{A}\,utomation, САПР автоматизации проектирования электроники}\ для
  разработки и подготовки производства печатных плат
  \item Программирование микроконтроллеров и их сопряжение
  \item Транзистор в ключевом режиме
  \item Теория источников питания
  \item Принципы и схемы электропривода
  \item Моделирование решений через тестирование и испытания
  \item Следование кодексу практики
  \item Безопасные приемы работы
\end{itemize}

\section{Ваше обучение по специальности <<Технология>>}

\begin{itemize}

\item \textbf{Технологическая практика}

\begin{itemize}

\item\textbf{Быть четким}: разработка четких спецификаций для ваших
технологических проектов.

\item\textbf{Планирование}: думать прежде чем делать, и использовать во время
работы наброски типа блок-схем, принципиальных схем, чертежей разводки плат,
диаграмм и эскизов.

\item\textbf{Работа на результат}: испытания, тестирование и сборка электронных
схем, проектирование и изготовление печатных плат, написание программ для
микроконтроллеров.

\end{itemize}

\item \textbf{Технологические знания}

\begin{itemize}

\item\textbf{Технологическое моделирование}: прежде чем строить электронное
устройство, важно понять как оно работает сначала путем моделирования и/или
тестирования аппаратного и программного обеспечения.

\item\textbf{Технологические продукты}: знания о компонентах и ​​их
характеристиках.

\item\textbf{Технологические системы}: электронное устройство является более,
чем набором компонентов, это функционирующая система с входами, выходами и
контролирующим процессом.

\end{itemize}

\item \textbf{Nature of Technology}

\begin{itemize}

\item\textbf{Characteristics of Technological Outcomes} – knowing about
electronic components especially microcontrollers as the basis for modern technologies.

\item\textbf{Characteristics of Technology} – electronic devices now play a
central role in the infrastructure of our modern society; are we their masters, how have they changed our lives?

\end{itemize}

\end{itemize}

\section{Ключевые компетенции Ново-Зеландской программы}

\begin{itemize}

\item\textbf{Thinking} – to me the subject of technology is all about thinking.
My goal is to have students understand the technologies embedded within electronic devices. To achieve this students
must actively enage with their work at the earliest stage so that they can construct their own
understandings and go on to become good problem solvers. In the beginning of their learning
in electronics this requires students to make sense of the instructions they have been given
and search for clarity when they do not understand them. After that there are many new and
different pieces of knowledge introduced in class and students are given problem solving
exercises to help them think logically. The copying of someone elses answer is flawed but
working together is encouraged. At the core of learning isbuilding correct conceptual models
and to have things in the context of the ‘big picture’.

\item\textbf{Relating to others} – working together in pairs and groups is as
essential in the classroom as it is in any other situation in life; we all have to share and negotiate resources and equipment
with others; it is essential therefore to actively communicate with each other and assist one
other.

\item\textbf{Using language symbols and texts} – At the heart of our subject is
the language we use for communicating electronic circuits, concepts, algorithms and computer programming syntax; so
the ability to recognise and using symbols and diagrams correctly for the work we do is vital.

\item\textbf{Managing self} – This is about students taking personal
responsibility for their own learning; it is about challenging students who expect to read answers in a book or have a teacher tell them
what to do. It means that students need to engage with the material in front of them.
Sometimes the answers will come easily, sometimes they will not; often our subject involves a
lot of trial and error (mostly error). Students should know that it is in the tough times that the
most is learnt. And not to give up keep searching for understanding.

\item\textbf{Participating and contributing} – We live in a world that is
incredibly dependent upon technology especially electronics, students need to develop an awareness of the importance
of this area of human creativity to our daily lives and to recognise that our projects have a
social function as well as a technical one.

\end{itemize}
