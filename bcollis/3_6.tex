% % 3.6 PCB Making  38


% 3.6 PCB Making
% PCB Board Layers
% 
% Measure, Cut:
% Photosensitive board is expensive, so it is
% important not to waste it and cut it to the
% right size.
% It is also sensitive to ordinary light so when
% cutting it don’t leave it lying around.
% 
% Expose:
% This over head projector is a great source of
% UV – ultra violet light, it takes three minutes
% on the OHP in my classroom.
% The overhead transparency produced earlier
% must have some text on it. The text acts as
% a cue or indication of which way around the
% acetate and board should be. We want the
% text on the board to be around the right way.
% 
% Develop:
% The developer chemical we use is sodium
% metasillicate which is a clear base or alkali.
% It will ruin your clothes so do not splash it
% around, it is a strong cleaning agent!
% It should be heated to speed up the process.
% The development process takes anywhere
% from 20 seconds to 2 minutes. The reason
% being that the chemical dilutes over time
% making the reaction slower.
% The board should be removed twice during
% the process and washed gently in water to
% check the progress.
% 
% 38
% 
% Rinse:
% The developer must becompletely removed
% from the board.
% At this stage if there is not time to etch the
% board, dry it and store it in a dark place.
% 
% Etch:
% The etching chemical we use if ferric
% chloride, it is an acid and will stain your
% clothes.
% The tank heats the etching solution and
% there is a pump to blow bubbles through the
% liquid, this speeds the process up radically
% so always use the pump.
% Etching may take from 10 to 30 minutes
% depending upon the strength of the solution.
% 
% Rinse:
% Thoroughly clean the board.
% 
% Remove Photosensitve Resist:
% The photosensitive layer left on the tracks
% after etching is complete must be removed.
% Thee asiest way to do this is to put the
% board back into the developer again. This
% may take about 15 minutes.
% 
% 39
% 
% Laquer:
% The copper tracks on the board will oxidise
% very quickly (within minutes the board may
% be ruined), so the tracks must be protected
% straight away, they can be sprayed with a
% special solder through laquer (or tinned).
% 
% Drilling & Safety:
% Generally we use a 0.9mm drill in class. This
% suits almost all the components we use.
% Take you time with drilling as the drill bit is
% very small and breaks easily.
% As always wear safety glasses!
% 
% Use a third hand:
% When soldering use something to support
% the board. Also bend the wires just a little to
% hold the component in place (do not bend
% them flat onto the track as this makes them
% very hard to remove if you make a mistake).
% 
% 40
