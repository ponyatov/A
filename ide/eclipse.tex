\chapter{\eclipse}\label{eclipse}

\includegraphics[height=0.5\textheight]{logo/eclipse.png}

\section{Установка \eclipse\ под \win}

\menu{\winr>\url{https://eclipse.org/}>Download}

\menu{Eclipse Luna release for>\win}

Качаем архив базовой системы:

\menu{Eclipse IDE for Java Developers>\win\ 32/64 Bit}

Или сразу сборку CDT\eclipse:

\menu{Eclipse IDE for C/C++ Developers>\win\ 32/64 Bit}

\section{Установка \eclipse\ под \linux}

\menu{\winr>\url{https://eclipse.org/}>Download}

\menu{Eclipse Luna release for>\win}

Качаем архив базовой системы:

\menu{Eclipse IDE for Java Developers>\linux\ 32/64 Bit}

Или сразу сборку CDT\eclipse:

\menu{Eclipse IDE for C/C++ Developers>\linux\ 32/64 Bit}

\bigskip

Устанавливаем в систему Java-рантайм:

\begin{verbatim}
sudo aptitude install openjdk-7-jre
\end{verbatim}

Распаковывем полученный архив
\file{eclipse-java-luna-SR1-linux-gtk-x86\_64.tar.gz}
в \file{\$HOME}:

\begin{verbatim}
cd ~
tar zx < Downloads/eclipse-java-luna-SR1-linux-gtk-x86_64.tar.gz 
ls -la eclipse/eclipse
-rwxr-xr-x 1 user user 74675 Авг 13 16:12 eclipse/eclipse
\end{verbatim}

Прописываем запуск \eclipse\ в ваш оконный менеджер или \file{.blackboxmenu}
с параметром \file{-noSplash} для лечения глюка с запуском на x64-битных
системах:

\lst{.blackbox.menu}{}{ide/eclipse_blackbox.menu}

\section{Установка CDT}

\href{https://eclipse.org/cdt/}{\prog{CDT}}\ --- расширение \eclipse\ для
разработки на Си/\cpp, редактирования make-файлов. Это расширение критически
важно для вашей работы, поэтому ставить его обязательно, или сразу качать сборку
CDT\eclipse.
\bigskip

\menu{\eclipse>Help>Install New Software\ldots}

\menu{Work with>Add>Add repository}

\menu{Name>CDT}

\menu{Location>\url{http://download.eclipse.org/tools/cdt/releases/8.5}}

\menu{OK}

\menu{Work with>CDT}

\menu{CDT Main Features>\checkbox\ C/C++ Development Tools}

\menu{CDT Optional Features}

Парсер файлов исходников на диалекте С99:
\menu{\checkbox\ C99 LR Parser}

Поддержка \prog{gcc}\ в режиме кросс-компиляции:
\menu{\checkbox\ GCC Cross Compiler Support}

Аппаратная отладка через \prog{gdb}:
\menu{\checkbox\ GDB Hardware Debugging}

\menu{Next>Next>Accept>Finish}

\section{Установка TeXlipse}

Если планируете работать с документацией в формате \latex, установите расширение
\href{http://texlipse.sourceforge.net/}{\prog{TeXlipse}}:
\bigskip

\menu{\eclipse>Help>Install New Software\ldots}

\menu{Work with>Add>Add repository}

\menu{Name>TeXlipse}

\menu{Location>\url{http://texlipse.sourceforge.net/}}

\menu{OK}

\menu{Work with>TeXlipse}

Это расширение поддерживает подсветку синтаксиса, автодополнение, построение
динамического оглавления, автокомпиляцию по сохранению, и несколько визардов
создания проекта.

\section{Редактирование файлов в формате XML и производных}

Установите пакет \eclipse\ WST:

\menu{Help>Install New Software}

\menu{Work with:>Luna - http://download.eclipse.org/releases/luna}

\menu{Filter:>WST>Eclipse WST>Next>Next>Restart>OK}

\section{Проверка орфографии}

\cp{http://www.simplecoding.org/proverka-orfografii-v-eclipse.html}

То, что проверка орфографии очень удобная вещь вряд ли нужно объяснять. Есть
конечно люди, которые не обращают на неё внимание, но это чаще всего из-за
экономии времени и отсутствия удобных средств проверки.

Действительно, удобная автоматическая проверка орфографии есть в офисных
пакетах, но мне сложно представить разработчика, который будет переносить
комментарии в Word и обратно \smiley.

Поэтому очень удобно иметь \emph{проверку правописания прямо в IDE}. И \eclipse\
в этом смысле полностью соответствует ожиданиям.

Долго объяснять что к чему нет смысла. Проверка орфографии встроена в \eclipse\
и если вы пишите только на английском, то может быть не захотите ничего менять.

Кроме того, есть
\href{http://www.102degrees.com/blog/2007/07/09/spell-checking-in-eclipse-pdt/}{статья
Aaron'а} (en) в которой автор рассказывает о подключении дополнительных словарей
и плагине \file{eSpell}.

Но \emph{русских словарей в дистрибутиве нет}, а при подключении внешних есть
нюансы. Поэтому мы максимально подробно рассмотрим \emph{подготовку и добавление
русских словарей}.

Первый вопрос. В каком виде должны быть словари и где их взять?

Тут всё просто. Формат словаря\ --- обычный текстовый файл, в котором каждое
слово начинается с новой строки. И нам вполне подойдут свободно распространяемые
словари \file{aSpell}.

Установка состоит из \ref{aspellecl}\ шагов:
\begin{enumerate}
  \item качаем \href{}{aSpell}\ и словари для нужных языков
  
  \menu{\winr>\url{http://aspell.net/win32/}>}
  
  \menu{Binaries>Full installer}
  
  \menu{Precompiled dictionaries>English}
  
  \menu{Precompiled dictionaries>Russian}
  
  \item устанавливаем сначала \file{aSpell}, потом отдельно каждый словарь
  
  \menu{\file{Aspell-0-50-3-3-Setup.exe}>Setup GNU Aspell>Next>License>Next}
  
  \menu{Directory>\file{C:/GnuWin32/Aspell}>Next>Next}
  
  \menu{Additional>Next>Install>Next>\uncheckbox\ View manual>Finish}
  
  \menu{\file{Aspell-en-0.50-2-3.exe}>Aspell English Dictionary>Next>License>Next}
  
  \menu{Directory>\file{C:/GnuWin32/Aspell}>Next>Next>Install>Finish}
  
  \menu{\file{Aspell-ru-0.50-2-3.exe}>Aspell Russian Dictionary>Next>License>Next}
  
  \menu{Directory>\file{C:/GnuWin32/Aspell}>Next>Next>Install>Finish}
  
  \item делаем дамп словарей, перекодируем из koi8r в utf8 и объединяем
  
  \menu{\winr cmd}

\begin{lstlisting}
cd \GnuWin32\Aspell
bin\aspell dump master en > en.dict
bin\aspell dump master ru > ru.koi8
iconv -f koi8-r -t utf-8 < ru.koi8 > ru.dict
copy en.dict + ru.dict enru.dict
\end{lstlisting}
  
  \item \label{aspellecl} настраиваем \emph{spell-checker} \eclipse
  
  \menu{\eclipse>Window>Preferences>Editors>Text editors>Spelling}
  
  \menu{User defined dictionary>\file{C:/GnuWin32/Aspell/enru.dict}}
  
  \menu{Encoding>UTF-8}
  
  \menu{Apply>OK}
  
\end{enumerate}

