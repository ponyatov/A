\chapter{\eclipse}\label{eclipse}

\includegraphics[height=0.5\textheight]{logo/eclipse.png}

\section{Проверка орфографии}

\cp{http://www.simplecoding.org/proverka-orfografii-v-eclipse.html}

То, что проверка орфографии очень удобная вещь вряд ли нужно объяснять. Есть
конечно люди, которые не обращают на неё внимание, но это чаще всего из-за
экономии времени и отсутствия удобных средств проверки.

Действительно, удобная автоматическая проверка орфографии есть в офисных
пакетах, но мне сложно представить разработчика, который будет переносить
комментарии в Word и обратно \smiley.

Поэтому очень удобно иметь \emph{проверку правописания прямо в IDE}. И \eclipse\
в этом смысле полностью соответствует ожиданиям.

Долго объяснять что к чему нет смысла. Проверка орфографии встроена в \eclipse\
и если вы пишите только на английском, то может быть не захотите ничего менять.

Кроме того, есть
\href{http://www.102degrees.com/blog/2007/07/09/spell-checking-in-eclipse-pdt/}{статья
Aaron'а} (en) в которой автор рассказывает о подключении дополнительных словарей
и плагине \file{eSpell}.

Но \emph{русских словарей в дистрибутиве нет}, а при подключении внешних есть
нюансы. Поэтому мы максимально подробно рассмотрим \emph{подготовку и добавление
русских словарей}.

Первый вопрос. В каком виде должны быть словари и где их взять?

Тут всё просто. Формат словаря\ --- обычный текстовый файл, в котором каждое
слово начинается с новой строки. И нам вполне подойдут свободно распространяемые
словари \file{aSpell}.

Установка состоит из \ref{aspellecl}\ шагов:
\begin{enumerate}
  \item качаем \href{}{aSpell}\ и словари для нужных языков
  
  \menu{\winr>\url{http://aspell.net/win32/}>}
  
  \menu{Binaries>Full installer}
  
  \menu{Precompiled dictionaries>English}
  
  \menu{Precompiled dictionaries>Russian}
  
  \item устанавливаем сначала \file{aSpell}, потом отдельно каждый словарь
  
  \menu{\file{Aspell-0-50-3-3-Setup.exe}>Setup GNU Aspell>Next>License>Next}
  
  \menu{Directory>\file{C:/GnuWin32/Aspell}>Next>Next}
  
  \menu{Additional>Next>Install>Next>\uncheckbox\ View manual>Finish}
  
  \menu{\file{Aspell-en-0.50-2-3.exe}>Aspell English Dictionary>Next>License>Next}
  
  \menu{Directory>\file{C:/GnuWin32/Aspell}>Next>Next>Install>Finish}
  
  \menu{\file{Aspell-ru-0.50-2-3.exe}>Aspell Russian Dictionary>Next>License>Next}
  
  \menu{Directory>\file{C:/GnuWin32/Aspell}>Next>Next>Install>Finish}
  
  \item делаем дамп словарей, перекодируем из koi8r в utf8 и объединяем
  
  \menu{\winr cmd}

\begin{lstlisting}
cd \GnuWin32\Aspell
bin\aspell dump master en > en.dict
bin\aspell dump master ru > ru.koi8
iconv -f koi8-r -t utf-8 < ru.koi8 > ru.dict
copy en.dict + ru.dict enru.dict
\end{lstlisting}
  
  \item \label{aspellecl} настраиваем \emph{spell-checker} \eclipse
  
  \menu{\eclipse>Window>Preferences>Editors>Text editors>Spelling}
  
  \menu{User defined dictionary>\file{C:/GnuWin32/Aspell/enru.dict}}
  
  \menu{Encoding>UTF-8}
  
  \menu{Apply>OK}
  
\end{enumerate}

