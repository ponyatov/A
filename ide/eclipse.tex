\chapter{\eclipse}\label{eclipse}

\section{Проверка орфографии}

\cp{http://www.simplecoding.org/proverka-orfografii-v-eclipse.html}

То, что проверка орфографии очень удобная вещь вряд ли нужно объяснять. Есть
конечно люди, которые не обращают на неё внимание, но это чаще всего из-за
экономии времени и отсутствия удобных средств проверки.

Действительно, удобная автоматическая проверка орфографии есть в офисных
пакетах, но мне сложно представить разработчика, который будет переносить
комментарии в Word и обратно \smiley.

Поэтому очень удобно иметь \emph{проверку правописания прямо в IDE}. И \eclipse\
в этом смысле полностью соответствует ожиданиям.

Долго объяснять что к чему нет смысла. Проверка орфографии встроена в \eclipse\
и если вы пишите только на английском, то может быть не захотите ничего менять.

Кроме того, есть
\href{http://www.102degrees.com/blog/2007/07/09/spell-checking-in-eclipse-pdt/}{статья
Aaron'а} (en) в которой автор рассказывает о подключении дополнительных словарей
и плагине \file{eSpell}.

Но \emph{русских словарей в дистрибутиве нет}, а при подключении внешних есть
нюансы. Поэтому мы максимально подробно рассмотрим \emph{подготовку и добавление
русских словарей}.

Первый вопрос. В каком виде должны быть словари и где их взять?

Тут всё просто. Формат словаря\ --- обычный текстовый файл, в котором каждое
слово начинается с новой строки. И нам вполне подойдут свободно распространяемые
словари \file{aSpell}.

Установка состоит из \ref{aspellecl}\ шагов:
\begin{enumerate}
  \item качаем \href{}{aSpell}\ и словари для нужных языков
  \item устанавливаем сначала \file{aSpell}, потом отдельно каждый словарь
  \item делаем дамп словарей 
  \item объединяем дампы словарей
  \item \label{aspellecl} настраиваем \emph{spell-checker} \eclipse
\end{enumerate}

