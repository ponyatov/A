\begin{thebibliography}{99}
\addcontentsline{toc}{chapter}{Литература}

\bibitem{pyotkidach}

Г. Россум, Ф.Л.Дж. Дрейк, Д.С. Откидач, М. Задка, М. Левис, С. Монтаро, Э.С.
Реймонд, А.М. Кучлинг, М.-А. Лембург, К.-П. Йи, Д. Ксиллаг, Х.Г. Петрилли, Б.А.
Варсав, Дж.К. Ахлстром, Дж. Роскинд, Н. Шеменор, С. Мулендер.

\href{http://rus-linux.net/MyLDP/BOOKS/python.pdf}{Язык программирования Python}. 
/ 2001\ --- 454 c.

Python является простым и, в то же время, мощным интерпретируемым
объектно-ориентированным языком программирования. Он предоставляет структуры
данных высокого уровня, имеет изящный синтаксис и использует динамический
контроль типов, что делает его идеальным языком для быстрого написания различных
приложений, работающих на большинстве распространенных платформ. Книга содержит
вводное руководство, которое может служить учебником для начинающих, и
справочный материал с подробным описанием грамматики языка, встроенных
возможностей и возможностей, предоставляемых модулями стандартной библиотеки.
Описание охватывает наиболее распространенные версии Python: от 1.5.2 до 2.0.

\copyright\ Stichting Mathematisch Centrum, 1990–1995

\copyright\ Corporation for National Research Initiatives, 1995–2000

\copyright\ BeOpen.com, 2000

\copyright\ Д.С. Откидач, 2001


\bibitem{pythink} 

Аллен Дауни

\href{https://drive.google.com/file/d/0B0u4WeMjO894Q2hWV1QwOFFQOVk/view?usp=sharing}{Думать
на языке \py: Думать как компьютерный специалист} 

версия 1.1.24+Kart (Python 3.2), перевод версия 1.06 

\end{thebibliography}