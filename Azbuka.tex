\documentclass[oneside,12pt]{book}

% e-book format
\usepackage[paperwidth=210mm,paperheight=148mm,margin=10mm]{geometry}

% Cyrillization
\usepackage[T1,T2A]{fontenc}
\usepackage[utf8]{inputenc}
\usepackage[english,russian]{babel}
\usepackage{indentfirst}

% font setup for screen reading
\renewcommand{\familydefault}{\sfdefault}
\normalfont

% pdflatex options
\usepackage[unicode,colorlinks,linkcolor=blue,bookmarks=true]{hyperref}
\usepackage[pdftex]{graphicx}
\usepackage[usenames,dvipsnames,svgnames]{xcolor}

% listings
\usepackage{verbatim}
\usepackage{listings}
\lstset{
basicstyle=\small, % or \tiny \small or \footnotesize
extendedchars=true,inputencoding=utf8, % i18n
frame=single, % show frames around
numbers=left, numberstyle=\small,numbersep=1mm,% line numbering
tabsize=4, % tab style
keywordstyle=\color{Blue},%\texttt,
keywordstyle={[2]\color{Green}},%\texttt,
keywordstyle={[3]\color{Brown}},%\texttt,
keywordstyle={[4]\color{Red}},%\texttt,
keywordstyle={[5]\color{Blue}},%\texttt,
commentstyle=\color{Cyan}%\texttt%,
% showspaces=false
}

\usepackage{lstmk}\lstdefinestyle{mk}{language=mk}
\usepackage{lstrc}\lstdefinestyle{rc}{language=rc}

\newcommand{\lst}[3]{\lstinputlisting[title=\href{#2}{#1}]{#3}}
\newcommand{\lstx}[4]{\lstinputlisting[title=\href{#2}{#1},language=#4]{#3}}

% software menu & keys
\usepackage[os=win]{menukeys} 
\usepackage{amssymb} % windows key
\newcommand{\winstart}{$\boxplus$}
\newcommand{\winr}{\keys{\winstart+R}}
\newcommand{\file}[1]{\textbf{\textsf{#1}}}
\newcommand{\lms}{$\lhd$}
\newcommand{\dblms}{$\lhd\lhd$}
\newcommand{\rms}{$\rhd$}
\newcommand{\checkbox}{$\boxtimes$}
\newcommand{\uncheckbox}{$\square$}

% disable oneliner page breaks
\usepackage[defaultlines=2,all]{nowidow}

% books bib management
\usepackage{biblatex}
\addbibresource{../bib/python.bib}
\addbibresource{../bib/eskd.bib}
\addbibresource{../bib/electronics.bib}
\addbibresource{../bib/latex.bib}
\addbibresource{../bib/sat.bib}
\addbibresource{../bib/math.bib}
\addbibresource{../bib/sysdesign.bib}

\usepackage{makeidx}
\makeindex

% extra char sets
\usepackage{wasysym} % smileys

% set lists style
% \usepackage{enumitem}
% \setlist{nosep}

% misc

% \usepackage{titling}

\newcommand{\email}[1]{$<$\href{mailto:#1}{#1}$>$}
\newcommand{\internet}{Internet}

\newcommand{\cm}[1]{Cortex-M#1}
\newcommand{\cmx}{\cm{x}}

\newcommand{\linux}{Linux}
\newcommand{\emlinux}{em\linux}

\newcommand{\cpp}{$C^{+}_{+}$}
\newcommand{\py}{Python}

\newcommand{\vcs}{\hyperref[vcs]{VCS}}
\newcommand{\make}{\hyperref[make]{Make}}
\newcommand{\spice}{ngSPICE}
\newcommand{\latex}{\LaTeX}

\newcommand{\eclipse}{\textcircled{$\equiv$}\textsc{eclipse}}
\newcommand{\vim}{(g)Vim}

\newcommand{\note}[1]{\footnote{\ #1}}
\newcommand{\cp}[1]{\note{копипаста \url{#1}}}

\newcommand{\win}{\winstart Windows}

\newcommand{\mk}{МК}

\newcommand{\ram}{RAM}


\newcommand{\pref}[1]{/стр.\pageref{#1}/}

% selecting
\usepackage{framed}
\newcommand{\term}[1]{\textcolor{Green}{#1}}
\renewcommand{\emph}[1]{\textcolor{Blue}{#1}}
\newcommand{\prog}[1]{\textcolor{Brown}{#1}}
\newcommand{\pack}[1]{\textcolor{Magenta}{#1}}

% math
\usepackage{cancel}

% titles

\hypersetup{
	pdftitle={Азбука халтурщика-ARMатурщика},
	pdfauthor={ruOpenWrt, HackSpace <<Чебураторный завод>>, Консорциум хоббитов
	России, Bill Collis (Часть 1)}, 
	pdfsubject={https://github.com/ponyatov/Azbuka}
}


\author{\copyright\
\href{https://groups.google.com/forum/\#!forum/openwrt2ru}{ruOpenWrt}\\
\copyright\
HackSpace
<<\href{https://github.com/ponyatov/CHBZ/raw/master/presentation.pdf}{Чебураторный
завод}>>\\
\copyright\
Консорциум хоббитов России
}

\title{
\textbf{Азбука халтурщика-ARMатурщика}\\
разработка встраиваемых систем\\
основы бытовой автоматики,\\
систем управления и сбора данных
}

\begin{document}
\maketitle
\tableofcontents\clearpage

\clearpage\secly{О книге}

Эта книга\ --- комплект документации по аппаратно-программной платформе
ALYEH:

\begin{itemize}[nosep]
  \item \textcolor{red}{А}збука ARMатурщика
  \item \textcolor{red}{L}inux для встраиваемых систем
  \item д\textcolor{red}{Y}намический язык программирования \termdef{Ы}{Ы}
  \item библиотека \cpp\ для встраива\textcolor{red}{E}мых систем  
  \item \textcolor{red}{H}ardware библиотека универсальных модулей
\end{itemize}
\bigskip

\emph{В текущем состоянии эта книга\ --- конспект материалов, которые я сейчас
собираю, в черновой верстке. Объем материала очень большой, фактически это целая
специальность для приличного техникума, что-то типа ``Технология цифрового
производства''. Поэтому 146\% пока составляет сырая копипаста, с редкими
вкраплениями собственного бреда. В процессе адаптации, обкатки на студентах и
доработки эта поделка должна принять более вменяемый вид. Но учитывая полное
отсутствие обратной связи, этого никогда не случиться.}
\bigskip

Это учебное пособие было создано для интересующихся любительской электроникой,
самодельными цифровыми системами управления (Arduino, устройствами на
микроконтроллерах и т.п.), и программистов-лю\-би\-те\-лей. В связи с полной
деградацией системы образования пособие также рекомедуется для применения при
обучении в ВУЗах по специализациям, связанным с применением цифровой электроники
и компьютерной техники.

Большой упор был сделан на использование открытого некоммерческого программного
обеспечения, для удешевления учебного процесса, уменьшения себестоимости ваших
проектов\note{вряд ли ли у вас окажется лишняя пачка килобаксов на покупку пары
коммерческих САПР, по крайней мере пока ваш стартап не взлетит в Top\$100K}, и
стимулирования вашего участия в развитии этих программных пакетов.

Книга очень объемна и разнообразна по материалу, и построена как справочник с
группировкой материала по тематике. Для тех, кто только начинает, в разделе
\ref{learnplans}\ расписаны \termdef{пошаговые учебные планы}{учебный план} с
точки зрения параллельного изучения нескольких предметов с постепенным
усложнением\note{как это происходит при традиционном offline обучении}. Как
известно, главная часть любого обучения\ --- практическая. Особое внимание
уделено набору лабораторных работ.

\bigskip
В качестве видеоматериала были использованы 
\href{https://www.youtube.com/playlist?list=PLddc343N7YqgCWlspw08g6t0iFos9gAi4}{видеоуроки
физики}\\
\copyright\ Ерюткин Е.С., учитель физики высшей категории 
\href{http://sch1360v.mskobr.ru/}{ГБОУ СОШ №1360}, г.Москва

\bigskip
Мы признательны Bill Collis за разрешение использовать материалы его книги
<<\href{www.techideas.co.nz}{An Introduction to
Practical Electronics,
Microcontrollers and
Software Design}>> \cite{bcollis} в
русскоязычном варианте <<Азбуки>> (\ref{bcollis}), и конечно он вполне
заслуженно включен в основные соавторы этой книги.

\bigskip
Так как для работы в области электроники необходимо владение технологиями
изготовления конструктива, в книгу включен соответствующий раздел. 
Эти книги рекомендуются популярным поставщиком хоббийных настольных
микро-станков \href{http://sherline.com/}{Sherline Products}. Так как от
владельцев авторских прав не получено разрешение на полный официальный перевод,
для этих книг сделан только перевод-подстрочник, который поможет вам читать
оригинал:
\begin{itemize}
  \item Joe Martin, Craig Libuse \textbf{Tabletop Machining}
  \cite{tabletop} (\ref{tabletop})
  \item Doug Briney \textbf{Home Machinists Handbook}
  \cite{briney} (\ref{briney})
\end{itemize}

Отечественных книг по использованию маленьких ``часовых'' и настольных станков
просто не существует, хотя они и выпускались серийно. Исключение\ --- книга
Евгений Васильев \textbf{Маленькие станки} \cite{vasil}, но она имеет обзорный
характер.

\bigskip
\textbf{Лицензия на эту книгу пока не выбрана, так что она пока просто пишется в
духе OpenSource: любой может использовать ее часть, изменять или дополнять, до
тех пор, пока не накладываются какие-либо административные, финансовые или
юридические ограничения на распространение и развитие оригинальной версии или ее
открытых форков: \url{https://github.com/ponyatov/A}}
\bigskip

Приглашаем всех желающих участвовать в развитии этого учебного пособия на форум
\href{https://groups.google.com/forum/\#!forum/openwrt2ru}{ruOpenWrt} и в группу
\url{http://vk.com/samarahackerspace}, нам нужна обратная связь по качеству
материала, результаты тестирования на вас или ваших студентах, дополнения и
замечания.


\part{Основы электроники}

Здесь идет список ссылок на онлайн лекции в edX, Coursera, и т.п.

\chapter{Линейные схемы на пассивных элементах, основы электротехники}

\chapter{Симуляция и расчет схем в \spice}

\chapter{KiCAD} 

\section{Отрисовка схем в KiCAD}

\section{Библиотеки элементов}

\section{Передача схемы в \spice}

\chapter{Простейшие полупроводниковые элементы}

\section{Оптоэлектроника}

\section{Схемы на биполярных транзисорах} 

\section{Схемы на на полевых транзисорах}

\chapter{Операционные усилители}

\chapter{Источники питания}

\section{Батарейное питание}

\section{Линейные стабилизаторы}

\section{Импульсные преобразователи на ШИМ-контроллерах} 

\section{Цепи защиты и гашения кондуктивных помех}

\chapter{Цифровая электроника}

\chapter{Компьютерные интерфейсы}

\section{Поколение 90х: COM, LPT, ISA}

\subsection{Резервный программатор AVR ``пять проводков''}

\section{Сеть CAN}

\section{Интерфейсные модули USB}

\subsection{Универсальный высокоскоростной конвертер FTDI FT2232H}

\subsection{JTAG-адаптер}

\subsection{Отладочный модуль CAN}

\section{Интерфейсные модули Ethernet}

\chapter{ПЛИС}

\chapter{Датчики}

\chapter{Электропривод и исполнительные устройства}

\part{Основы конструирования РЭС}

\chapter{Пакеты моделирования на основе OpenFOAM}

\chapter{Обеспечение теплового режима}

\chapter{Электромагнитная совместимость}

\section{Кондуктивные помехи}

\section{Компоновочные модели и оптимизация кабельной сети}

\part{Технология РЭС}

\chapter{Трассировка плат и подготовка производства в KiCAD}

\section{Технология ЛУТ (Лазерный УТюг)}

\section{Технология фоторезиста}

\section{Формат Gerber и подготвка промышленного производства}

\chapter{FreeCAD}

\section{Чертеж}

\section{Эскиз}

\section{Деталь}

\section{Сборка}

\section{Автогенерация конструкторской докуметации}

\section{Скрипты и пользовательские расширения}

\chapter{Эксплуатация станочного оборудования}

\chapter{Основы ЧПУ и цифрового производства}

\section{CAM-пакеты для FreeCAD}

\part{Основы теории систем автоматического управления}

\chapter{Математический аппарат}

\section{Передаточная функция}

\section{Устойчивость САУ}

\section{Сети Петри}

\section{Автоматы Маркова}

\chapter{Релейное управление}

\chapter{Пропорциональные САУ}

\chapter{ПИДn-регуляторы}

\part{Разработка ПО для встраиваемых систем}

\chapter{Вспомогательные скрипты на языке Python}

\chapter{Make: управление сборкой проектов}

\chapter{VCS: cистемы контроля версий}

\section{CVS}

\section{Subversion}

\section{Git}

\subsection{GitHub}

\chapter{Основы Си и \cpp}

\subsection{Установка MinGW (win32)}

\section{Особенности \cpp\ в embedded}

\chapter{LLVM и разработка собственных компиляторов}

\section{Лексический и синтаксический анализ}

\section{Применение flex/bison для разбора текстовых форматов данных}

\section{Компилятор Паскаля}

\chapter{Сборка кросс-компилятора GNU toolchain}

\part{Микроконтроллеры \cmx}

\part{Периферия}

\part{Встраиваемый \emlinux}

\chapter{cross}

\chapter{BuildRoot}

\chapter{Особенности OpenWrt}

\chapter{Библиотека SDL}

\section{Реализация microGUI}

\chapter{Приложения для X Window}

\chapter{Программирование сетевых приложений}

\chapter{Сборка кросс-компиляторя GNU мальтийским крестом}

\part{Подготовка публикаций в \latex}
\section{Установка MikTeX (win32)}
\section{Структура документа}
\subsection{Заголовочный файл или блок}
\subsection{Стили документа}
\subsection{Пакеты}
\subsection{Автор и название}
\subsection{Верстка титульных страниц}
\subsection{Оглавление}
\section{Верстка слайдов}
\section{Список литературы и цитирование}
\section{Команды секционирования: часть, глава, раздел,..}
\section{Таблицы}
\section{Формулы}
\section{Перекрестные ссылки и гипессылки}
\section{Листинги скриптов и текстовых данных}
\section{Подготовка иллюстраций}
\subsection{Графики GNUPLOT}
\subsection{Схемы и графы в GraphViz}

\end{document}
