\documentclass[oneside,12pt]{book}

% e-book format
\usepackage[paperwidth=210mm,paperheight=148mm,margin=10mm]{geometry}

% Cyrillization
\usepackage[T1,T2A]{fontenc}
\usepackage[utf8]{inputenc}
\usepackage[english,russian]{babel}
\usepackage{indentfirst}

% font setup for screen reading
\renewcommand{\familydefault}{\sfdefault}
\normalfont

% pdflatex options
\usepackage[unicode,colorlinks,linkcolor=blue,bookmarks=true]{hyperref}
\usepackage[pdftex]{graphicx}

% listings
\usepackage{verbatim}
\usepackage{listings}
\lstset{
extendedchars=true,inputencoding=utf8, % i18n
frame=single, % show frames around
numbers=left, numberstyle=\small,numbersep=1mm,% line numbering
tabsize=4, % tab style
% keywordstyle=\color{blue}\texttt,
% commentstyle=\color{magenta}\texttt,
% showspaces=false
}
\newcommand{\lst}[3]{\lstinputlisting[title=\href{#2}{#1}]{#3}}

% software menu & keys
\usepackage[os=win]{menukeys} 
\usepackage{amssymb} % windows key
\newcommand{\winstart}{$\boxplus$}
\newcommand{\winr}{\keys{\winstart+R}}
\newcommand{\file}[1]{\textbf{\textsf{#1}}}
\newcommand{\lms}{$\lhd$}
\newcommand{\dblms}{$\lhd\lhd$}
\newcommand{\rms}{$\rhd$}
\newcommand{\checkbox}{$\boxtimes$}
\newcommand{\uncheckbox}{$\square$}

% disable oneliner page breaks
\usepackage[defaultlines=2,all]{nowidow}

% books bib management
\usepackage{biblatex}
\addbibresource{../bib/python.bib}
\addbibresource{../bib/eskd.bib}
\addbibresource{../bib/electronics.bib}
\addbibresource{../bib/latex.bib}
\addbibresource{../bib/sat.bib}


% extra char sets
\usepackage{wasysym} % smileys

% set lists style
% \usepackage{enumitem}
% \setlist{nosep}

% misc

% \usepackage{titling}

\newcommand{\email}[1]{$<$\href{mailto:#1}{#1}$>$}
\newcommand{\internet}{Internet}

\newcommand{\cm}[1]{Cortex-M#1}
\newcommand{\cmx}{\cm{x}}

\newcommand{\linux}{Linux}
\newcommand{\emlinux}{em\linux}

\newcommand{\cpp}{$C^{+}_{+}$}
\newcommand{\py}{Python}

\newcommand{\vcs}{\hyperref[vcs]{VCS}}
\newcommand{\make}{\hyperref[make]{Make}}
\newcommand{\spice}{ngSPICE}
\newcommand{\latex}{\LaTeX}

\newcommand{\eclipse}{\textcircled{$\equiv$}\textsc{eclipse}}
\newcommand{\vim}{(g)Vim}

\newcommand{\note}[1]{\footnote{\ #1}}
\newcommand{\cp}[1]{\note{копипаста \url{#1}}}

\newcommand{\win}{\winstart Windows}

\newcommand{\mk}{МК}

\newcommand{\term}[1]{\emph{#1}}

\newcommand{\pref}[1]{/стр.\pageref{#1}/}

% math
\usepackage{cancel}

% titles

\hypersetup{
	pdftitle={Азбука халтурщика-ARMатурщика},
	pdfauthor={ruOpenWrt, HackSpace <<Чебураторный завод>>, Консорциум хоббитов
	России, Bill Collis (Часть 1)}, 
	pdfsubject={https://github.com/ponyatov/Azbuka}
}

\begin{document}
\begin{titlepage}

\noindent
\includegraphics[height=0.4\textheight]{logo/CHBZ.png}
\hspace{1cm}
\includegraphics[height=0.15\textheight]{logo/LinuxPowered.png}
\hspace{1cm}
\includegraphics[height=0.4\textheight]{logo/OpenHardware.png}

\begin{centering}

\begin{framed}
\begin{framed}
{\Huge \textbf{Азбука ARMатурщика}}
\end{framed}

{\large \textsc{основы бытовой автоматики, электроники и систем управления}}
\end{framed}

\end{centering}

\bigskip
\begin{tabular}{p{1cm} l}
{\Large \copyright}
& Консорциум хоббитов России \\
& HackSpace
<<\href{https://github.com/ponyatov/CHBZ/raw/master/presentation.pdf}{Чебураторный
завод}>> \\
& Дмитрий Понятов \email{dponyatov@gmail.com} \\
& группа \href{https://groups.google.com/forum/\#!forum/openwrt2ru}{ruOpenWrt}\\
& Bill Collis (\ref{bcollis}) \\
& Andreas Fester (\ref{spice})\\
& Joe Martin, Craig Libuse (\ref{tabletop}) \\
\end{tabular}

\end{titlepage}
\tableofcontents\clearpage

Linux для встраиваемых систем\note{будем называть его \emlinux}\ ---
популярный метод быстрого создания комплекса ПО для больших сложных приложений,
работающих на достаточно мощном железе, особенно предполагающих интенсивное
использование сетевых технологий.

За счет использования уже существующей и очень большой базы исходных текстов
ядра, библиотек и программ для \linux, бесплатно доступных в т.ч. и для
коммерческих приложений, можно на порядки сократить стоимость разработки
собственных программных компонентов, и при этом получить готовую команду
бесплатных стронних разработчиков, уже знакомых с созданием ПО для \linux.

Из недостатков можно отметить:
\begin{itemize}
  \item Отсутствие полноценной поддержки режима жесткого реального времени;
  \item Тяжелое ядро;
  \begin{itemize}
  \item Поддерживаются только мощные семейства процессоров\note{32-бит,
  необходим блок MMU};
  \item Значительные требования по объему \ram\ и общей производительности;
  \end{itemize}
  \item Дремучесть техспециалистов, контуженных ТурбоПаскалем и
Win\-dows\-ом;
\end{itemize}

Для \term{сборки}\ \emlinux-системы используется метод \term{кросс-компиляции},
когда используется \term{кросс-тулчейн}, компилирующий весь комплект ПО для
компьютера с другой архитектурой. Типичный пример\ --- сборка ПО на
ПК c процессором Intel i7 для Raspberry Pi или планшета на процессоре
AllWinner/Tegra/\ldots.

\emlinux\ очень широко применяется на рынке мобильных устройств\note{в т.ч.
является основой Android}, и устройств интенсивно использующих сетевые протоколы
(роутеры, медиацентры).

В качествe примера применения рассмотрим относительно простое приложение:
многофункциональные настенные часы с синхронизацией времени через \internet, с
будильником, медиапроигрывателем, блэкджеком и плюшками.


\secrel{Введение в практическую электронику} \label{bcollis}\secdown
\input{bcollis/thanks}
\chapter{1 Введение в практическую электронику 13}

Эта книга\note{оригинал: B.Collis The Introduction to Practical 
Electronics\ldots}\ имеет слеующий ряд основных направлений:

\begin{itemize}
  \item Распознавание электронных компонентов и их правильное использование
  \item Наработка цельного набора компетенций в базовой электронике
  \item Использование макетных плат
  \item Навыки ручной пайки
  \item Использование закона Ома для выбора токоограничивающих резисторов
  \item Делитель напряжения
  \item Использование EDA CAD\note{\keys{E}\,lectronic \keys{D}\,esign
  \keys{A}\,utomation, САПР автоматизации проектирования электроники}\ для
  разработки и подготовки производства печатных плат
  \item Программирование микроконтроллеров и их сопряжение
  \item Транзистор в ключевом режиме
  \item Теория источников питания
  \item Принципы и схемы электропривода
  \item Моделирование решений через тестирование и испытания
  \item Следование кодексу практики
  \item Безопасные приемы работы
\end{itemize}

\section{Ваше обучение по специальности <<Технология>>}

\begin{itemize}

\item \textbf{Технологическая практика}

\begin{itemize}

\item\textbf{Быть четким}: разработка четких спецификаций для ваших
технологических проектов.

\item\textbf{Планирование}: думать прежде чем делать, и использовать во время
работы наброски типа блок-схем, принципиальных схем, чертежей разводки плат,
диаграмм и эскизов.

\item\textbf{Работа на результат}: испытания, тестирование и сборка электронных
схем, проектирование и изготовление печатных плат, написание программ для
микроконтроллеров.

\end{itemize}

\item \textbf{Технологические знания}

\begin{itemize}

\item\textbf{Технологическое моделирование}: прежде чем строить электронное
устройство, важно понять как оно работает сначала путем моделирования и/или
тестирования аппаратного и программного обеспечения.

\item\textbf{Технологические продукты}: знания о компонентах и ​​их
характеристиках.

\item\textbf{Технологические системы}: электронное устройство является более,
чем набором компонентов, это функционирующая система с входами, выходами и
контролирующим процессом.

\end{itemize}

\item \textbf{Природа технологии}

\begin{itemize}

\item\textbf{Характеристики технологических достижений}: знания об электронных
компонентах, особенно микроконтроллерах как основа современных технологий.

\item\textbf{Роль технологии}: электронные устройства в настоящее время играют
центральную роль в инфраструктуре нашего современного общества; мы их хозяева,
как они изменили нашу жизнь?

\end{itemize}

\end{itemize}

\section{Ключевые компетенции Ново-Зеландской программы}

\begin{itemize}

\item\textbf{Размышление}: для меня задачей технологии является все что касается
размышления. Моя цель: заставить студентов понимать технологии, использованные в
электронных устройствах. Для достижения этой цели студенты должны активно
взаимодействовать с их работой на самом раннем этапе, чтобы они могли построить
свое собственное понимание и пойти дальше, чтобы стать хорошими решалами
проблем. В начале своего обучения в электронике это требует от студентов
понимания инструкций, которые им дают и поиск ясности, когда они не понимают их.
Для этого многие новые и различные элементы знаний рассматриваются на занятиях,
и студентам выдаются задания на решение проблем чтобы помочь им мыслить
логически. Копирование чужого ответа является ошибочным, но приветствуется
совместная работа. В основе обучения лежит построение правильных концептуальных
моделей и анализ в контексте "большой картины".

\item\textbf{Relating to others} – working together in pairs and groups is as
essential in the classroom as it is in any other situation in life; we all have to share and negotiate resources and equipment
with others; it is essential therefore to actively communicate with each other and assist one
other.

\item\textbf{Using language symbols and texts} – At the heart of our subject is
the language we use for communicating electronic circuits, concepts, algorithms and computer programming syntax; so
the ability to recognise and using symbols and diagrams correctly for the work we do is vital.

\item\textbf{Managing self} – This is about students taking personal
responsibility for their own learning; it is about challenging students who expect to read answers in a book or have a teacher tell them
what to do. It means that students need to engage with the material in front of them.
Sometimes the answers will come easily, sometimes they will not; often our subject involves a
lot of trial and error (mostly error). Students should know that it is in the tough times that the
most is learnt. And not to give up keep searching for understanding.

\item\textbf{Participating and contributing} – We live in a world that is
incredibly dependent upon technology especially electronics, students need to develop an awareness of the importance
of this area of human creativity to our daily lives and to recognise that our projects have a
social function as well as a technical one.

\end{itemize}


\chapter{2 Вводная электронная схема 15}

\section{2.1 Где купить комплектующие? 15}

В Новой Зеландии есть некоторое количество отличных поставщиков компонентов с
разумными ценами, включающих \url{www.surplustronics.co.nz}, и
\url{www.activecomponents.com}. Зарубежные поставщики, которых я использую,
включают \url{www.digikey.co.nz}, \url{www.sparkfun.com}, \url{ebay.com}
и \url{aliexpress.com}

\bigskip
\includegraphics[width=0.8\textwidth]{bcollis/breadboard2.jpg}

Макетная плата (breadboard)\ --- пластмассовый блок с отверстиями и
металлическими полосковыми металлическими зажимами, создающими соединения между
элементами схемы. Отверстия расположены так, что компоненты могут быть соединены
вместе формируя схему. Верхние и нижние ряды, как правило, используется для шин
питания, красный сверху для плюса, и внизу черный/синий для для минуса (общий
провод).

\includegraphics[width=0.8\textwidth]{bcollis/BreadboardConnections.jpg}

\bigskip

Эта схема \ref{ch21sch}\ может быть собрана вот так \ref{ch21lay}, обратите
внимание, что светодиод должен находиться в правильном положении. Если у вас
есть светодиод и резистор, соединенные в замкнутый контур, светодиод должен
загореться.

\bigskip
Принципиальная схема \label{ch21sch}

\bigskip
Компоновка \label{ch21lay}

\bigskip
The LED requires 2V the battery is 9V, if you put the LED across the battery it
would stop working! So a 1k (1000ohm) resistor is used to reduce the voltage to
the LED and the current through it, get a multimeter and measure the voltage
across the resistor, is it close to 7V? If you disconnect any wire within the
circuit it stops working, a circuit needs to be complete before electrons can
flow.

\section{2.2 Определение сопротивления резистора по цветовому коду 16}

.

\section{2.3 Светодиоды 17}

.

\section{2,4 Некоторые технические характеристики светодиода 17}

.

\section{2.5 Задание на исследование светодиода 17}

.

\section{2.6 Добавление выключателя  в схему 18}

.

\section{2.7 Задание на установку выключателя 18}

.

\section{2,8 Важные понятия схемы 19}

.

\section{2.9 Изменение величины сопротивления 19}

.

\section{2.10 Добавление транзистора в схему 20}

.

\section{2.11 Чтение схем 21}

.

\section{2.12 Входная цепь\ --- LDR 22}

.

\section{2.13 Рабочая схема датчика темноты 23}

.

\section{2,14 Защитные цепи - использование диода 24}

.

\section{2.15 Задача исследования диода 24}

.

\section{2.13 Финальная схема датчика темноты 23}

.



% \chapter{3 Вводное конструирование печатной платы 26}
% % 3.1 Eagle Schematic and Layout Editor Tutorial  26
% % 3.2 An Introduction to Eagle  27
% % 3.3 The Schematic Editor  28
% % 3.4 The Board Editor . 33
% % 3.5 Making Negative Printouts . 37
% % 3.6 PCB Making  38
% 
% \chapter{4 Пайка, припой и паяльники 41}
% % 4.1 Soldering facts . 42
% % 4.2 Soldering Safety  42
% % 4.3 Soldering wires to switches  43
% % 4.4 Codes of practice  44
% % 4.5 Good and bad solder joints  45
% % 4.6 Short circuits . 46
% % 4.7 Soldering wires to LED’s . 48
% \chapter{5 Введение в теорию электроники 49}
% % 5.1 Making electricity  49
% % 5.2 ESD electrostatic discharge  51
% % 5.3 Magnets, wires and motion  52
% % 5.4 Group Power Assignment  52
% % 5.5 Electricity supply in New Zealand  53
% % 5.6 Conductors  54
% % 5.7 Insulators  54
% % 5.8 Choosing the right wire  55
% % 5.9 Resistors . 56
% % 5.10 Resistor Assignment . 56
% % 5.11 Resistivity  56
% % 5.12 Resistor prefixes . 57
% % 5.13 Resistor Values Exercises . 58
% % 5.14 Capacitors  60
% % 5.15 Component symbols reference  61
% % 5.16 Year 10/11 - Typical test questions so far  62
% \chapter{6 Введение в электронику микроконтроллера 63}
% % 6.1 What is a computer? . 64
% % 6.2 What does a computer system do?  64
% % 6.3 What does a microcontroller system do? . 65
% % 6.4 What exactly is a microcontroller?  66
% % 6.5 Getting started with AVR Programming  67
% % 6.6 Breadboard  67
% % 6.7 Breadboard+Prototyping board circuit  68
% % 6.8 Checking your workmanship . 70
% % 6.9 Getting started with Bascom & AVR  71
% % 6.10 The compiler . 71
% % 6.11 The programmer . 71
% % 6.12 USBASP programming cable  72
% % 6.13 Your first circuit  73
% % 6.14 An introduction to flowcharts . 74
% % 6.15 Bascom output commands . 75
% % 6.16 Exercises  76
% % 6.17 Two delays . 77
% % 6.18 Syntax errors -‘bugs’ . 78
% % 6.19 Microcontroller portswrite a Knightrider program using LED’s  79
% % 6.20 Knightrider v2  80
% % 6.21 Knightrider v3  81
% % 6.22 Commenting your programs  83
% % 6.23 Learning review  83
% % 6.24 What is a piezo and how does it make sound?. 84
% % 6.25 Sounding Off . 85
% % 6.26 Sound exercises . 87
% % 6.27 Amp it up . 88
% \chapter{7 Входные цепи микроконтроллера 91}
% % 7.1 Single push button switch  91
% % 7.2 Pullup resistor theory  93
% % 7.3 Switch in a breadboard circuit  93
% % 7.4 Checking switches in your program  94
% % 7.5 Program Logic – the ‘If-Then’ Switch Test  95
% % 7.6 If-then exercises  96
% % 7.7 Switch contact bounce . 97
% % 7.8 Reading multiple switches  99
% % 7.9 Bascom debounce command . 100
% % 7.10 Different types of switches you can use  101
% % 7.11 Reflective opto switch  102
% \chapter{8 Обзор программирования 104}
% % 8.1 Three steps to help you write good programs  104
% % 8.2 Saving Programs . 104
% % 8.3 Organisation is everything  104
% % 8.4 Programming template  105
% % 8.5 What you do when learning to program . 106
% % 8.6 AVR microcontroller hardware  107
% % 8.7 Power supplies 107
% % 8.8 Programming words you need to be able to use correctly  110
% % 8.9 Year10/11 typical test questions so far  111
% \chapter{9 Введение в поток выполнения программы 112}
% % 9.1 Pedestrian crossing lights controller  112
% % 9.2 Pedestrian Crossing Lights schematic  113
% % 9.3 Pedestrian Crossing Lights PCB Layout 114
% % 9.4 Algorithm planning example – pedestrian crossing lights . 115
% % 9.5 Flowchart planning example – pedestrian crossing lights . 116
% % 9.6 Getting started code  117
% % 9.7 Modification exercise for the pedestrian crossing . 117
% % 9.8 Traffic lights program flow  118
% \chapter{10 Вводное программирование: использование подпрограмм 126}
% % 10.1 Sending Morse code  127
% % 10.2 LM386 audio amplifier PCB. 130
% % 10.3 LM386 PCB Layout . 132
% \chapter{11 Вводное программирование: Использование переменных 134}
% % 11.1 Stepping or counting using variables  135
% % 11.2 For-Next  137
% % 11.3 Siren sound - programming using variables  139
% % 11.4 Make a simple siren  141
% % 11.5 Siren exercise  142
% % 11.6 A note about layout of program code . 143
% % 11.7 Using variables for data  144
% % 11.8 Different types of variables  145
% % 11.9 Variables and their uses  146
% % 11.10 Vehicle counter  147
% % 11.11 Rules about variables  148
% % 11.12 Examples of variables in use . 148
% % 11.13 Byte variable limitations  149
% % 11.14 Random Numbers  150
% % 11.15 The Bascom-AVR simulator  151
% % 11.16 Electronic dice project  152
% % 11.17 Programming using variables – dice  152
% % 11.18 Dice layout stage 1 . 153
% % 11.19 Dice layout stage 2 . 154
% % 11.20 Dice Layout final 155
% % 11.21 First Dice Program flowchart . 156
% % 11.22 A note about the Bascom Rnd command  157
% % 11.23 Modified dice . 158
% % 11.24 Modified Knightrider  160
% \chapter{12 Основные дисплеи 161}
% % 12.1 7 segment displays . 161
% % 12.2 Alphanumeric LED displays  172
% \chapter{13 Проект портативного аудиоусилителя на TDA2822M 174}
% % 13.1 Portfolio Assessment Schedule . 175
% % 13.2 Initial One Page Brief . 176
% % 13.3 TDA2822M specifications  177
% % 13.4 Making a PCB for the TDA2822 Amp Project . 178
% % 13.5 Extra PCB making information  182
% % 13.6 Component Forming Codes of Practice. 183
% % 13.7 TDA2811 wiring diagram  184
% % 13.8 SKETCHUP Quick Start Tutorial  185
% % 13.9 Creating reusable components in SketchUp  186
% \chapter{14 Основы логического программирования 187}
% % 14.1 Quiz Game Controller  187
% % 14.2 Quiz game controller system context diagram  188
% % 14.3 Quiz game controller block diagram  188
% % 14.4 Quiz game controller Algorithm . 190
% % 14.5 Quiz game schematic . 191
% % 14.6 Quiz game board veroboard layout . 192
% % 14.7 Quiz Controller flowchart  196
% % 14.8 'Quiz Controller program code 197
% % 14.9 Don’t delay - use logic  199
% \chapter{15 Разработка алгоритма: Система сигнализации 202}
% % 15.1 Simple alarm system – stage 1 . 202
% % 15.2 Alarm System Schematic . 203
% % 15.3 A simple alarm system – stage 2 208
% % 15.4 A simple alarm system – stage 3 209
% % 15.5 A simple alarm system – stage 4 210
% % 15.6 More complex alarm system . 211
% % 15.7 Alarm unit algorithm 5 212
% % 15.8 Alarm 6 algorithm 213
% \chapter{16 Основы теории цепей постоянного тока 215}
% % 16.1 Conventional Current . 215
% % 16.2 Ground  215
% % 16.3 Preferred resistor values  215
% % 16.4 Resistor Tolerances  216
% % 16.5 Combining resistors in series . 216
% % 16.6 Combining resistors in parallel  217
% % 16.7 Resistor Combination Circuits  218
% % 16.8 Multimeters . 219
% % 16.9 Multimeter controls  220
% % 16.10 Choosing correct meter settings  221
% % 16.11 Ohms law  222
% % 16.12 Voltage & Current Measurements  223
% % 16.14 Continuity  224
% % 16.15 Variable Resistors  225
% % 16.16 Capacitors  226
% % 16.17 Capacitor Codes and Values . 226
% % 16.18 Converting Capacitor Values uF, nF , pF  226
% % 16.19 Capacitor action in DC circuits . 227
% % 16.20 The Voltage Divider  228
% % 16.21 Using semiconductors  229
% % 16.22 Calculating current limit resistors for an LED . 230
% % 16.23 The Bipolar Junction Transistor  231
% % 16.24 Transistor Specifications Assignment  232
% % 16.25 Transistor Case styles  232
% % 16.26 Transistor amplifier in a microcontroller circuit  232
% % 16.27 Transistor Audio Amplifier  233
% % 16.28 Speakers . 234
% % 16.29 Switch types and symbols  235
% \chapter{17 Основы планирования проекта 236}
% % 17.1 System Designer  237
% % 17.2 Project mind map . 241
% % 17.3 Project timeline . 243
% % 17.4 System context diagram  245
% % 17.5 Block Diagram 256
% % 17.6 Board Layouts  258
% % 17.7 Algorithm design  263
% % 17.8 Flowcharts  265
% \chapter{18 Пример дизайна системы: Таймер клеевого пистолета 268}
% % 18.1 System context diagram  268
% % 18.2 Hot glue gun timer block diagram . 269
% % 18.3 Hot glue gun timer algorithm . 270
% % 18.4 Hot glue gun timer flowchart  271
% % 18.5 Hot glue gun timer program. 272
% \chapter{19 Основные интерфейсы и их программирование 273}
% % 19.1 Parallel data communications  274
% % 19.2 LCDs (liquid crystal displays) . 275
% % 19.3 Alphanumeric LCDs  276
% % 19.4 ATTINY461 Development PCB with LCD . 277
% % 19.5 Completing the wiring for the LCD 279
% % 19.6 LCD Contrast Control  280
% % 19.7 Learning to use the LCD  281
% % 19.8 Repetition again - the ‘For-Next’ and the LCD  282
% % 19.9 LCD Exerises . 283
% % 19.10 Defining your own LCD characters . 286
% % 19.11 LCD custom character program  286
% % 19.12 A simple digital clock . 288
% % 19.13 Adding more interfaces to the ATTiny461 Development board . 290
% % 19.14 Ohms law in action – a multicoloured LED . 292
% \chapter{20 Основы интерфейса аналого-цифрового преобразования 295}
% % 20.1 ADC - Analog to Digital conversion . 295
% % 20.2 Light level sensing  295
% % 20.3 Voltage dividers review . 296
% % 20.4 AVR ADC connections  296
% % 20.5 Select-Case  297
% % 20.6 Reading an LDR’s values . 299
% % 20.7 Marcus’ year10 night light project . 301
% % 20.8 Temperature measurement using the LM35  304
% % 20.9 A simple temperature display . 305
% % 20.10 LM35 temperature display  309
% % 20.11 Force Sensitive Resistors  312
% % 20.12 Piezo sensor . 312
% % 20.13 Multiple switches and ADC . 313
% \chapter{21 Основы проектирования системы 314}
% % 21.1 Understanding how systems are put together  314
% % 21.2 Food Processor system block diagram . 314
% % 21.3 Subsystems  314
% % 21.4 Food Processor system functional attributes - algorithm  314
% % 21.5 Food Processor system flowchart  315
% % 21.6 Toaster Design 316
% % 21.7 Toaster - system block diagram  316
% % 21.8 Toaster Algortihm . 316
% \chapter{22 Основы проектирования системы: Тайм-трекер 317}
% % 22.1 System context diagram and brief  318
% % 22.2 Time tracker block diagram . 319
% % 22.3 Algorithm development . 320
% % 22.4 Schematic  320
% % 22.5 Time tracker flowchart and program version 1 321
% % 22.6 Time Tracker stage 2 . 322
% % 22.7 Time Tracker stage 3 . 324
% % 22.8 Time Tracker stage 4 . 326
% \chapter{23 Основы вычислений времени 330}
% % 23.1 Ohms law calculator  330
% % 23.2 more maths - multiplication . 335
% % 23.3 Algorithms for multiplication of very large numbers . 337
% % 23.4 Program ideas - algorithm and flowchart exercises . 339
% \chapter{24 Основы строковых переменных 340}
% % 24.1 Strings assignment . 342
% % 24.2 ASCII Assignment  344
% % 24.3 Time in a string . 347
% % 24.4 Date in a string  349
% % 24.5 Scrolling message assignment  351
% % 24.6 Some LCD programming exercises.  352
% \chapter{25 Cиловые интерфейсы 353}
% % 25.1 Microcontroller power limitations  353
% % 25.2 Power  355
% % 25.3 Power dissipation in resistors . 355
% % 25.4 Diode characteristics  356
% % 25.5 Using Zener diodes . 357
% % 25.6 How diodes work  358
% % 25.7 How does a LED give off light? . 359
% % 25.8 LCD Backlight Data . 360
% % 25.9 Transistors as power switches  361
% % 25.10 High power loads  362
% % 25.11 AVR Power matters  362
% % 25.12 Darlington transistors - high power . 364
% % 25.13 ULN2803 Octal Darlington Driver  366
% % 25.14 Connecting a FET backlight control to your microcontroller  368
% % 25.15 FET backlight control  369
% \chapter{26 Теория источников питания 370}
% % 26.1 Typical PSUs . 371
% % 26.2 The four stages of a PSU (power supply unit)  372
% % 26.3 Stage 1step down transformer  372
% % 26.4 Stage 2AC to DC Conversion  374
% % 26.5 Stage 3Filtering AC component . 375
% % 26.6 Stage 4Voltage Regulation  375
% % 26.7 Ripple (decibel & dB) . 379
% % 26.8 Line Regulation . 380
% % 26.9 Load Regulation  380
% % 26.10 Current Limit . 381
% % 26.11 Power, temperature and heatsinking . 384
% % 26.12 Typical PSU circuit designs  386
% % 26.13 PSU block diagram  386
% % 26.14 PSU Schematic  386
% % 26.15 Practical current limit circuit.  389
% % 26.16 Voltage measurement using a voltage divider  391
% % 26.17 Variable power supply voltmeter program  393
% \chapter{27 Типичные вопросы тестирования 2011/12/13 годов 395}
% \chapter{28 Расширенное программирование: Массивы 397}
% \chapter{29 Подтягивающие резисторы AVR 402}
% \chapter{30 Дополнительноподключение клавиатуры 403}
% % 30.1 Keypad program 1  403
% % 30.2 Keypad program 2  405
% % 30.3 Keypad program 3 – cursor control . 406
% % 30.4 Keypad texter program V1  409
% % 30.5 Keypad texter program 1a  413
% % 30.6 ADC keypad interface  414
% \chapter{31 Тонкости циклов Do-Loop \& While-Wend 417}
% % 31.1 While-Wend or Do-Loop-Until or For-Next? . 418
% \chapter{32 Подключение двигателя постоянного тока 423}
% % 32.1 H-мост 425
% % 32.2 H-Bridge Braking  427
% % 32.3 L293D H-Bridge IC  428
% % 32.4 L298 H-Bridge IC  430
% % 32.5 LMD18200 H-Bridge IC . 431
% % 32.6 LMD18200 program  434
% % 32.7 Darlington H-Bridge  435
% % 32.8 Stepper motors . 438
% % 32.9 PWM - pulse width modulation  445
% % 32.10 PWM outputs  446
% % 32.11 Uses for PWM  447
% % 32.12 ATMEL AVRs PWM pins . 448
% % 32.13 PWM on any port  449
% % 32.14 PWM internals  450
% \chapter{33 Пример расширенной системы: Будильник 452}
% % 33.2 Analogue seconds display on an LCD  457
% % 33.3 LCD big digits  460
% \chapter{34 Резистивный сенсорный экран 468}
% % 34.1 Keeping control so you dont lose your ‘stack’ . 474
% \chapter{35 Пример проектирования системы: Регулятор температуры 475}
% \chapter{36 Расширенное программирование: Машины состояний 478}
% % 36.1 Daily routine state machine . 478
% % 36.2 Truck driving state machine  480
% % 36.3 Developing a state machine  484
% % 36.4 A state machine for the temperature alarm system . 485
% % 36.5 Using System Designer software to design state machines  488
% % 36.6 State machine to program code  490
% % 36.7 The power of state machines over flowcharts  493
% % 36.8 Bike light – state machine example . 495
% % 36.9 Bike light program version1b  497
% % 36.10 Bike light program version2  499
% \chapter{37 Переработанный проект будильника 501}
% % 37.1 System Designer to develop a Product Brainstorm . 501
% % 37.2 Initial block diagram for the alarm clock  503
% % 37.3 A first (simple) algorithm is developed  505
% % 37.4 A statemachine for the first clock . 506
% % 37.5 Alarm clock state machine and code version 2 508
% % 37.6 Token game – state machine design example  509
% \chapter{38 Студенческий проект: Расширенный оконный контроллер 514}
% % 38.1 Window controller state machine #1  514
% % 38.2 Window controller state machine #3.  515
% % 38.3 Window controller state machine #5  516
% % 38.4 Window controller program . 517
% \chapter{39 Альтернативные техники кодирования машин состояния 524}
% \chapter{40 Сложно: последовательная связь 526}
% % 40.1 Simplex and duplex . 526
% % 40.2 Synchronous and asynchronous  526
% % 40.3 Serial communications, Bascom and the AVR  527
% % 40.4 RS232 serial communications  528
% % 40.5 Build your own RS232 buffer  530
% % 40.6 Talking to an AVR from Windows XP . 531
% % 40.7 Talking to an AVR from Win7 . 533
% % 40.8 First Bascom RS-232 program  535
% % 40.9 Receiving text from a PC . 536
% % 40.10 BASCOM serial commands  537
% % 40.11 Serial IO using Inkey()  538
% % 40.12 Creating your own software to communicate with the AVR  541
% % 40.13 Microsoft Visual Basic 2008 Express Edition . 542
% % 40.14 Stage 1 – GUI creation . 543
% % 40.15 Stage 2 – Coding and understanding event programming  552
% % 40.16 Microsoft Visual C# commport application  557
% % 40.17 Microcontroller with serial IO.  562
% % 40.18 PC software (C#) to communicate with the AVR . 567
% % 40.19 Using excel to capture serial data  571
% % 40.20 PLX-DAQ  573
% % 40.21 StampPlot  574
% % 40.22 Serial to parallel  576
% % 40.23 Keyboard interfacing – synchronous serial data  581
% % 40.24 Keyboard as asynchronous data . 588
% % 40.25 GPS  591
% \chapter{41 Цифровой радиоканал 597}
% % 41.1 An Introduction to data over radio  597
% % 41.2 HT12E Datasheet, transmission and timing  604
% % 41.3 HT12 test setup . 607
% % 41.4 HT12E Program  609
% % 41.5 HT12D datasheet . 610
% % 41.6 HT12D Program  612
% % 41.7 Replacing the HT12E encoding with software  613
% \chapter{42 Введение в I2C 617}
% % 42.1 I2C Real Time Clocks  618
% % 42.2 Real time clocks  619
% % 42.3 Connecting the RTC  619
% % 42.4 Connecting the RTC to the board . 619
% % 42.5 Internal features  620
% % 42.6 DS1307 RTC code  621
% % 42.7 DS1678 RTC code  626
% \chapter{43 Студенческий проект: Таймер полива теплицы 631}
% % 43.1 System block diagram  631
% % 43.2 State machine  631
% % 43.3 Program code  632
% \chapter{44 Проект Велосипедного аудиоусилителя 642}
% \chapter{45 Графические LCD 648}
% % 45.1 The T6963 controller  648
% % 45.2 Graphics LCD (128x64) – KS0108  653
% % 45.3 Generating a negative supply for a graphics LCD  658
% \chapter{46 Проект Отслеживания температуры GLCD 660}
% % 46.1 Project hardware  660
% % 46.2 Project software planning . 662
% % 46.3 Draw the graph scales  663
% % 46.4 Read the values  664
% % 46.5 Store the values  666
% % 46.6 Plot the values as a graph  667
% % 46.7 Full software listing . 669
% \chapter{47 Прерывания 672}
% % 47.1 Switch bounce problem investigation . 674
% % 47.2 Keypad- polling versus interrupt driven . 675
% % 47.3 Improving the HT12 radio system by using interrupts . 680
% % 47.4 Magnetic Card Reader  682
% % 47.5 Card reader data structure  682
% % 47.6 Card reader data timing  683
% % 47.7 Card reader data formats . 684
% % 47.8 Understanding interrupts in Bascom- trialling . 684
% % 47.9 Planning the program . 687
% % 47.10 Pin Change Interrupts PCINT0-31  690
% \chapter{48 Таймеры/Счётчики 692}
% % 48.1 Timer2 (16 bit) Program  693
% % 48.2 Timer0 (8bit) Program  694
% % 48.3 Accurate tones using a timer (Middle C) 695
% % 48.4 Timer1 Calculator Program . 696
% % 48.5 Timer code to make a siren by varying the preload value . 697
% \chapter{49 Проект скроллинга графического LED дисплея: массивы и таймеры 698}
% % 49.1 Scrolling text code  701
% % 49.2 Scrolling text – algorithm design  703
% % 49.3 Scrolling test - code  704
% \chapter{50 Проект медицинского прибора: реализация таймера 709}
% % 50.1 Block diagram  709
% % 50.2 Blower - state machine  710
% % 50.3 Blower program code . 711
% \chapter{Проект часов на 7-сегментном индикаторе\\
% реализация на сдвоенном таймере}
% % 51.1 Understanding the complexities of the situation  715
% % 51.2 Hardware understanding. 716
% % 51.3 Classroom clock – block diagram . 717
% % 51.4 Classroom clock - schematic  718
% % 51.5 Classroom clock - PCB layout  718
% % 51.6 Relay Circuit Example  719
% % 51.7 Classroom clock – flowcharts . 723
% % 51.8 Classroom clock – program . 724
% \chapter{52 ИС драйвера дисплея MAX 7219/7221 739}
% % 52.1 AVR clock/oscillator  743
% \chapter{53 Подключение через мобильную связь: ADH8066 744}
% % 53.1 ADH prototype development  745
% % 53.2 ADH initial test setup block diagram  747
% % 53.3 Process for using the ADH  748
% % 53.4 ADH communications . 750
% % 53.5 Initial state machine  751
% % 53.6 Status flags . 752
% % 53.7 Second state machine  753
% % 53.8 StateMachine 3 . 754
% % 53.9 Sending an SMS text . 755
% % 53.10 Receiving an SMS text . 756
% % 53.11 Splitting a large string (SMS message)  757
% % 53.12 Converting strings to numbers  760
% % 53.13 Full Program listing for SM3  761
% \chapter{54 Передача данных через \internet\ 778}
% % 54.1 IP address  779
% % 54.2 MAC (physical) address  779
% % 54.3 Subnet mask  780
% % 54.4 Ping  780
% % 54.5 Ports  781
% % 54.6 Packets . 781
% % 54.7 Gateway 782
% % 54.8 DNS  784
% % 54.9 WIZNET812  785
% % 54.10 Wiznet 812 Webserver V1  792
% % 54.11 Transmitting data  797
% % 54.12 Wiznet Server2 (version1)  809
% % 54.13 ‘Main do loop . 811
% % 54.14 process any messages received from browser . 812
% % 54.15 Served webpage . 814
% \chapter{55 Задание: математика в реальном мире 816}
% % 55.1 Math sssignment - part 1  819
% % 55.2 Math assignment - part 2 . 820
% % 55.3 Math assignment - part 3 . 821
% % 55.4 Math assignment - part 4 . 822
% % 55.5 Math assignment - part 5 . 823
% % 55.6 Math assignment - part 6 . 824
% % 55.7 Extension exercise  824
% \chapter{56 Цветной графический LCD на основе SSD1928 825}
% % 56.1 System block diagram  825
% % 56.2 TFT LCDs  826
% % 56.3 System memory requirements  827
% % 56.4 System speed  827
% % 56.5 SSD and HX ICs  827
% % 56.6 Colour capability  827
% % 56.7 SSD1928 and HX8238 control requirements  828
% % 56.8 SSD1928 Software . 829
% % 56.9 SSD1928 microcontroller hardware interface . 833
% % 56.10 Accessing SSD control registers . 834
% % 56.11 SSD1928_Register_routines.bas  836
% % 56.12 Accessing the HX8238.  840
% % 56.13 SSD1928_GPIO_routines.bas  840
% % 56.14 LCD timing signals . 842
% % 56.15 HX setups  843
% % 56.16 SSD setups  844
% % 56.17 SSD line / HSync timing  845
% % 56.18 SSD row / VSync/ frame timing  846
% % 56.19 HX and SSD setup routine . 848
% % 56.20 'SSD1928_HardwareSetup_Routines.bas  848
% % 56.21 SSD1928_Window_Control_Routines.bas . 852
% % 56.22 Colour data in the SSD memory  855
% % 56.23 Accessing the SSD1928 colour memory . 856
% % 56.24 'SSD1928_Memory_Routines.bas  856
% % 56.25 Drawing simple graphics . 858
% % 56.26 'SSD1928_Simple_Graphics_Routines.bas  858
% % 56.27 SSD1928_text_routines  861
% \chapter{57 Светофор: помощь и решение 865}
% \chapter{58 Компьютерное программирование: низкоуровневые детали 869}
% % 58.1 Low level languages869
% % 58.2 AVR Internals – how the microcontroller works . 870
% % 58.3 1. The 8bit data bus  871
% % 58.4 2. Memory  871
% % 58.5 3. Special Function registers  872
% % 58.6 A simple program to demonstrate the AVR in operation  872
% % 58.7 Bascom keyword reference . 874
% \chapter{59 USB-программатор: USBASP 876}
% \chapter{60 Программатор USBTinyISP 877}
% \chapter{61 Программирование на Си и AVR 881}
% % 61.1 Configuring a programmer  882
% % 61.2 First program . 884
% % 61.3 Output window  886
% % 61.4 Configuring inputs & outputs  887
% % 61.5 Making a single pin an input  888
% % 61.6 Making a single pin an output . 889
% % 61.7 Microcontroller type . 890
% % 61.8 Includes  890
% % 61.9 Main function . 891
% % 61.10 The blinkyelled program  892
% % 61.11 Counting your bytes  893
% % 61.12 Optimising your code  895
% % 61.13 Reading input switches  896
% % 61.14 Macros . 897
% % 61.15 Auto-generated config from System Designer  898
% % 61.16 Writing your own functions . 900
% % 61.17 AVR Studio editor features . 902
% % 61.18 AVR hardware registers  903
% % 61.19 Character LCD programming in C  904
% % 61.20 CharLCD.h Header file . 904
% % 61.21 Manipulating AVR register addresses  907
% % 61.22 Writing to the LCD  908
% % 61.23 Initialise the LCD . 910
% % 61.24 lcd commands  912
% % 61.25 Writing text to the LCD . 913
% % 61.26 Program Flash and Strings . 915
% % 61.27 LCD test program1 . 917
% % 61.28 CharLCD.h . 919
% % 61.29 CharLCD.c . 922
% \chapter{62 Объектно-Ориентированное Программирование (ООП) на \cpp\ и AVR 929}
% % 62.1 The black box concept  929
% % 62.2 The concept of a class  929
% % 62.3 First CPP program  930
% % 62.4 Creating an AVR CPP program in Atmel Studio 6  932
% % 62.5 Adding our class files to the project . 936
% % 62.6 First Input and output program  938
% % 62.7 Class OutputPin  940
% % 62.8 Class InputPin  940
% % 62.9 Inheritance  942
% % 62.10 Class IOPin  942
% % 62.11 Encapsulation  944
% % 62.12 Access within a class  944
% % 62.13 Class Char_LCD . 945
% % 62.14 Exercise – create your own Led class. . 950
% \chapter{63 Современные (2014) отладочные платы на AVR 953}
% % 63.1 Year10 ATTiny461 V4a development board  953
% % 63.2 Year11 ATTiny461 V7a development board  956
% % 63.3 Year 12 ATMega 20x4 Character LCD v6A . 959
% % 63.4 Year 13 ATMega GLCD 128x64 (2014) Veroboard . 961
% % 63.5 Year 13 ATMega GLCD (older pin connections)  966
% % 63.6 ATMEGA microcontroller pin connections  968
% % 63.7 ATMEGA16/644 40pin DIP package– pin connections . 969
% \chapter{64 Eagle: создание собственной библиотеки 970}
% % 64.1 Autorouting PCBS  977
% \chapter{65 Практические методы 979}
% % 65.1 PCB Mounting  979
% % 65.2 Countersink holes and joining MDF/wood  980
% % 65.3 MDF  981
% % 65.4 Plywood  981
% % 65.5 Acrylic  982
% % 65.6 Electrogalv  982
% % 65.7 Choosing fasteners . 983
% % 65.8 Workshop Machinery . 984
% % 65.9 Glues/Adhesives  986
% % 65.10 Wood Joining techniques  987
% % 65.11 Codes of Practice for student projects  988
% % 65.12 Fitness for purpose definitions and NZ legislation  989
% \chapter{66 ЧПУ 990}
% % 66.1 Machine overview  991
% % 66.2 Starting the CNC machine  992
% % 66.3 CamBam  993
% % 66.4 CamBam options . 993
% % 66.5 Drawing shapes in CamBam  994
% % 66.6 Machining commands  996
% % 66.7 A Box of Pi  997
% % 66.8 Holding Tabs  1003
% % 66.9 Engraving  1004
% % 66.10 Polylines  1005
% \chapter{67 Индекс 1008}

\secup


\part{Основы электроники}

Здесь идет список ссылок на онлайн лекции в edX, Coursera, и т.п.

\chapter{Линейные схемы на пассивных элементах, основы электротехники}

\chapter{Симуляция и расчет схем в \spice}

\secrel{САПР электронных устройств KiCAD}\label{kicad}\secdown\secdown

\noindent\includegraphics[height=0.5\textheight]{tmp/icon_kicad.png}

\cp{http://teholabs.com/knowledge/kicad.html}


\cp{http://ru.wikibooks.org/wiki/KiCad}

KiCad\ --- распространяемый по лицензии GNU GPL программный комплекс САПР EDA с
открытыми исходными текстами, предназначенный для разработки электрических схем
и печатных плат.

Кроссплатформенность компонентов KiCad обеспечивается использованием 
библиотеки wxWidgets. Поддерживаются операционные системы Linux, 
Windows NT 5.x, Free\-BSD и Solaris.

Разработчик\ --- Жан-Пьер Шарра (фр. Jean-Pierre Charras), исследователь 
в LIS (фр. Laboratoire des Images et des Signaux\ --- Лаборатория Изображений 
и Сигналов) и преподаватель электроники и обработки изображений в фр. 
IUT de Saint Martin d’Hères (Франция).

\cp{http://ru.wikibooks.org/wiki/KiCad/Miniurok}

Этот раздел познакомит Вас с основами использования системы KiCad. Он содержит
информацию о всех шагах создания простой печатной платы: от рисования
электрической схемы до печати готового рисунка платы. Вам будут представлены
различные возможности KiCad и предложены эффективные пути решения различных
задач.

Руководство пользователя, поставляемое вместе с KiCad, содержит значительно
больше информации, чем этот урок. Ознакомтесь с ним, чтобы узнать больше об
использовании программы.



\section{Установка под \win}\label{kicadinst}

\menu{\winr{\url{http://www.kicad-pcb.org/}}>Download>\winstart}

\menu{\winr{\url{http://kicad.nosoftware.cz/}}>
\file{KiCad\_testing\-201x.xx.xx-BZRxxxx\_Win\_full\_version.exe}}

\bigskip

\menu{Installer Language>\emph{English}>Ok} 
в русифицированном инсталляторе кривые шрифты

\menu{KiCAD 20xx.xx.xx Setup>Next}

\menu{Лицензия>Agree}

\menu{Components>\checkbox\ все>Next}

\menu{Location>\file{C:/KiCad}>Install}

\menu{Completing Setup>\checkbox Wings3D>Finish}

\subsection{Установка \prog{Wings3D}}

\menu{\winr{\url{http://www.wings3d.com/}}>Downloads>Stable Release>\win
(32/64b)}

\menu{file{wings-n.n.n.exe}}

\menu{Compononets>\checkbox QuickLaunch>Next}

\menu{Location>\file{C:/Program Files/wings3d\_n.n.n}>Next>Install>Close}% 


\clearpage\secrel{Маршрут проектирования}

\noindent
\includegraphics[height=0.92\textheight]{kicad/march.pdf}


\secrel{Создание проекта в менеджере проектов \prog{kicad}}

\win: \menu{\winstart>Программы>KiCAD>KiCAD}

\linux: \verb|user@host$ kicad|

\bigskip
\includegraphics[height=0.8\textheight]{kicad/projman.png}

\bigskip
В верхней части панели \term{менеджера проектов} \prog{kicad} имеются большие
кнопки запуска компонентов KiCad:

\begin{itemize}
\item\icon{kicad/icon_eeschema.png}
\prog{eeschema}\ --- Редактор принципиальных схем
\item\icon{kicad/icon_pcbnew.png}
\prog{pcbnew}\ --- Редактор печатных плат
\item\icon{kicad/icon_cvpcb.png}
\prog{cvpcb}\ --- Программа редактирования \termdef{падстеков}{падстек}
(отверстий и площадок)
\end{itemize}

Каждая кнопка запускает соответствующую программу. Мы будем использовать эти
программы по мере изучения.


\begin{itemize}
\item
\includegraphics[height=0.1\textheight]{kicad/icon_gerbview.png}
\prog{gerbview}\ --- Программа просмотра фотошаблонов в формате Gerber
\item
\prog{bitmap2component}\ --- Создание компонента из черно-белого изображения
(например логотипа)
\item
\includegraphics[height=0.1\textheight]{kicad/icon_pcbcalculator.png}
\prog{PcbCalculator}\ --- Калькулятор для печатных плат
\item
\includegraphics[height=0.1\textheight]{kicad/icon_pagelayout.png}
\prog{PageLayout}\ --- редактор формата листа схемы
\end{itemize}

% % \bigskip
% % Лучше всего для каждого проекта использовать раздельные папки; в противном
% % случае система может сбиться с толку, если файлы из разных проектов будут лежать
% % в одной папке. Проделайте следующие шаги:
% % 
% % % \begin{enumerate}
% % %   \item Создайте папку проекта \file{D:/ARM/SpindleDriver}
% % %   \item Запустите программу KiCad
% % %   \item Создайте проект (project)
% % %   \begin{itemize}
% % %     \item 
% % % На панели инструментов KiCad выберите левую иконку с подсказкой\\
% % % \menu{Начать новый проект}, используйте команду меню
% % % \menu{Файл>Новый>Пустой} или сочетание клавиш \keys{Ctrl+N}.
% % %     \item 
% % % В диалоге \menu{Создать новый проект} выберите созданную папку
% % % выберите только что созданную папку \file{D:/ARM/SpindleDriver} и
% % % введите имя проекта \menu{\file{SpindleDriver}} и нажмите \menu{Сохранить}.
% % % 	\item
% % % Если папка проекта содержит какие-то файлы, будет выведено окно выбора:
% % % создать подпапку с именем проекта \menu{Yes}, или записать файл проекта
% % % в указанную папку как есть \menu{No}. Нажмите No.
% % %     \item 
% % % Сохраните проект кнопкой \menu{Сохранить текущий проект}, \menu{Файл>Сохранить}
% % % или \keys{Ctrl+S}.
% % % 	\item
% % % В папке появится файл \file{SpindleDriver.pro}, содержащий установки вашего 
% % % проекта. Файл имеет тектовый формат, поэтому при необходимости его можно открыть
% % % в любом редакторе и вручную аккуратно подправить, например скорректировать
% % % настройки зазоров печатной платы.
% % %   \end{itemize}
% % % \end{enumerate}

\secrel{Создание принципиальной схемы в \prog{eeschema}\ (часть 1)}


Запустите из менеджера проектов, графической оболочки или командной строки
\linux\ модуль \eeschema: 

\bigskip
\noindent\verb|user@host$ eeschema &|.

\bigskip
\noindent\includegraphics[height=0.1\textheight]{kicad/icon_eeschema.png}

\clearpage\noindent
\includegraphics[width=\textwidth]{kicad/ee15.png}

На правом краю окна редактора схем есть вертикальная панель инструментов,
которые мы и будем использовать для рисования схемы. Этими инструментами можно
выбирать объекты, размещать компоненты, вводить связи и т.д.


% % % При первом запуске \prog{eeschema}\ стартует с новым проектом и
% % % показывает предупреждение, что файла схемы еще нет. Просто нажмите \menu{ОК}.
% % 
% % % Если вас не устраивает черный фон рабочец области или цвета элементов схемы,
% % % поменяйте настроки цветов \menu{Настройки>Цвета}.
% % % 
% % % Завершение работы инструмента: вы можете выбрать другой инструмент из правой
% % % инструментальной панели или же указать \menu{Отложить инструмент} по правому
% % % клику мышки \keys{\rms}.
% % 
% % \secrel{Инструмент \emph{Добавить компоненты}}
% % 
% % % \begin{itemize}
% % %   \item 
% % % На правой панели нажмите кнопку \menu{Разместить компонент}\
% % % \includegraphics[height=2em]{kicad/ee21.png}. Курсор изменится со стрелки на
% % % карандаш. Удобнее использовать сочетание клавиш \keys{Shift+A}.
% % % Кликните в поле схемы чтобы начать размещение компонента. Появится диалог
% % % \menu{Выбор компонента}. Вы можете выбрать компонент несколькими путями:
% % %   \item
% % %   \begin{enumerate}
% % %     \item 
% % % Если вы знаете точное имя копонента, введите его в поле \menu{Имя}, а
% % % затем нажмите \keys{Enter} или \keys{OK}.
% % %     \item 
% % % Если вы знаете имя только приблизительно, в поле \menu{Имя} введите шаблон для
% % % поиска, например, \menu{*BD*}, затем нажмите \keys{Enter} или \keys{OK}. Вы
% % % увидите окно \\\menu{Выбрать компонент} со списком найденных компонентов.
% % % 
% % % \includegraphics[height=0.5\textheight]{kicad/ee16.png}
% % %     \item 
% % % Вы можете искать компонент по ключевому слову, введя его в поле \menu{Имя},
% % % затем кликнув \menu{Поиск по ключевому слову}. Однако на данный момент качество
% % % библиотек все еще низкое, и немногие компоненты имеют ключевые слова, поэтому
% % % эта возможность полезна косвенно.
% % %     \item 
% % % Можно выбрать недавно использованные компоненты из \menu{Списка истории}.
% % %     \item 
% % % Кнопка \menu{Список всех} вызывает диалог, в котором можно выбрать сначала
% % % библиотеку \menu{74xx}, а затем ее компонент \menu{74HCT04}.
% % %     \item 
% % % Кнопка \menu{Выбор просмотром} вызывает \menu{Обзор библиотек}, позволяя
% % % просмотреть библиотеки и находящиеся в них условные графические изображения.
% % % 
% % % \includegraphics[height=0.5\textheight]{kicad/ee19.png}
% % % 
% % %   \end{enumerate} 
% % % \end{itemize}
% % % 
% % % Вы также можете
% % % вызвать обозреватель библиотек кнопкой\\
% % % \menu{Просмотр библиотек и
% % % компонентов}\ \includegraphics[height=2em]{kicad/ee20.png}
% % % 
% % % Выбрав элемент \dblms, вставьте символ в нужное место схемы \lms.
% % % Позже вы сможете переместить его если нужно.
% % % Зеркальное отражение компонента можно произвести следующим образом:
% % % 
% % % \begin{itemize}
% % %   \item Поместите курсор на компоненте.
% % %   \item По \rms\ выберите \menu{Ориентация компонента>Отражение}. 
% % %   \item Без использования \term{контекстного меню}\ --- наведите мышь на
% % %   компонент и нажмите кнопку \keys{X}\ или \keys{Y}.
% % % \end{itemize}


\input{kicad/eeschema}

% % \input{kicad/libs}
% % 
% % \section{\prog{gerbview}: просмотр фотошаблонов}

позволяет просматривать Gerber-файлы перед передачей печатных плат в
производство.

% % 
% % \secrel{Программа \prog{Wings3D} для создания 3D моделей}

Эта программа может вам пригодиться если вы планируете создавать 3D модели для PCB элементов.

Архив и файлы документации (Linux и Windows) в папке kicad/wings3d.

Взгляните на домашнюю страницу Wings3D чтобы узнать подробнее о программе.

pcbnew использует файлы в формате wrml (.wrl) экспортируемые из Wings3D (родной формат Wings3D - это .wings).

\secdown
\secrel{Установка \prog{Wings3D} под \win}

\menu{\winr{\url{http://www.wings3d.com/}}>Downloads>Stable Release>\win
(32/64b)}

\menu{file{wings-n.n.n.exe}}

\menu{Compononets>\checkbox QuickLaunch>Next}

\menu{Location>\file{C:/Program Files/wings3d\_n.n.n}>Next>Install>Close}% 
\secup


\secup
\secup


\chapter{Простейшие полупроводниковые элементы}

\section{Оптоэлектроника}

\section{Схемы на биполярных транзисорах} 

\section{Схемы на на полевых транзисорах}

\chapter{Операционные усилители}

\chapter{Источники питания}

\section{Батарейное питание}

\section{Линейные стабилизаторы}

\section{Импульсные преобразователи на ШИМ-контроллерах} 

\section{Цепи защиты и гашения кондуктивных помех}

\chapter{Цифровая электроника}

\chapter{Компьютерные интерфейсы}

\section{Поколение 90х: COM, LPT, ISA}

\subsection{Резервный программатор AVR ``пять проводков''}

\section{Сеть CAN}

\section{Интерфейсные модули USB}

\subsection{Универсальный высокоскоростной конвертер FTDI FT2232H}

\subsection{JTAG-адаптер}

\subsection{Отладочный модуль CAN}

\section{Интерфейсные модули Ethernet}

\chapter{ПЛИС}

\chapter{Датчики}

\chapter{Электропривод и исполнительные устройства}

\part{Основы конструирования РЭС}

\chapter{Пакеты моделирования на основе OpenFOAM}

\chapter{Обеспечение теплового режима}

\chapter{Электромагнитная совместимость}

\section{Кондуктивные помехи}

\section{Компоновочные модели и оптимизация кабельной сети}

\part{Технология РЭС}

\secrel{Инструменты и электронное оборудование}\secdown

\section{Радиомонтажный инструмент}

Пара надфилей, заточной камень на дрель, комплект сверел и несколько листов
наждачки.

\subsection{Pro'sKit}
Отдельного обзора заслуживает инструмент и наборы Pro'sKit
% \href{http://www.proskit.com/}{ProsKit}
%  / \href{http://www.proskit.msk.ru/index.html}{ru}.

\clearpage
\noindent\includegraphics[height=0.95\textheight]{tech/tools/proskit/PK5308BM.jpg}

\textbf{PK-5308BM универсальный набор инструментов}

\clearpage
\noindent\includegraphics[height=0.95\textheight]{tech/tools/proskit/1PK-616B.jpg}

\textbf{1PK-616B Набор инструментов для электроники профессиональный}

\clearpage\label{1PK-813B}
\noindent\includegraphics[height=0.95\textheight]{tech/tools/proskit/1PK-813B.jpg}

\textbf{1PK-813B Набор базовых инструментов для электроники}

\clearpage

По личному опыту: в 1PK-813B не хватает

\begin{itemize}
  \item мелкого мультиметра, 
  \item стриппера 1PK-3001E, 
  \item микрокусачек типа 8PK-30D, 
  \item канифоли, 
  \item ножа, 
  \item настроечную отвертку заменить индикаторной.
\end{itemize} 

\clearpage
\subsubsection{Инструмент до 1000\,В}

Для электромонтажных работ обязательно приобретите комплект
высоковольтного инструмента до 1000\,В:

\begin{tabular}{p{0.45\textwidth} p{0.45\textwidth}}
\noindent\includegraphics[width=0.45\textwidth]{tech/tools/proskit/PM-911.jpg}
&
\noindent\includegraphics[width=0.35\textwidth]{tech/tools/proskit/PM-917.jpg}
\\

\textbf{PM-911 Пассатижи 1\,кВ}
&
\textbf{PM-917 Кусачки (бокорезы) 1\,кВ}
\\
\end{tabular}
\clearpage

\subsubsection{Хранение}

\begin{tabular}{p{0.45\textwidth} p{0.45\textwidth}}
\noindent\includegraphics[width=0.45\textwidth]{tech/tools/proskit/103-132D.jpg}
&
\noindent\includegraphics[width=0.45\textwidth]{tech/tools/proskit/SB-3428SB.jpg}
\\
\textbf{103-132D Кассетница для деталей и компонентов}
&
\textbf{SB-3428SB Портативная кассетница для саморезов и т.п.}
\\
\end{tabular}
\clearpage

\subsubsection{Радиомонтаж}

\begin{tabular}{p{0.45\textwidth} p{0.45\textwidth}}
\noindent\includegraphics[width=0.45\textwidth]{tech/tools/proskit/8PK-30D.jpg}
&
\noindent\includegraphics[width=0.45\textwidth]{tech/tools/proskit/1PK-709.jpg}
\\
\textbf{8PK-30D Кусачки миниатюрные}
&
\textbf{1PK-709 Длинногубцы-кусачки}
\\
\end{tabular}
\clearpage

\begin{tabular}{p{0.45\textwidth} p{0.45\textwidth}}
\noindent\includegraphics[width=0.45\textwidth]{tech/tools/proskit/1PK-055S.jpg}
&
\noindent\includegraphics[width=0.45\textwidth]{tech/tools/proskit/1PK-29.jpg}
\\
\textbf{1PK-055S Длинногубцы изогнутые}
&
\textbf{1PK-29 Круглогубцы}
\\
\end{tabular}
\clearpage

\begin{tabular}{p{0.45\textwidth} p{0.45\textwidth}}
\noindent\includegraphics[width=0.45\textwidth]{tech/tools/proskit/1PK-101T.jpg}
&
\noindent\includegraphics[width=0.45\textwidth]{tech/tools/proskit/1PK-3001E.jpg}
\\
\textbf{1PK-101T Пинцет прямой}
&
\textbf{1PK-3001E Клещи для зачистки проводов прецизионные (стриппер)}
\\
\end{tabular}
\clearpage

\begin{tabular}{p{0.45\textwidth} p{0.45\textwidth}}
\noindent\includegraphics[width=0.45\textwidth]{tech/tools/proskit/PD-374.jpg}
&\\
\textbf{PD-374 Тиски на струбцине}
&\\
\end{tabular}
\clearpage

% \subsubsection{Прочие}
%
% Попалась интересная недорогая отвертка:
%
% \begin{multicols}{2}
% \noindent\includegraphics[width=0.90.3\textwidth]{tech/tools/proskit/P1020966.jpg}
%
% \noindent\includegraphics[width=0.90.3\textwidth]{tech/tools/proskit/P1020967.jpg}
% \end{multicols}
%
% Фиксация четкая, исполнение очень неплохое, позволяет добраться до узких мест.
% Из минусов: ручка похоже не цельнометаллическая, при изломе есть риск
% распороть руку.



\secrel{Паяльное оборудование}\secdown

\secrel{Паяльник}

Паяльник\ --- обязателен дешевый сетевой мощностью не менее 20\,Вт, типа
ЭПСН-25/220. Ограничитель мощности или регулятор температуры легко собрать
самостоятельно.

Для сборки электроники хорошо также иметь маленький монтажный 12\,В 8\,Вт от
паяльной станции ZD-927 ($\sim$100\,р), без самой станции. Если не жалко 500\,р,
берите станцию ZD-927 целиком, внутри простейший регулятор мощности, и вам не
понадобится источник питания на 12\,В, который вы еще не сделали.

\noindent\includegraphics[width=0.4\textwidth]{tech/tools/solder/EPSN25.jpg}
\textbf{Паяльник ЭПСН-25/220}

\noindent\includegraphics[width=0.4\textwidth]{tech/tools/solder/SV-55310-25.jpg}
\textbf{Паяльник 220В 25Вт, СВЕТОЗАР, SV-55310-25 230\,р.}

\noindent\includegraphics[width=0.4\textwidth]{tech/tools/solder/ZD-721N.jpg}
\textbf{Паяльник 220В 25Вт ZD-721N 175\,р.}

\noindent\includegraphics[width=0.4\textwidth]{tech/tools/solder/Iron8W.jpg}
\textbf{Паяльник для станции ZD-927 12\,В 8\,Вт 85\,р.}

\secrel{Паяльная станция}

Из всего разнообразия для хоббита оптимальным являются паяльные станции Lukey
702/853D (3000+\,р). Для работы или регулярного хобби паяльная станция с феном,
а может даже и встроенным источником питания, вещь незаменимая, и не такая уж
дорогая.

\includegraphics[width=0.45\textwidth]{tech/tools/solder/ZD927.jpg}
\textbf{Паяльная станция ZD-927 520\,р.}

\includegraphics[width=0.45\textwidth]{tech/tools/solder/Lukey702.jpg}
\textbf{Паяльная станция LUKEY 702 3100\,р.}

\clearpage
\includegraphics[width=0.95\textwidth]{tech/tools/solder/Lukey853D.jpg}

\textbf{Паяльная станция LUKEY 853D с источником питания 5200\,р.}

\secup


% \section{JTAG-адаптер}
% 
% % \input{jtag/soft}
% % \input{stlink/stlink1}
% 
% \section{Отладочные платы}
% 
% Прежде чем начать работать с отдельными \mk, устанавливая их на плату
% собственной разработки, для быстрого старта используют \term{отладочные
% платы}\note{development board, demo board}
% 
% \subsection{Arduino /Atmel Mega AVR8/}
% 
% \subsection{Cortex-Mx} %См. \ref{devkitcmx}
% 
% \subsection{CubieBoard /Cortex-A8 AllWinner A10/}
% 
% \subsection{Raspberry Pi /ARM11 BCM3032/}
% 
% \subsection{BlackSwift /MIPS/}
% 
% \subsection{VoCore /MIPS/}

\secrel{Измерительное оборудование}\secdown

\secrel{Мультиметр}\label{mmetr}

\emph{Мультиметр\ --- обязателен, без него работать невозможно}\note{или
собирать замену на паре измерительных головок тока/напряжения, и делителях}.
Для совсем начинающего больше всего подойдет M320\ref{mmetr320} c
автодиапазоном, когда освоитесь возьмете вторым прибором что-то из крупных серий
M89x/MY6x с измерением температуры\note{иногда нужно для измерения температуры
корпусов элементов, радиаторов, растворов если возитесь с электрохимией}
или ``рыльцеметр''\ref{rlcmetr} (RLC).

\secdown
\secrel{Mastech M838}\label{mmetr838}

\begin{tabular}{p{0.3\textwidth} p{0.6\textwidth}}
\noindent\includegraphics[width=0.3\textwidth]{tech/tools/mes/M838.jpg}
&
Простой, компактный, дешевый, \emph{с измерением температуры}
\\
\end{tabular}

\secrel{Mastech M300}\label{mmetr300}

\begin{tabular}{p{0.3\textwidth} p{0.6\textwidth}}
\noindent\includegraphics[width=0.3\textwidth]{tech/tools/mes/M300.jpg}
&
Простой, \emph{очень компактный}, дешевый, в чехле очень удачно умещается в
набор инструментов.
\\
\end{tabular}

\secrel{Mastech M320}\label{mmetr320}

\begin{tabular}{p{0.3\textwidth} p{0.6\textwidth}}
\noindent\includegraphics[width=0.3\textwidth]{tech/tools/mes/M320.jpg}
&
То же что и M300\ref{mmetr300}, но с \emph{автодиапазоном}, т.е. не требует
переключения диапзонов измерения вручную. На любителя, возможно \emph{удобен для
совсем начинающих}, но слишком медленен если требуется измерение меняющегося
тока/напряжения.
\\
\end{tabular}

\secup

\secrel{Осциллограф}

\secrel{Логический анализатор}

\secrel{Генератор сигналов}

\secrel{Рыльцеметр RLC}\label{rlcmetr}

\secup

\secrel{Электроинструмент}\secdown

\secrel{Дрелъ}

% \noindent
% \begin{tabular}{p{0.5\textwidth} p{0.5\textwidth}}
% \noindent
% \includegraphics[width=0.45\textwidth]{tech/tools/PraktylR.jpg}
% &
% \noindent
% \includegraphics[width=0.45\textwidth]{tech/tools/D_11_530ER.jpg}
% \\
% \textbf{Дрель ударная сетевая} & \textbf{Дрель безударная сетевая} \\
% \textbf{Praktyl-R PID13D01 400\,Вт} 
% \href{http://leroymerlin.ru/catalogue/instrumenty/elektroinstrument/dreli\_udarnye/13805983/}{\textbf{(!)395\,р.}}
% &
% \textbf{Интерскол Д-11/530ЭР (с БЗП)}
% \href{http://leroymerlin.ru/catalogue/instrumenty/elektroinstrument/dreli\_bezudarnye/11857763/}{\textbf{1120\,р.}}
% \\
% \end{tabular}
% \bigskip
% 
% Дрель\ --- одноразовая китайчатина от 400\,р. Продаются уже брендированные на
% Леруа Мерлен, наклейка <<PID13D01 Ударная дрель 400\,Вт, 13\,мм>>. Скорость
% регулируется глубиной нажатия курка, крутилка на курке ограничивает глубину
% механически, фиксатор держит скорость близко к минимальной, запаха горелой
% пластмассы через несколько минут работы на холостом ходу нет.
% 
% По надежности рекомедуется Интерскол 1100+\,р. Надежность Интерскола\ --- не
% <<китай>>, классика ДУ-580ЭР работает в хвост и гриву, используется криворукими
% студентами, лежит в подвале в пыли от точила, и никаких вопросов даже со
% щетками.
% 
% Если не планируете много сверлить бетон, \textbf{берите дрель без ударного
% механизма}: отсутствуют лишние продольные перемещения, что может быть важно при
% использовании в качестве шпинделя сверлильного станка, и механизации других
% технологических поделок.
% 
% У шуруповерта нет 43\,мм шейки для фиксации, поэтому как средство электропривода
% он практически бесполезен, и нужен собственно для заворачивания большого
% количества саморезов. Хотя наличие ограничителя крутящего момента и малые
% габариты удобны при сверлении и сборке поделок.
% 
% \bigskip
% Имея некоторое количество поделочного материала, кривые руки и особенно доступ к
% станочному оборудованию, можно сколкозить некоторое подобие настольных
% станочков\ \pref{fig:drelstans}\ для механизации некоторых работ,
% используя дрель в качестве привода.
% 
% Главным элементом такой оснастки\ --- зажим на шейку дрели 43\,мм. Особых
% требований по его точности и качеству нет, т.к. сама шейка обычно пластиковая, и
% никакой доводки по круглости и параллельности оси инструмента не проходит.
% 
% \clearpage
% \phantomsection\label{fig:drelstans}
% \noindent\includegraphics[height=0.528\textheight]{tech/tools/DrelLathe.jpg}
% \noindent\includegraphics[height=0.528\textheight]{tech/tools/DrelShliph.jpg}
% 
% \noindent\includegraphics[height=0.465\textheight]{tech/tools/DrelLathe2.jpg}
% \noindent\includegraphics[height=0.465\textheight]{tech/tools/DrelBoren.jpg}
% \clearpage

\secrel{Лобзик}

% \noindent
% \begin{tabular}{p{0.5\textwidth} p{0.5\textwidth}}
% \noindent
% \includegraphics[width=0.45\textwidth]{tech/tools/LobzPraktyl.jpg}
% &
% \noindent
% \includegraphics[width=0.45\textwidth]{tech/tools/LobzMakita4329.jpg}
% \\
% \href{http://leroymerlin.ru/catalogue/instrumenty/elektroinstrument/lobziki/13805991/}{\textbf{Praktyl
% 350 Вт 356\,р.}} 
% & 
% \href{http://leroymerlin.ru/catalogue/instrumenty/elektroinstrument/lobziki/12114283/}{\textbf{Makita
% 4329 2260\,р.}}
% \\
% \end{tabular}
% \bigskip
% 
% Лобзик полезен при разделке стеклотекстолита, и изготовлении технологической
% мебели (стеллажи, рабочие столы и т.п.).

\secrel{Жвигатель}

% Если у вас возникло желание механизировать изготовление механических деталей, а
% свободного доступа к настоящему станочному оборудованию нет, есть смысл
% рассмотреть изготовление самодельной механизированной оснастки 
% типа\ \pref{fig:drelstans}, или даже самодельных станочков. В этом случае надо
% рассмотреть применения универсального привода.
% 
% Первый кандидат на место универсального электропривода достается той самой
% дрели, не забываем об обязательном наличии 43\,мм монтажной шейки.
% Достоинство дрели как привода\ --- прямое подключение к сети, встроенный
% редуктор, есть модели с простой регулировкой оборотов, есть резьба и отверстие
% под винт на валу, в комплекте есть патрон для зажима мелких деталей в
% точилке\footnote{\ БЗП удобен, патрон с ключем дает лучший зажим и возможно
% точнее}.
% 
% Ограниченно доставаемые двигатели от стиральных машин, отличаются мощностю и
% оборотистостью, особенно от старых моделей. Часто доступны сразу с готовым
% шкивом на валу, который иногда проще использовать, чем снять.
% 
% Автозапчасти: привод печки Камаза, двигатель постоянного тока 
% 24\,В 50\,Вт
% 
% Новые асинхронные двигатели АИРЕ 56 B2/B4 (3000/1500 об.) с заводским
% конденсатором, подключается к сети $\sim$220\,В, цена от 2500\,р.
% С ростом размеров и мощности цена резко повышается.
% Следует обратить внимание на возможность монтажа на дополнительный фланцевый
% подшипниковый щит, (?) с моделями АИРЕ 80.
% 
% Для самодельных серлилок и микроинструмента хороши китайские воздушные шпиндели
% постоянного тока с цанговыми патронами ER11. Требуют источник питания
% постоянного тока 9$\div$48\,В. В магазинах не попадались, необходима прямая
% покупка с \href{http://www.aliexpress.com/}{AliExpress}\note{пользуйтесь
% английской версией\ --- переводная жуткое УГ}\ по почте.
% 
% % \clearpage
% \begin{tabular}{l l}
% 
% \noindent\includegraphics[width=0.37\textwidth]{tech/tools/VyatkaDvig.jpg} 
% & 
% \noindent\includegraphics[width=0.37\textwidth]{tech/tools/KamazDvig.jpg}
% \\
% \textbf{Жвигатель Вятка-Автомат 19??\,г.}
% &
% \textbf{Двигатель печки Камаза}
% \\
% 
% \noindent\includegraphics[width=0.37\textwidth]{tech/tools/AIRE.jpg}
% & 
% \noindent\includegraphics[width=0.37\textwidth]{tech/tools/ER11.jpg}
% \\
% \textbf{АИРЕ 56 B2, 0.2\,КВт}
% &
% \textbf{Воздушный шпиндель с цангой ER11}
% \\
% 
% \end{tabular}
% \clearpage
% 
% Съемные фрезерные шпиндели, поставляются отдельно или в комплекте с насадкой
% ручного фрезера по дереву. Лучшие, со стальной шейкой\ --- Kress, активно
% применяются хобби-ЧПУшниками. Попроще и сильно дешевле делал Интерскол, иногда
% попадается noname. Недостаток как универсального привода\ --- они
% высокоскоростные, возникают проблемы с понижающими передачами. Применение\ ---
% приводной высокоскоростной инструмент: боры, фрезы по дереву, микроинструмент
% для граверов (микродиски, шарошки). Цанга 8\,мм. Для некоторых моделей бывают
% наборы цанг на мелкий инструмент.
% 
% \bigskip
% \begin{tabular}{p{0.3\textwidth} p{0.3\textwidth} p{0.3\textwidth} }
% \noindent\includegraphics[height=0.3\textheight,width=0.3\textwidth,keepaspectratio]{tech/tools/Kress530.jpg}
% &
% \noindent\includegraphics[height=0.3\textheight,width=0.3\textwidth,keepaspectratio]{tech/tools/Interskol30.jpg}
% &
% \noindent\includegraphics[height=0.3\textheight,width=0.3\textwidth,keepaspectratio]{tech/tools/InterskolFM55.jpg}
% \\
% KRESS 530/800/1050 FM(E)
% &
% Интерскол ФМ-30/750
% &
% Интерскол ФМ-55/1000 Э
% \\
% \href{http://kress-shop.ru/product/frezernyj-dvigatel-530-fm-kress-06082302/}{5600+\,р.}
% &
% /снят с производства/
% &
% \href{http://www.kuvalda.ru/catalog/1867/27920/}{5050\,р.}
% \\
% \end{tabular}

\secup



\secup


\chapter{Трассировка плат и подготовка производства в KiCAD}

\section{Технология ЛУТ (Лазерный УТюг)}

\section{Технология фоторезиста}

\section{Формат Gerber и подготвка промышленного производства}


\secrel{Разработка конструкции в САПР FreeCAD}\secdown

\includegraphics[height=0.5\textheight]{logo/FreeCAD.png}

\cp{https://ru.wikipedia.org/wiki/FreeCAD_(Juergen_Riegel\%27s)}
В среде специалистов ряда отраслей известна проблема создания полноценной САПР в
рамках OpenSource, и хотя FreeCAD ещё не является кандидатом на такую
«полноценность», этот продукт может рассматриваться как одна из попыток создания
базы для решения этой проблемы. Разработчик FreeCAD Юрген Ригель, работающий в
корпорации DaimlerChrysler, позиционирует свою программу как первый бесплатный
инструмент проектирования механики (сравнивая свой продукт с такими развитыми
проприетарными системами как CATIA версий 4 и 5, SolidWorks), созданный на
основе библиотеки \textbf{Open CASCADE}. Цель программы\ --- предоставить
базовый инструментарий этой библиотеки в интерактивном режиме.

Следует отметить, что имеет место ещё один программный продукт имеющий название
freeCAD, его разработчик\ --- Aik-Siong Koh, и он не связан с FreeCAD’ом Юргена
Ригеля.

\bigskip\cp{http://www.freecadweb.org/wiki/index.php?title=Getting\_started}
FreeCAD\ --- CAD/CAE приложение трёхмерного параметрического моделирования.
Оно в основном сделано для механического проектирования, но также может быть
использовано для любых других случаев, в которых вам нужно точно моделировать
трёхмерные объекты с контролем над историей моделирования.

FreeCAD все еще находится в ранней стадии разработки, так что, хотя он уже
предлагает Вам большой (и растущий) список функций, многого еще не хватает,
особенно если сравнивать его с коммерческими решениями, и вы можете не найти его
достаточно развитым для использования в производственной среде. Тем не менее,
есть быстрорастущее сообщество пользователей-энтузиастов, и вы уже можете найти
много примеров качественных проектов, разработанных с FreeCAD.

Как и все проекты с открытым исходным кодом, проект FreeCAD не единственый
способ работы обеспеченный Вам его разработчиками. Это во многом зависит от
роста его сообществу пользователей и разработчиком, доработки функций и
стабилизации кода (да здравствует исправление ошибок!). Так что не забывайте об
этом, когда начинаете использовать FreeCAD, если вам он нравится, вы можете
непосредственно влиять и помочь проекту!

\secrel{Установка}\secdown

\secrel{\win}

\menu{\winr>\url{http://www.freecadweb.org/}>Download>\win>\href{http://sourceforge.net/projects/free-cad/files/FreeCAD Windows/FreeCAD 0.14/}{\file{FreeCAD
0.14}}>\file{\ldots\_setup.exe}}

\menu{\file{FreeCAD 0.14.3700\_x86\_setup}>FreeCAD 0.14 License>I agree}

\menu{Distination Folder>\file{C:/FreeCAD}>Install>Completed>Close}

\menu{\winstart>Программы>FreeCAD 0.14>FreeCAD>\rms>Закрепить в панели задач}

\includegraphics[height=0.8\textheight]{freecad/about.png}

\secrel{\linux}

\secup

% \secrel{Чертеж}
% 
% \secrel{Эскиз}
% 
% \secrel{Деталь}
% 
% \secrel{Сборка}
% 
% \secrel{Автогенерация конструкторской докуметации}
% 
% \secrel{Скрипты и пользовательские расширения}
% 

\secup

\chapter{Эксплуатация станочного оборудования}

\chapter{Основы ЧПУ и цифрового производства}

\section{CAM-пакеты для FreeCAD}

\part{Основы теории систем автоматического управления}

\chapter{Математический аппарат}

\section{Передаточная функция}

\section{Устойчивость САУ}

\section{Сети Петри}

\section{Автоматы Маркова}

\chapter{Релейное управление}

\chapter{Пропорциональные САУ}

\chapter{ПИДn-регуляторы}

\secrel{Разработка ПО для встраиваемых систем}\secdown

% \chapter{Архитектура программных систем}

\section{Ортогональность}

\cp{http://ru.wikibooks.org/wiki/\%D0\%9E\%D1\%80\%D1\%82\%D0\%BE\%D0\%B3\%D0\%BE\%D0\%BD\%D0\%B0\%D0\%BB\%D1\%8C\%D0\%BD\%D0\%BE\%D1\%81\%D1\%82\%D1\%8C}

\term{Ортогональность}\ очень важна, если вы хотите создавать системы, которые
легко поддаются проектированию, сборке, тестированию и расширению. Однако этому
принципу редко обучают непосредственно. Часто он является лишь скрытым
достоинством других разнообразных методик, которые вы изучаете. Это неправильно.
Как только вы научитесь непосредственно применять принципы ортогональности, вы
сразу заметите, как улучшилось качество создаваемых вами систем.

\paragraph{Что такое ортогональность?}

Термин "ортогональность"\ заимствован из геометрии. Две линии являются
ортогональными, если они пересекаются под прямым углом, например, оси координат
на графике. В терминах векторной алгебры две \emph{такие линии перемещения
являются независимыми}. Если двигаться параллельно оси X вдоль одной из линий,
то проекция движущейся точки на другую линию не меняется. Этот термин был введен
в информатике для обозначения некой разновидности независимости или
несвязанности. В грамотно спроектированной системе программа базы данных будет
ортогональной к интерфейсу пользователя: вы можете менять интерфейс пользователя
без воздействия на базу данных и менять местами базы данных, не меняя
интерфейса. Перед тем как рассмотреть преимущества ортогональных систем,
познакомимся с неортогональной системой.

\paragraph{Неортогональная система}

Предположим, вы находитесь в экскурсионном вертолете, совершающем полет над
Гранд-Каньоном, когда пилот, который совершил ошибку, наевшись рыбы за обедом
внезапно вскрикивает и теряет сознание. По счастливой случайности это
происходит, когда вы парите на высоте 30 метров. Вы догадываетесь, что рычаг
управления общим шагом несущего винта обеспечивает подъем машины, так что, если
его слегка опустить, вертолет начнет плавно снижаться. Однако когда вы пытаетесь
сделать это, то осознаете, что жизнь\ --- не такая уж простая штука. Вертолет
клюет носом, и вас начинает вращать по спирали влево. Внезапно вы понимаете, что
управляете системой, в которой каждое воздействие имеет побочные эффекты. При
нажатии на левый рычаг вам придется сделать уравновешивающее движение назад
правым рычагом и нажать на правую педаль. Но при этом каждое из этих действий
вновь повлияет на все органы управления. Неожиданно вам приходится жонглировать
невероятно сложной системой, в которой любое изменение влияет на все остальные
управляющие воздействия. Вы испытываете феноменальную нагрузку: ваши руки и ноги
находятся в постоянном движении, пытаясь уравновесить все взаимодействующие
силы. Органы управления вертолетом определенно не являются ортогональными.

\paragraph{Преимущества ортогональности}

Как показывает пример с вертолетом, неортогональные системы сложнее изменять и
контролировать. Если составляющие системы отличаются высокой степенью
взаимозависимости, то невозможно устранить какую-либо неисправность лишь на
локальном уровне.

\bigskip
\emph{Исключайте взаимодействие между объектами, не относящимися друг к другу}
\bigskip

Мы хотим спроектировать компоненты, которые являются самодостаточным
независимыми, с единственным, четким назначением. Когда компоненты изолированы
друг от друга, вы уверены, что можно изменить один из них, не заботясь об
остальных. Пока внешние интерфейсы этого компонента остаются неизменными можете
быть спокойны, что не создадите проблем, которые распространятся по
\emph{всей}\ системе. С созданием ортогональных систем у вас появятся два
больших преимущества: увеличение производительности и снижение риска.

\paragraph{Увеличение производительности}

\begin{itemize}
\item Изменения в системе локализуются, поэтому периоды разработки и
тестирования сократятся. Легче написать относительно небольшие, самодостаточные
компоненты, чем один большой программный модуль. Простые компоненты могут быть
спроектированы, запрограммированы, протестированы и затем забыты\ --- не нужно
непрерывно менять существующий текст по мере того, как к нему добавляются новые
фрагменты.

\item Ортогональный подход также способствует многократному использованию
компонентов. Если компоненты имеют определенную, четкую сферу ответственности,
они могут комбинироваться с новыми компонентами способами, которые не
предполагались при их первоначальной реализации. Чем меньше связанность в
системах, тем легче их перенастроить и провести их обратное проектирование.

\item При комбинировании ортогональных компонентов происходит заметное
увеличение производительности. Предположим, что один компонент способен
осуществлять М, а второй\ --- N различных операций. Если эти компоненты
ортогональны и комбинируются, то в сумме они способны осуществить $M\times N$
различных операций. Но если два компонента не являются ортогональными, они будут
перекрываться, и результат их действия будет меньшим по сравнении с
ортогональными компонентами. Вы получаете большее количество функциональных
возможностей в пересчете на единичное усилие, если комбинирует между собой
ортогональные компоненты.
\end{itemize}

\paragraph{Снижение риска}

Ортогональный подход приводит к снижению уровня риска, присущего любой
разработке.

\begin{itemize}
\item Ошибочные фрагменты текста программы изолируются. Если модуль содержит
ошибку, то вероятность ее распространения на всю систему уменьшается. Кроме
того, ошибочный фрагмент может быть извлечен и заменен новым (исправленным).

\item Конечный продукт (система) становится менее хрупким. Проблемы,
появляющиеся при внесении небольших изменений и устранении недочетов на
определенном участке, не проходят дальше этого участка.

\item Ортогональная система способствует повышению качества тестирования,
поскольку облегчается проектирование и тестирование отдельных ее компонентов.

\item Вы не будете слишком сильно привязаны к определенному субподрядчику,
программному продукту или платформе, поскольку интерфейсы между компонентами,
производимыми фирмами-субподрядчиками, не будут играть главенствующей роли в
проекте.
\end{itemize}




\secrel{VCS: cистемы контроля версий}\label{vcs}\secdown

\secrel{CVS}

\secrel{Subversion}

\secrel{Git}
\secdown

\secrel{GitHub}

\secup

\secup


\chapter{Вспомогательные скрипты на языке \py}

\begin{tabular}{p{0.1\textwidth} p{0.8\textwidth}}
\includegraphics[width=0.1\textwidth]{python/logo.png}&
\emph{
Название языка произошло вовсе не от вида пресмыкающихся. Автор назвал язык в
честь популярного британского комедийного телешоу 1970-х «Летающий цирк Монти
Пайтона». Впрочем, всё равно название языка чаще ассоциируют именно со змеёй,
нежели с передачей\ --- пиктограммы файлов в KDE или в Microsoft Windows и даже
эмблема на сайте \url{http://www.python.org} (до выхода версии 2.5) изображают
змеиные головы.
}
\\
\end{tabular}
\bigskip

\py\note{в оригинале читается \textbf{п\'{а}йтон}, но давно русифицировался как
\textbf{пит\'{о}н}}\ --- высокоуровневый язык программирования общего
назначения, ориентированный на повышение производительности разработчика и
читаемости кода.

\py\ удобно применять для написания различных вспомогательных скриптов.
Часто его используют при разработке сложных программных систем для написания
первых версий. В процессе работы над большими программами часто перерабатываются
большие объемы кода, поэтому для ускорения разработки требуется максимально
высокоуровневый язык. После того как архитектура программы стабилизируется,
узким местом становится производительность, и программу переписывают на более
низкоуровневом компилируемом языке, чаще всего \cpp.

Написание программ упрощают:

\begin{itemize}
  \item \textbf{объектно-ориентированное программирование} облегчает разработку
  программ, позволяет переопределить стандартные операторы для пользовательских
  типов данных, упрощая синтаксис
  \item \textbf{динамическая типизация} не требуется заранее упределять
  переменные, они создаются простым присваиванием
  \item \textbf{обработка исключений} для секции кода можно определить
  обработчик ошибок
  \item \textbf{высокоуровневые структуры данных}\ --- списки, словари (набор
  элементов ключ:значение), очереди
  \item богатая стандартная библиотека и множество дополнительных библиотек на
  все случаи
\end{itemize}

\section{Установка под \win}\label{pywinstall}

\bigskip
\menu{
\keys{\winstart+R}
>
\url{http://www.python.org}
>
Downloads
>
\href{https://www.python.org/ftp/python/2.7.8/python-2.7.8.msi}{Python 2.7.8}
}

\bigskip
\menu{\file{python-2.7.8.msi}
>
Setup>
for all users/for me
}

\menu{Destination Directory > \file{C:/Python/} > Next}

\nopagebreak
\bigskip
\includegraphics[width=0.45\textwidth]{python/install/037.png}
\includegraphics[width=0.45\textwidth]{python/install/038.png}
\bigskip


\menu{Customize > Python > Add python.exe to PATH > Next > Finish}

\nopagebreak
\bigskip
\includegraphics[width=0.45\textwidth]{python/install/039.png}
\includegraphics[width=0.45\textwidth]{python/install/045.png}
\bigskip

\section{Дополнительные материалы}

\cite{pyotkidach} Г. Россум, Ф.Л.Дж. Дрейк, Д.С. Откидач, 
\href{http://rus-linux.net/MyLDP/BOOKS/python.pdf}{Язык программирования Python}

\cite{pythink} Аллен Дауни
\href{https://drive.google.com/file/d/0B0u4WeMjO894Q2hWV1QwOFFQOVk/view?usp=sharing}{Думать
на языке \py: Думать как компьютерный специалист}



% \chapter{Make: управление сборкой проектов}\label{make}
% 
\secrel{Основы Си и \cpp}\secdown
\secup


% \subsection{Установка MinGW (win32)}
% 
% \section{Особенности \cpp\ в embedded}
% 
% \chapter{LLVM и разработка собственных компиляторов}
% 
% \secrel{Лексический и синтаксический анализ}\secdown

Очень часто в практике возникает необходимость работы с данными в текстовых
форматах\ --- \termdef{plain text}{plain text} файлы, в которых в каком-либо
формате (на языке разметки, или \termdef{DDL}{DDL}: \termdef{[D]ata
[D]efinition [L]anguage}{Data Definition Language}) описаны данные. И от вас
требуется реализовать разбор такого файла, выделяя синтаксические структуры и
элементы данных, чтобы в дальнейшем после их преобразования например записать
текстовый файл в другом формате.

В таком виде хранятся результаты расчетных программ, работащих в пакетном
режиме, данные с измерительных систем, поток данных с приемников
GPS\note{протокол NMEA 0183}, очень популярный мета-формат XML со всеми его
частными случаями типа HTML, XLIFF\ref{xliff}, OpenDocument, тексты программ
для станков с ЧПУ,\ldots

С некоторыми хинтами точно так же можно работать и с бинарными файлами,
преобразовав их сначала в текстовую форму (в простейшем случае просто сделав hex
dump).

В некоторых случаях необходимо написание трансляторов форматов (текстовых)
данных, или даже интерпретаторов/компиляторов языков программирования.

\bigskip
Все эти техники с использованием стандартных утилит \prog{flex}\ и \prog{bison}\
будут кратко описаны в этой главе. Подробнее эти техники рассмотрены в
книгах\ref{lexlit}, особенно стоит отметить талмуд
\textbf{DragonBook}:

\bigskip

\label{exdragon}\cite{dragonbook} \textbf{Книга Дракона}: Ахо, Сети, Ульман
Принципы построения компиляторов.

\bigskip

\href{http://habrahabr.ru/post/99162/}{Habr: Компиляция. 1: лексер}

\href{http://habrahabr.ru/post/99298/}{Habr: Компиляция. 2: грамматики}

\href{http://habrahabr.ru/post/99366/}{Habr: Компиляция. 3: бизон}

\href{http://habrahabr.ru/post/99397/}{Habr: Компиляция. 4: игрушечный ЯП}

\href{http://habrahabr.ru/post/99466/}{Habr: Компиляция. 5: нисходящий разбор}

\href{http://habrahabr.ru/post/102597/}{Habr: Компиляция. $5\frac{1}{2}$: llvm
как back-end}

\href{http://habrahabr.ru/post/99592/}{Habr: Компиляция. 6: промежуточный код}


\secrel{Лексер и лексический анализ, утилита \prog{flex}}

\begin{framed}
\noindent
\termdef{Л\'{е}ксер}{лексер}/\termdef{сканер}{сканер}\ --- программа или ее
часть, которая \begin{enumerate}[nosep]
  \item
получает на вход исходные
данные в виде сплошного потока одиночных сиволов,
  \item
группирует символы согласно
набору правил (заданных \termdef{регулярными выражениями}{регулярное
выражение}) и
  \item
отдает на выходе символы, уже сгруппированные в \termdef{лекс\'{е}мы}{лексема}\
или \termdef{ток\'{е}ны}{токен}.
\end{enumerate}
Цель лексера\ --- подготовить последовательность лексем для входа другой
программы. \\
В самый простых случаях на лексер можно возложить простые преобразования текста.
\end{framed}

\termdef{Лексический анализ}{лексический анализ}\ --- процесс программного
разбора входной последовательности символов\note{например, такой как исходный
код на одном из языков программирования} с целью получения на выходе
последовательности групп символов\ --- \term{токенов}, имеющих собственное
смысловое значение\note{подобно группировке букв в слово}. Как правило,
лексический анализ производится в соответствии набора правил определённого
\term{формального, искуственного или компьютерного языка}.

\begin{framed}\noindent
\term{Компьютерный язык}, а точнее его \termdef{грамматика}{грамматика}, задаёт
определённый набор лексем, которые могут встретиться на входе лексера, и набор
правил, по которым их следует группировать.
\end{framed}

Традиционно принято организовывать процесс лексического анализа, рассматривая
входную последовательность символов как поток одиночных символов. При такой
организации \term{лексер}\ самостоятельно управляет выборкой отдельных символов
из входного потока.

Распознавание лексем с учетом грамматики обычно производится путём их
идентификации согласно идентификаторам токенов, определяемых грамматикой языка.
При этом любая последовательность символов входного потока (лекс\'{е}ма),
которая согласно грамматике не может быть идентифицирована как токен языка,
обычно рассматривается как специальный \term{токен-ошибка}.

\bigskip
Каждый выделенный токен можно представить в виде парной структуры,
содержащей
\begin{enumerate}
  \item идентификатор токена и
  \item саму последовательность символов лексемы, выделенной из входного
потока\note{запись строки, числа и т. д.}.
\end{enumerate}

\bigskip
Рассморим обработку текстового файла: разделение текстового фрагмента на абзацы,
запись в формате
\href{http://docs.oasis-open.org/xliff/xliff-core/xliff-core.html}{XLIFF} для
перевода в системе \href{http://smartcat.pro}{ABBYY SmartCAT}, и обратной
трансляции из XLIFF в \LaTeX-совместимое форматирование.

\bigskip
Так как задача построения \termdef{лексических анализаторов}{лексический
анализатор} является стандартной задачей информатики, был разработан типовой
инструмент: \term{генератор лексических анализаторов} \prog{flex}. Эта программа
транслирует описание лексера на своем высококоуровневом языке в фрагмент
программы на языке Си/\cpp\ или самостоятельную программу. Описание лексера
прописывается в \file{.lex}-файле в формате:

\begin{verbatim}
определения, опции, декларации
%%
правила выделения токенов через регулярки
%%
сишный код
\end{verbatim}

\paragraph{Комментарии} поддерживаются сишные \verb|/* */| комментарии.

\paragraph{Опции} \verb|%option|

\bigskip\noindent
\begin{tabular}{l l}
noyywrap & отключает вызов лексером функции \verb|yywrap()| при достижении
конца текущего файла \\
main & включение типовой функции \verb|main()| вместо заданной пользователем \\
case-insensitive & регистро-независимый лексический анализ, большие/маленькие
буквы не различаются \\
yylineno & в глобальной си-переменной \verb|yylineno| доступен номер
текущей строки \\
\end{tabular}

\paragraph{Формат опреления} \verb|имя определение|:
\begin{verbatim}
digit    [0-9]
number   [\+\-]{0,1}{digit}+\.{digit}*
\end{verbatim}

\paragraph{Декларации на Си} прописываются в скобках \verb|%{ }%|

\cp{http://matt.might.net/articles/standalone-lexers-with-lex/}

\lst{xliff/txt/txt2xliff.lex}{}{xliff/txt/txt2xliff.lex}

\secrel{Генератор синтаксических анализаторов \prog{bison}}

\begin{framed}
\termdef{Синтаксический анализ}{синтаксический анализ}\ --- процесс анализа
последовательности токенов с определением их грамматической структуры. На этом
этапе выделяются \term{синтаксичекие ошибки}.
\end{framed}

\secrel{Семантический анализ}

\begin{framed}
\termdef{Семантический анализ}{семантический анализ}\ --- процесс выполнения
\term{семантических проверок}: контроль типов, привязка объектов и т.д. 
\end{framed}

\secrel{Оптимизация}

Выполнение формальных преобразований структур данных, описываютщих
компилируюемую программу, с целью построения более компактного/быстрого
машинного кода.

\secrel{Кодогенерация}

Получение реального машинного кода в бинарном представлении, или в виде
ассемблерных текстовых файлов.

\secrel{Дополнительная литература}\label{lexlit}

\bigskip

\url{http://alumni.cs.ucr.edu/~lgao/teaching/flex.html}

\url{http://www.capsl.udel.edu/courses/cpeg421/2012/slides/Tutorial-Flex\_Bison.pdf}

\secrel{Транслятор Паскаля}

\secrel{LLVM и разработка собственных компиляторов}

\secup

% 
% \chapter{Сборка кросс-компилятора GNU toolchain}

\secup


\part{Микроконтроллеры \cmx}

\chapter{Отладочные платы}\label{devkitcmx}

\section{STM32DISCOVERY /Cortex-M3 STM32F103/}

\section{STM32F4DISCOVERY /Cortex-M4 STM32F407/}



\part{Периферия}

\part{Встраиваемый \emlinux}

\chapter{cross}

\chapter{BuildRoot}

\chapter{Особенности OpenWrt}

\chapter{Библиотека SDL}

\section{Реализация microGUI}

\chapter{Приложения для X Window}

\chapter{Программирование сетевых приложений}

\chapter{Сборка кросс-компиляторя GNU мальтийским крестом}

\part{IDE}

IDE\ --- Integrated Development Environment, интегрированная среда разработки.

Программный пакет, включающий 
\begin{itemize}
  \item средства управления проектом,
  \item отслеживание зависимостей между файлами (в т.ч. с анализом исходного
  текста программ на конструкции типа \verb|#include|, \verb|module|,
  \verb|uses|),
  \item автозапуском компиляторов для изменившихся файлов,
  \item GUI для отладчиков (gdb),
  \item специализированный редактор plain text\note{файлы не включающие
  непечатаемых символов и бинарных данных, которые можно причитать простым
  выводом на экран командами типа \textbf{type}, \textbf{cat}, \textbf{more}}
  файлов c
  \begin{itemize}
    \item цветовой и шрифтовой \term{подстветкой синтаксиса},
  	\item \term{автодополнением}: дописываются имена объектов программ, 
  	синтаксические конструкции и параметры функций,
  	\item \term{автоформатированием}: фрагмент текста переформатируется в
  	соответствии с синтаксисом языка редактируемого файла, проставляются отступы в
  	зависимости от вложенности синтаксических конструкций типа циклов и условных
  	блоков)
  	\item выделением строк, на которые указывают сообщения об ошибках
  	компиляторов,
  	\item маркеры точек останова отладчика
  \end{itemize}
  \item отображение структуры программ, например деревья классов и структур
  данных
  \item контекстные справочники по используемым языкам программирования,
  автоматический вывод списка параметров при вводе имени функции
  \item отображение дизассемблерных листингов для компилируемых языков
  \item отображение браузера как вкладки или MDI окна
  \item отображение вывода \term{статических анализаторов} программ c
  кликабельными ссылками
  \item вывод компиляторов и трансляторов с цветовым выделением и переход на
  ошибочную строку в редакторе при щелчке на ошибке
  \item \ldots
\end{itemize}

В этой книге рассмотрены три бесплатных мультиплатформенных OpenSource IDE, в
порядке навороченности, универсальности, и требуемым ресурсам для работы самой
среды:

\begin{enumerate}
  \item \eclipse\ \ref{eclipse}: самая навороченная и ресурсоемкая IDE, написана
  на Java, имеет десятки дополнительных модулей на все случаи, умеет работать со
  всеми распространенными языками программирования, жрет память, и требует
  современного компьютера минимум с 2+ Гб ОЗУ. Последний релиз \eclipse\ Luna
  работает заметно быстрее (особенно при запуске). 
  \item Code::Blocks \ref{cb}: легкая среда для разработки на C/\cpp, для других
  языков модет потребоваться написать свои модули или файлы описания синтаксиса
  \item \vim\ \ref{vim}: самый легкий и \emph{портабельный} универсальный
  текстовый редактор с расширенными функциями, работает на всех
  существующих платформах (кроме совсем уж embedded), использует минимум
  ресурсов, но требует некоторого обучения даже чтобы выйти из \verb|vim|
  \smiley
\end{enumerate}

\chapter{\eclipse}\label{eclipse}

\includegraphics[height=0.5\textheight]{logo/eclipse.png}

\section{Установка \eclipse\ под \win}

\menu{\winr>\url{https://eclipse.org/}>Download}

\menu{Eclipse Luna release for>\win}

Качаем архив базовой системы:

\menu{Eclipse IDE for Java Developers>\win\ 32/64 Bit}

Или сразу сборку CDT\eclipse:

\menu{Eclipse IDE for C/C++ Developers>\win\ 32/64 Bit}

\section{Установка \eclipse\ под \linux}

\menu{\winr>\url{https://eclipse.org/}>Download}

\menu{Eclipse Luna release for>\win}

Качаем архив базовой системы:

\menu{Eclipse IDE for Java Developers>\linux\ 32/64 Bit}

Или сразу сборку CDT\eclipse:

\menu{Eclipse IDE for C/C++ Developers>\linux\ 32/64 Bit}

\bigskip

Устанавливаем в систему Java-рантайм:

\begin{verbatim}
sudo aptitude install openjdk-7-jre
\end{verbatim}

Распаковывем полученный архив
\file{eclipse-java-luna-SR1-linux-gtk-x86\_64.tar.gz}
в \file{\$HOME}:

\begin{verbatim}
cd ~
tar zx < Downloads/eclipse-java-luna-SR1-linux-gtk-x86_64.tar.gz 
ls -la eclipse/eclipse
-rwxr-xr-x 1 user user 74675 Авг 13 16:12 eclipse/eclipse
\end{verbatim}

Прописываем запуск \eclipse\ в ваш оконный менеджер или \file{.blackboxmenu}
с параметром \file{-noSplash} для лечения глюка с запуском на x64-битных
системах:

\lst{.blackbox.menu}{}{ide/eclipse_blackbox.menu}

\section{Установка CDT}

\href{https://eclipse.org/cdt/}{\prog{CDT}}\ --- расширение \eclipse\ для
разработки на Си/\cpp, редактирования make-файлов. Это расширение критически
важно для вашей работы, поэтому ставить его обязательно, или сразу качать сборку
CDT\eclipse.
\bigskip

\menu{\eclipse>Help>Install New Software\ldots}

\menu{Work with>Add>Add repository}

\menu{Name>CDT}

\menu{Location>\url{http://download.eclipse.org/tools/cdt/releases/8.5}}

\menu{OK}

\menu{Work with>CDT}

\menu{CDT Main Features>\checkbox\ C/C++ Development Tools}

\menu{CDT Optional Features}

Парсер файлов исходников на диалекте С99:
\menu{\checkbox\ C99 LR Parser}

Поддержка \prog{gcc}\ в режиме кросс-компиляции:
\menu{\checkbox\ GCC Cross Compiler Support}

Аппаратная отладка через \prog{gdb}:
\menu{\checkbox\ GDB Hardware Debugging}

\menu{Next>Next>Accept>Finish}

\section{Установка TeXlipse}

Если планируете работать с документацией в формате \latex, установите расширение
\href{http://texlipse.sourceforge.net/}{\prog{TeXlipse}}:
\bigskip

\menu{\eclipse>Help>Install New Software\ldots}

\menu{Work with>Add>Add repository}

\menu{Name>TeXlipse}

\menu{Location>\url{http://texlipse.sourceforge.net/}}

\menu{OK}

\menu{Work with>TeXlipse}

Это расширение поддерживает подсветку синтаксиса, автодополнение, построение
динамического оглавления, автокомпиляцию по сохранению, и несколько визардов
создания проекта.

\section{Редактирование файлов в формате XML и производных}

Установите пакет \eclipse\ WST:

\menu{Help>Install New Software}

\menu{Work with:>Luna - http://download.eclipse.org/releases/luna}

\menu{Filter:>WST>Eclipse WST>Next>Next>Restart>OK}

\section{Проверка орфографии}

\cp{http://www.simplecoding.org/proverka-orfografii-v-eclipse.html}

То, что проверка орфографии очень удобная вещь вряд ли нужно объяснять. Есть
конечно люди, которые не обращают на неё внимание, но это чаще всего из-за
экономии времени и отсутствия удобных средств проверки.

Действительно, удобная автоматическая проверка орфографии есть в офисных
пакетах, но мне сложно представить разработчика, который будет переносить
комментарии в Word и обратно \smiley.

Поэтому очень удобно иметь \emph{проверку правописания прямо в IDE}. И \eclipse\
в этом смысле полностью соответствует ожиданиям.

Долго объяснять что к чему нет смысла. Проверка орфографии встроена в \eclipse\
и если вы пишите только на английском, то может быть не захотите ничего менять.

Кроме того, есть
\href{http://www.102degrees.com/blog/2007/07/09/spell-checking-in-eclipse-pdt/}{статья
Aaron'а} (en) в которой автор рассказывает о подключении дополнительных словарей
и плагине \file{eSpell}.

Но \emph{русских словарей в дистрибутиве нет}, а при подключении внешних есть
нюансы. Поэтому мы максимально подробно рассмотрим \emph{подготовку и добавление
русских словарей}.

Первый вопрос. В каком виде должны быть словари и где их взять?

Тут всё просто. Формат словаря\ --- обычный текстовый файл, в котором каждое
слово начинается с новой строки. И нам вполне подойдут свободно распространяемые
словари \file{aSpell}.

Установка состоит из \ref{aspellecl}\ шагов:
\begin{enumerate}
  \item качаем \href{}{aSpell}\ и словари для нужных языков
  
  \menu{\winr>\url{http://aspell.net/win32/}>}
  
  \menu{Binaries>Full installer}
  
  \menu{Precompiled dictionaries>English}
  
  \menu{Precompiled dictionaries>Russian}
  
  \item устанавливаем сначала \file{aSpell}, потом отдельно каждый словарь
  
  \menu{\file{Aspell-0-50-3-3-Setup.exe}>Setup GNU Aspell>Next>License>Next}
  
  \menu{Directory>\file{C:/GnuWin32/Aspell}>Next>Next}
  
  \menu{Additional>Next>Install>Next>\uncheckbox\ View manual>Finish}
  
  \menu{\file{Aspell-en-0.50-2-3.exe}>Aspell English Dictionary>Next>License>Next}
  
  \menu{Directory>\file{C:/GnuWin32/Aspell}>Next>Next>Install>Finish}
  
  \menu{\file{Aspell-ru-0.50-2-3.exe}>Aspell Russian Dictionary>Next>License>Next}
  
  \menu{Directory>\file{C:/GnuWin32/Aspell}>Next>Next>Install>Finish}
  
  \item делаем дамп словарей, перекодируем из koi8r в utf8 и объединяем
  
  \menu{\winr cmd}

\begin{lstlisting}
cd \GnuWin32\Aspell
bin\aspell dump master en > en.dict
bin\aspell dump master ru > ru.koi8
iconv -f koi8-r -t utf-8 < ru.koi8 > ru.dict
copy en.dict + ru.dict enru.dict
\end{lstlisting}
  
  \item \label{aspellecl} настраиваем \emph{spell-checker} \eclipse
  
  \menu{\eclipse>Window>Preferences>Editors>Text editors>Spelling}
  
  \menu{User defined dictionary>\file{C:/GnuWin32/Aspell/enru.dict}}
  
  \menu{Encoding>UTF-8}
  
  \menu{Apply>OK}
  
\end{enumerate}


\secrel{\cb}\label{cb}\secdown
\secup

\chapter{\vim}\label{vim}

\includegraphics[height=0.5\textheight]{logo/vim.png}

\section{Установка под \win}

\menu{\winr cmd > \url{http://www.vim.org/} > Download > PC: MS-DOS and
MS-Windows > \href{ftp://ftp.vim.org/pub/vim/pc/gvim74.exe}{\file{gvim74.exe}}}

\menu{Vim 7.4 Setup>This will install>Да}

\menu{License>I'm Angry}

\menu{Installation Options>\checkbox\ Create .bat files>Next}

\menu{Installation Folder>Install}

\menu{Completed>Close}

\menu{Do you want to see README>\textbf{Да}}
\bigskip

Теперь можно настроить темную тему и выключение подстветки синтаксиса, по
умолчанию после установки используется светлая тема и подстветка выключена:

\nopagebreak
\menu{меню>Правка>Настройка запуска}
\bigskip

Переходим в конец файла и включаем \emph{режим вставки}

\keys{Ctrl+Down}\ \keys{Ins}\ \keys{Enter}\keys{Enter}

\begin{lstlisting}
syntax on
colorscheme pablo
\end{lstlisting}
\bigskip

Выходим в \emph{режим команд} и принудительно сохраняем

\keys{Esc}:\keys{w}\keys{!}\keys{Enter}\keys{Enter}
\bigskip

\textbf{Выходим из \vim}

\keys{Esc}:\keys{q}\keys{!}\keys{Enter}

\bigskip
Если не получилось (под Windows 7):

\bigskip
\menu{\winr cmd > \file{/Program Files (x86)/Vim/}}
\bigskip

Копируем файл \file{\_vimrc}\ в любой каталог, например в \file{/tmp/},
затем \menu{\rms\rms>Edit with Vim}, и повторяем редактирование еще раз.

\bigskip
Затем копируем \file{\_vimrc}\ обратно в \file{/Program Files (x86)/Vim/}\ с
заменой.

\bigskip
Если теперь открыть на редактирование тот же файл, или любой другой текстовый,
получим более удобный вид: для файлов известных типов будет работать подсветка
синтаксиса.

\nopagebreak\bigskip
\includegraphics[height=0.9\textheight]{ide/vim28.png}

\section{Выход из \vim}

\keys{Esc}\ :\ \keys{!}\ \keys{q}\ \keys{Enter}

\subsection{Выход с автосохранением}

\keys{Esc}\ \keys{Shift+Z}\ \keys{Shift+Z}

\section{Переход в режим редактирования}

\vim\ запускается в \emph{командном режиме}, для перехода в режим редактирования
используются следующие клавиатурные команды:

\begin{itemize}
  \item \keys{Ins}\ или \keys{i}: включение \emph{режима вставки} по текущему
  положению курсора
  \item \keys{Ins}\keys{Ins}\ или \keys{r}: включение \emph{режима перезаписи}
  поверх текста после курсора
  \item \keys{Shift+A}: включение режима вставки \emph{в конец текущей строки}
\end{itemize}

\section{Переход в режим команд}

\keys{Esc}

\section{Запись редактируемого файла}

\keys{Esc}\ :\ \keys{w}\ \keys{Enter}
\bigskip

Если выводится предупреждение типа ``файл защищен от записи'' или подобное,
может сработать принудительная запись:

\bigskip
\keys{Esc}\ :\ \keys{!}\ \keys{w}\ \keys{Enter}

\section{Перезагрузка файла}

Для перезагрузки возможно изменененного извне файла или отмены всех
несохраненных изменений

\bigskip
\keys{Esc}\ :\ \keys{e}\ \keys{Enter}

\section{Отмена последних изменений (undo)}

\keys{Esc}\keys{u}\keys{u}\ldots



\part{Замечания для участников проекта}

\section{Набор репозиториев на GitHub}

\begin{tabular}{l l}

\url{https://github.com/ponyatov/Azbuka}
& основная репа \\

\url{https://github.com/ponyatov/bib}
& библиографические базы данных \\

\url{https://github.com/ponyatov/scratcher}
& журнал, используются некоторые материлы \\

\end{tabular}
\bigskip

Для работы с проектом сделайте собственный форк основной репы,
библиографическую базу и журнал можете клонироввать напрямую.
Создайте каталог и склонируйте репы:

\begin{lstlisting}
D:
cd \
mkdir w
cd \w\
git clone --depth=1 -o gh git@github.com:username/Azbuka.git

git clone --depth=1 -o gh git@github.com:ponyatov/bib.git
git clone --depth=1 -o gh git@github.com:ponyatov/scratcher.git
\end{lstlisting}

\section{Верстка в \latex}



\part{Подготовка публикаций в \latex}

\cp{https://ru.wikipedia.org/wiki/LaTeX}

LaTeX (по-русски произносится \textbf{лат\'eх})\ --- наиболее популярный набор
макрорасширений (или макропакет) системы компьютерной вёрстки \TeX, который
облегчает набор сложных документов. В типографском наборе форматируется как
\LaTeX.

Главная идея \latex\ состоит в том, что авторы должны думать о содержании, о
том, что они пишут, не беспокоясь о конечном визуальном облике (печатный
вариант, текст на экране монитора или что-то другое). Готовя свой документ,
автор указывает логическую структуру текста (разбивая его на главы, разделы,
таблицы, изображения), а \latex\ решает вопросы его отображения. Так содержание
отделяется от оформления. Оформление при этом или определяется заранее
(стандартное), или разрабатывается для конкретного документа.

В практическом смысле использование \latex\ позволяет (в порядке уменьшения
важности):
\begin{itemize}
  \item с помощью макросов и \TeX-программирования реализовывать любые стили и
  самую сложную верстку, существует множество готовых пакетов для верстки
  графических химических формул, разнообразных схем, транскрипционных знаков,
  внезапно электронных схем, цветных листингов и т.п. 
  \item автоматизировать работу с документами: пересобирать выходные файлы через
  \make, генерировать части документов с помощью своих скриптов\note{отчеты,
  стандартные формы, результаты работы любых программ}
  \item получить выходой документ в .pdf .html .txt .PostScript .djvu \ldots с
  кликабельными ссылками, анимированными, а иногда и интерактивными элементами
  \item не использовать файлы документов в закрытом формате
  \item легко держать набор файлов в \vcs
  \item не покупать текстовый процессор
\end{itemize}

Особенно важен пункт про сложную верстку: она всегда нужна в крупных технических
публикациях, особенно в учебной литературе, или отчетных работах. Вам
обязательно понадобиться вставлять графики экспериментальных данных, тематически
специфичные схемы, листинги, выходные данные работы ваших пограмм и т.п.

Традиционно \latex\ любим математиками, и всеми кто готовит публикации с большим
количеством формул и перекрестных ссылок: после небольшого обучения формулы
вводятся с листа со скоростью набора текста, особенно если ваш редактор умеет
\hyperref[autocomplition]{автодополнение}, и никакой мышиной возьни.

Естественно всякие чисто автоматические вещи типа автонумерации ссылок и формул,
сборки оглавлений и индексов, цветовая подсветка синтаксиса в листингах
программ, размещение \hyperref[floatfig]{плавающих иллюстраций} и т.п.
выполняются автоматически \TeX-процессором в пакетном режиме, и на выходе
получается красивый печатный или электронный (.pdf) документ.

Единственная область, не удобная в \latex-верстке\ --- создание сложных таблиц.
Для этого были созданы визуальные редакторы, позволяющие отрисовать структуру
таблицы мышью, а затем заполнить готовый шаблон данными.

\section{Установка MikTeX (win32)}
\section{Структура документа}
\subsection{Заголовочный файл или блок}
\subsection{Стили документа}
\subsection{Пакеты}
\subsection{Автор и название}
\subsection{Верстка титульных страниц}
\subsection{Оглавление}
\section{Верстка слайдов}
\section{Список литературы и цитирование}
\section{Команды секционирования: часть, глава, раздел,..}
\section{Таблицы}
\section{Формулы}
\section{Перекрестные ссылки и гипессылки}
\section{Листинги скриптов и текстовых данных}
\section{Подготовка иллюстраций}
\subsection{Графики GNUPLOT}
\subsection{Схемы и графы в GraphViz}


\part{Символьная и численная математика}

В практике любого инженера математика занимают главную роль. Без хорошего знания
математики, причем практически всех областей, от школьной до дифференциального
исчисления, работать в этой области практически невозможно.

Прежде всего свободное знание математики, физики, и химии необходимо для чтения
любой литературы, если вам нужно разобраться в какой-либо прикладной области.
Очень часто приходится реализовывать некоторые численные методы вычислений,
выполняющиеся в вашем устройстве в реальном времени, для управления процессами,
обработки сигналов с датчиков, принятия решений о включении исполнительных
устройств и т.п. Ну и конечно вы не сможете создать само устройство, не понимая
принципы его работы \smiley. Это конечно не относится к различным простейшим
устройствам типа таймеров или простой автоматики, но стоимость заказов такого
типа $\rightarrow 0$.

Если вы хотите поднять или восстановить свой уровень знания базовых наук (а
заодно и английского), удобно воспользоваться ресурсом
\url{https://www.khanacademy.org/}: это знаменитая on-line академия \textbf{Khan
Academy}, имеющая как набор видеолекций по базовым техническим наукам, так и
большую батарею тестов для проверки ваших знаний. Не забывайте периодически
проходить все тесты, чтобы поддерживать свои знания рабочими. Из недостатков\
--- отвратнейшая реализация на мобильных устройствах, часть тестов просто не
работает, а ввод ответов крайне неудобен из-за необходимости постоянно
пользоваться (полно)экранной клавиатурой и переключения на числовой ввод.

На русском языке ресурсов такого класса к сожалению пока не попадалось.
Кое-что есть кусочками, но по большей части только лекции в стиле <<книжкой по
башке>>, похоже навыков начального обучения в России просто не существует. Если
есть силы и желание, можете сами реализовать проект по созданию онлайн системы
базового образования \smiley.

\chapter{Пакет Maxima}

\section{Установка Maxima под \win}



\part{Куча}

В этот раздел собраны все материалы, не вошедшие в основную часть потому что
слишком сложны для начинающих, не попадают не в один раздел по тематике, или
не вписались по каким-то другим параметрам.

Все новые материалы также сначала попадают сюда, а потом принимается решение об
их переносе в основную часть.

Часто сюда пишут статьи те, кто принимает участие в создании книги эпизодически,
или те, у кого нет достаточно времени заниматься их подготовкой.




\addcontentsline{toc}{chapter}{Список литературы}
\printbibliography

\end{document}
