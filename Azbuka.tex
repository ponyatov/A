\documentclass[oneside,12pt]{book}

% e-book format
\usepackage[paperwidth=210mm,paperheight=148mm,margin=10mm]{geometry}

% Cyrillization
\usepackage[T1,T2A]{fontenc}
\usepackage[utf8]{inputenc}
\usepackage[english,russian]{babel}
\usepackage{indentfirst}

% font setup for screen reading
\renewcommand{\familydefault}{\sfdefault}
\normalfont

% pdflatex options
\usepackage[unicode,colorlinks,linkcolor=blue,bookmarks=true]{hyperref}
\usepackage[pdftex]{graphicx}

% listings
\usepackage{verbatim}
\usepackage{listings}
\lstset{
extendedchars=true,inputencoding=utf8, % i18n
frame=single, % show frames around
numbers=left, numberstyle=\small,numbersep=1mm,% line numbering
tabsize=4, % tab style
% keywordstyle=\color{blue}\texttt,
% commentstyle=\color{magenta}\texttt,
% showspaces=false
}
\newcommand{\lst}[3]{\lstinputlisting[title=\href{#2}{#1}]{#3}}

% software menu & keys
\usepackage[os=win]{menukeys} 
\usepackage{amssymb} % windows key
\newcommand{\winstart}{$\boxplus$}
\newcommand{\winr}{\keys{\winstart+R}}
\newcommand{\file}[1]{\textbf{\textsf{#1}}}
\newcommand{\lms}{$\lhd$}
\newcommand{\dblms}{$\lhd\lhd$}
\newcommand{\rms}{$\rhd$}
\newcommand{\checkbox}{$\boxtimes$}
\newcommand{\uncheckbox}{$\square$}

% disable oneliner page breaks
\usepackage[defaultlines=2,all]{nowidow}

% books bib management
\usepackage{biblatex}
\addbibresource{../bib/python.bib}
\addbibresource{../bib/eskd.bib}
\addbibresource{../bib/electronics.bib}
\addbibresource{../bib/latex.bib}
\addbibresource{../bib/sat.bib}


% extra char sets
\usepackage{wasysym} % smileys

% set lists style
% \usepackage{enumitem}
% \setlist{nosep}

% misc

% \usepackage{titling}

\newcommand{\email}[1]{$<$\href{mailto:#1}{#1}$>$}
\newcommand{\internet}{Internet}

\newcommand{\cm}[1]{Cortex-M#1}
\newcommand{\cmx}{\cm{x}}

\newcommand{\linux}{Linux}
\newcommand{\emlinux}{em\linux}

\newcommand{\cpp}{$C^{+}_{+}$}
\newcommand{\py}{Python}

\newcommand{\vcs}{\hyperref[vcs]{VCS}}
\newcommand{\make}{\hyperref[make]{Make}}
\newcommand{\spice}{ngSPICE}
\newcommand{\latex}{\LaTeX}

\newcommand{\eclipse}{\textcircled{$\equiv$}\textsc{eclipse}}
\newcommand{\vim}{(g)Vim}

\newcommand{\note}[1]{\footnote{\ #1}}
\newcommand{\cp}[1]{\note{копипаста \url{#1}}}

\newcommand{\win}{\winstart Windows}

\newcommand{\mk}{МК}

\newcommand{\term}[1]{\emph{#1}}

\newcommand{\pref}[1]{/стр.\pageref{#1}/}

% math
\usepackage{cancel}

% titles

\hypersetup{
	pdftitle={Азбука халтурщика-ARMатурщика},
	pdfauthor={ruOpenWrt, HackSpace <<Чебураторный завод>>, Консорциум хоббитов
	России, Bill Collis (Часть 1)}, 
	pdfsubject={https://github.com/ponyatov/Azbuka}
}


\author{\copyright\
\href{https://groups.google.com/forum/\#!forum/openwrt2ru}{ruOpenWrt}}
\title{Азбука халтурщика-ARMатурщика}

\begin{document}
\maketitle
\tableofcontents

\section*{Введение}

\part{Основы электроники}

Здесь идет список ссылок на онлайн лекции в edX, Coursera, и т.п.

\chapter{Симуляция и расчет схем в \spice}

\chapter{Линейные схемы на пассивных элементах, основы элекротехники}

\chapter{Простейшие полупрводниковые элементы}

\section{Схемы на биполярных транзисорах} 

\section{Схемы на на полевых транзисорах}

\part{KiCAD} 

\chapter{Отрисовка схем в KiCAD}

\section{Передача схемы в \spice}

\chapter{Трассировка плат и подготовка производства в KiCAD}

\section{Технология ЛУТ (Лазерный УТюг)}

\section{Технология фоторезиста}

\section{Формат Gerber и подготвка промышленного производства}

\part{Технология РЭС}

\chapter{Эксплуатация станочного оборудования}

\chapter{Основы ЧПУ и цифрового производства}

\chapter{FreeCAD}

\part{Разработка ПО для встраиваемых систем}

\chapter{Основы Си и \cpp}

\section{Особенности \cpp\ в embedded}

\chapter{LLVM и разработка собственных компиляторов}

\section{Лексический и синтаксический анализ}

\section{Применение flex/bison для разбора текстовых форматов данных}

\section{Компилятор Паскаля}

\chapter{Сборка кросс-компилятора GNU toolchain}

\chapter{Язык Python}

\part{Микроконтроллеры \cmx}

\part{Периферия}

\part{Встраиваемый \emlinux}

\chapter{cross}

\chapter{BuildRoot}

\chapter{Особенности OpenWrt}

\end{document}
