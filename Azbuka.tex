\documentclass[oneside,12pt]{book}

% e-book format
\usepackage[paperwidth=210mm,paperheight=148mm,margin=10mm]{geometry}

% Cyrillization
\usepackage[T1,T2A]{fontenc}
\usepackage[utf8]{inputenc}
\usepackage[english,russian]{babel}
\usepackage{indentfirst}

% font setup for screen reading
\renewcommand{\familydefault}{\sfdefault}
\normalfont

% pdflatex options
\usepackage[unicode,colorlinks,linkcolor=blue,bookmarks=true]{hyperref}
\usepackage[pdftex]{graphicx}

% listings
\usepackage{verbatim}
\usepackage{listings}
\lstset{
extendedchars=true,inputencoding=utf8, % i18n
frame=single, % show frames around
numbers=left, numberstyle=\small,numbersep=1mm,% line numbering
tabsize=4, % tab style
% keywordstyle=\color{blue}\texttt,
% commentstyle=\color{magenta}\texttt,
% showspaces=false
}
\newcommand{\lst}[3]{\lstinputlisting[title=\href{#2}{#1}]{#3}}

% software menu & keys
\usepackage[os=win]{menukeys} 
\usepackage{amssymb} % windows key
\newcommand{\winstart}{$\boxplus$}
\newcommand{\winr}{\keys{\winstart+R}}
\newcommand{\file}[1]{\textbf{\textsf{#1}}}
\newcommand{\lms}{$\lhd$}
\newcommand{\dblms}{$\lhd\lhd$}
\newcommand{\rms}{$\rhd$}
\newcommand{\checkbox}{$\boxtimes$}
\newcommand{\uncheckbox}{$\square$}

% disable oneliner page breaks
\usepackage[defaultlines=2,all]{nowidow}

% books bib management
\usepackage{biblatex}
\addbibresource{../bib/python.bib}
\addbibresource{../bib/eskd.bib}
\addbibresource{../bib/electronics.bib}
\addbibresource{../bib/latex.bib}
\addbibresource{../bib/sat.bib}


% extra char sets
\usepackage{wasysym} % smileys

% set lists style
% \usepackage{enumitem}
% \setlist{nosep}

% misc

% \usepackage{titling}

\newcommand{\email}[1]{$<$\href{mailto:#1}{#1}$>$}
\newcommand{\internet}{Internet}

\newcommand{\cm}[1]{Cortex-M#1}
\newcommand{\cmx}{\cm{x}}

\newcommand{\linux}{Linux}
\newcommand{\emlinux}{em\linux}

\newcommand{\cpp}{$C^{+}_{+}$}
\newcommand{\py}{Python}

\newcommand{\vcs}{\hyperref[vcs]{VCS}}
\newcommand{\make}{\hyperref[make]{Make}}
\newcommand{\spice}{ngSPICE}
\newcommand{\latex}{\LaTeX}

\newcommand{\eclipse}{\textcircled{$\equiv$}\textsc{eclipse}}
\newcommand{\vim}{(g)Vim}

\newcommand{\note}[1]{\footnote{\ #1}}
\newcommand{\cp}[1]{\note{копипаста \url{#1}}}

\newcommand{\win}{\winstart Windows}

\newcommand{\mk}{МК}

\newcommand{\term}[1]{\emph{#1}}

\newcommand{\pref}[1]{/стр.\pageref{#1}/}

% math
\usepackage{cancel}

% titles

\hypersetup{
	pdftitle={Азбука халтурщика-ARMатурщика},
	pdfauthor={ruOpenWrt, HackSpace <<Чебураторный завод>>, Консорциум хоббитов
	России, Bill Collis (Часть 1)}, 
	pdfsubject={https://github.com/ponyatov/Azbuka}
}


\author{\copyright\
\href{https://groups.google.com/forum/\#!forum/openwrt2ru}{ruOpenWrt}\\
\copyright\
HackSpace
<<\href{https://github.com/ponyatov/CHBZ/raw/master/presentation.pdf}{Чебураторный
завод}>>\\
\copyright\
Консорциум хоббитов России
}

\title{
\textbf{Азбука халтурщика-ARMатурщика}\\
разработка встраиваемых систем\\
основы бытовой автоматики,\\
систем управления и сбора данных
}

\begin{document}
\maketitle
\tableofcontents\clearpage

Linux для встраиваемых систем\note{будем называть его \emlinux}\ ---
популярный метод быстрого создания комплекса ПО для больших сложных приложений,
работающих на достаточно мощном железе, особенно предполагающих интенсивное
использование сетевых технологий.

За счет использования уже существующей и очень большой базы исходных текстов
ядра, библиотек и программ для \linux, бесплатно доступных в т.ч. и для
коммерческих приложений, можно на порядки сократить стоимость разработки
собственных программных компонентов, и при этом получить готовую команду
бесплатных стронних разработчиков, уже знакомых с созданием ПО для \linux.

Из недостатков можно отметить:
\begin{itemize}
  \item Отсутствие полноценной поддержки режима жесткого реального времени;
  \item Тяжелое ядро;
  \begin{itemize}
  \item Поддерживаются только мощные семейства процессоров\note{32-бит,
  необходим блок MMU};
  \item Значительные требования по объему \ram\ и общей производительности;
  \end{itemize}
  \item Дремучесть техспециалистов, контуженных ТурбоПаскалем и
Win\-dows\-ом;
\end{itemize}

Для \term{сборки}\ \emlinux-системы используется метод \term{кросс-компиляции},
когда используется \term{кросс-тулчейн}, компилирующий весь комплект ПО для
компьютера с другой архитектурой. Типичный пример\ --- сборка ПО на
ПК c процессором Intel i7 для Raspberry Pi или планшета на процессоре
AllWinner/Tegra/\ldots.

\emlinux\ очень широко применяется на рынке мобильных устройств\note{в т.ч.
является основой Android}, и устройств интенсивно использующих сетевые протоколы
(роутеры, медиацентры).

В качествe примера применения рассмотрим относительно простое приложение:
многофункциональные настенные часы с синхронизацией времени через \internet, с
будильником, медиапроигрывателем, блэкджеком и плюшками.


\part{Основы электроники}

Здесь идет список ссылок на онлайн лекции в edX, Coursera, и т.п.

\chapter{Линейные схемы на пассивных элементах, основы электротехники}

\chapter{Симуляция и расчет схем в \spice}

\secrel{САПР электронных устройств KiCAD}\label{kicad}\secdown\secdown

\noindent\includegraphics[height=0.5\textheight]{tmp/icon_kicad.png}

\cp{http://teholabs.com/knowledge/kicad.html}


\cp{http://ru.wikibooks.org/wiki/KiCad}

KiCad\ --- распространяемый по лицензии GNU GPL программный комплекс САПР EDA с
открытыми исходными текстами, предназначенный для разработки электрических схем
и печатных плат.

Кроссплатформенность компонентов KiCad обеспечивается использованием 
библиотеки wxWidgets. Поддерживаются операционные системы Linux, 
Windows NT 5.x, Free\-BSD и Solaris.

Разработчик\ --- Жан-Пьер Шарра (фр. Jean-Pierre Charras), исследователь 
в LIS (фр. Laboratoire des Images et des Signaux\ --- Лаборатория Изображений 
и Сигналов) и преподаватель электроники и обработки изображений в фр. 
IUT de Saint Martin d’Hères (Франция).

\cp{http://ru.wikibooks.org/wiki/KiCad/Miniurok}

Этот раздел познакомит Вас с основами использования системы KiCad. Он содержит
информацию о всех шагах создания простой печатной платы: от рисования
электрической схемы до печати готового рисунка платы. Вам будут представлены
различные возможности KiCad и предложены эффективные пути решения различных
задач.

Руководство пользователя, поставляемое вместе с KiCad, содержит значительно
больше информации, чем этот урок. Ознакомтесь с ним, чтобы узнать больше об
использовании программы.



\section{Установка под \win}\label{kicadinst}

\menu{\winr{\url{http://www.kicad-pcb.org/}}>Download>\winstart}

\menu{\winr{\url{http://kicad.nosoftware.cz/}}>
\file{KiCad\_testing\-201x.xx.xx-BZRxxxx\_Win\_full\_version.exe}}

\bigskip

\menu{Installer Language>\emph{English}>Ok} 
в русифицированном инсталляторе кривые шрифты

\menu{KiCAD 20xx.xx.xx Setup>Next}

\menu{Лицензия>Agree}

\menu{Components>\checkbox\ все>Next}

\menu{Location>\file{C:/KiCad}>Install}

\menu{Completing Setup>\checkbox Wings3D>Finish}

\subsection{Установка \prog{Wings3D}}

\menu{\winr{\url{http://www.wings3d.com/}}>Downloads>Stable Release>\win
(32/64b)}

\menu{file{wings-n.n.n.exe}}

\menu{Compononets>\checkbox QuickLaunch>Next}

\menu{Location>\file{C:/Program Files/wings3d\_n.n.n}>Next>Install>Close}% 


\clearpage\secrel{Маршрут проектирования}

\noindent
\includegraphics[height=0.92\textheight]{kicad/march.pdf}


\secrel{Создание проекта в менеджере проектов \prog{kicad}}

\win: \menu{\winstart>Программы>KiCAD>KiCAD}

\linux: \verb|user@host$ kicad|

\bigskip
\includegraphics[height=0.8\textheight]{kicad/projman.png}

\bigskip
В верхней части панели \term{менеджера проектов} \prog{kicad} имеются большие
кнопки запуска компонентов KiCad:

\begin{itemize}
\item\icon{kicad/icon_eeschema.png}
\prog{eeschema}\ --- Редактор принципиальных схем
\item\icon{kicad/icon_pcbnew.png}
\prog{pcbnew}\ --- Редактор печатных плат
\item\icon{kicad/icon_cvpcb.png}
\prog{cvpcb}\ --- Программа редактирования \termdef{падстеков}{падстек}
(отверстий и площадок)
\end{itemize}

Каждая кнопка запускает соответствующую программу. Мы будем использовать эти
программы по мере изучения.


\begin{itemize}
\item
\includegraphics[height=0.1\textheight]{kicad/icon_gerbview.png}
\prog{gerbview}\ --- Программа просмотра фотошаблонов в формате Gerber
\item
\prog{bitmap2component}\ --- Создание компонента из черно-белого изображения
(например логотипа)
\item
\includegraphics[height=0.1\textheight]{kicad/icon_pcbcalculator.png}
\prog{PcbCalculator}\ --- Калькулятор для печатных плат
\item
\includegraphics[height=0.1\textheight]{kicad/icon_pagelayout.png}
\prog{PageLayout}\ --- редактор формата листа схемы
\end{itemize}

% % \bigskip
% % Лучше всего для каждого проекта использовать раздельные папки; в противном
% % случае система может сбиться с толку, если файлы из разных проектов будут лежать
% % в одной папке. Проделайте следующие шаги:
% % 
% % % \begin{enumerate}
% % %   \item Создайте папку проекта \file{D:/ARM/SpindleDriver}
% % %   \item Запустите программу KiCad
% % %   \item Создайте проект (project)
% % %   \begin{itemize}
% % %     \item 
% % % На панели инструментов KiCad выберите левую иконку с подсказкой\\
% % % \menu{Начать новый проект}, используйте команду меню
% % % \menu{Файл>Новый>Пустой} или сочетание клавиш \keys{Ctrl+N}.
% % %     \item 
% % % В диалоге \menu{Создать новый проект} выберите созданную папку
% % % выберите только что созданную папку \file{D:/ARM/SpindleDriver} и
% % % введите имя проекта \menu{\file{SpindleDriver}} и нажмите \menu{Сохранить}.
% % % 	\item
% % % Если папка проекта содержит какие-то файлы, будет выведено окно выбора:
% % % создать подпапку с именем проекта \menu{Yes}, или записать файл проекта
% % % в указанную папку как есть \menu{No}. Нажмите No.
% % %     \item 
% % % Сохраните проект кнопкой \menu{Сохранить текущий проект}, \menu{Файл>Сохранить}
% % % или \keys{Ctrl+S}.
% % % 	\item
% % % В папке появится файл \file{SpindleDriver.pro}, содержащий установки вашего 
% % % проекта. Файл имеет тектовый формат, поэтому при необходимости его можно открыть
% % % в любом редакторе и вручную аккуратно подправить, например скорректировать
% % % настройки зазоров печатной платы.
% % %   \end{itemize}
% % % \end{enumerate}

\secrel{Создание принципиальной схемы в \prog{eeschema}\ (часть 1)}


Запустите из менеджера проектов, графической оболочки или командной строки
\linux\ модуль \eeschema: 

\bigskip
\noindent\verb|user@host$ eeschema &|.

\bigskip
\noindent\includegraphics[height=0.1\textheight]{kicad/icon_eeschema.png}

\clearpage\noindent
\includegraphics[width=\textwidth]{kicad/ee15.png}

На правом краю окна редактора схем есть вертикальная панель инструментов,
которые мы и будем использовать для рисования схемы. Этими инструментами можно
выбирать объекты, размещать компоненты, вводить связи и т.д.


% % % При первом запуске \prog{eeschema}\ стартует с новым проектом и
% % % показывает предупреждение, что файла схемы еще нет. Просто нажмите \menu{ОК}.
% % 
% % % Если вас не устраивает черный фон рабочец области или цвета элементов схемы,
% % % поменяйте настроки цветов \menu{Настройки>Цвета}.
% % % 
% % % Завершение работы инструмента: вы можете выбрать другой инструмент из правой
% % % инструментальной панели или же указать \menu{Отложить инструмент} по правому
% % % клику мышки \keys{\rms}.
% % 
% % \secrel{Инструмент \emph{Добавить компоненты}}
% % 
% % % \begin{itemize}
% % %   \item 
% % % На правой панели нажмите кнопку \menu{Разместить компонент}\
% % % \includegraphics[height=2em]{kicad/ee21.png}. Курсор изменится со стрелки на
% % % карандаш. Удобнее использовать сочетание клавиш \keys{Shift+A}.
% % % Кликните в поле схемы чтобы начать размещение компонента. Появится диалог
% % % \menu{Выбор компонента}. Вы можете выбрать компонент несколькими путями:
% % %   \item
% % %   \begin{enumerate}
% % %     \item 
% % % Если вы знаете точное имя копонента, введите его в поле \menu{Имя}, а
% % % затем нажмите \keys{Enter} или \keys{OK}.
% % %     \item 
% % % Если вы знаете имя только приблизительно, в поле \menu{Имя} введите шаблон для
% % % поиска, например, \menu{*BD*}, затем нажмите \keys{Enter} или \keys{OK}. Вы
% % % увидите окно \\\menu{Выбрать компонент} со списком найденных компонентов.
% % % 
% % % \includegraphics[height=0.5\textheight]{kicad/ee16.png}
% % %     \item 
% % % Вы можете искать компонент по ключевому слову, введя его в поле \menu{Имя},
% % % затем кликнув \menu{Поиск по ключевому слову}. Однако на данный момент качество
% % % библиотек все еще низкое, и немногие компоненты имеют ключевые слова, поэтому
% % % эта возможность полезна косвенно.
% % %     \item 
% % % Можно выбрать недавно использованные компоненты из \menu{Списка истории}.
% % %     \item 
% % % Кнопка \menu{Список всех} вызывает диалог, в котором можно выбрать сначала
% % % библиотеку \menu{74xx}, а затем ее компонент \menu{74HCT04}.
% % %     \item 
% % % Кнопка \menu{Выбор просмотром} вызывает \menu{Обзор библиотек}, позволяя
% % % просмотреть библиотеки и находящиеся в них условные графические изображения.
% % % 
% % % \includegraphics[height=0.5\textheight]{kicad/ee19.png}
% % % 
% % %   \end{enumerate} 
% % % \end{itemize}
% % % 
% % % Вы также можете
% % % вызвать обозреватель библиотек кнопкой\\
% % % \menu{Просмотр библиотек и
% % % компонентов}\ \includegraphics[height=2em]{kicad/ee20.png}
% % % 
% % % Выбрав элемент \dblms, вставьте символ в нужное место схемы \lms.
% % % Позже вы сможете переместить его если нужно.
% % % Зеркальное отражение компонента можно произвести следующим образом:
% % % 
% % % \begin{itemize}
% % %   \item Поместите курсор на компоненте.
% % %   \item По \rms\ выберите \menu{Ориентация компонента>Отражение}. 
% % %   \item Без использования \term{контекстного меню}\ --- наведите мышь на
% % %   компонент и нажмите кнопку \keys{X}\ или \keys{Y}.
% % % \end{itemize}


\input{kicad/eeschema}

% % \input{kicad/libs}
% % 
% % \section{\prog{gerbview}: просмотр фотошаблонов}

позволяет просматривать Gerber-файлы перед передачей печатных плат в
производство.

% % 
% % \secrel{Программа \prog{Wings3D} для создания 3D моделей}

Эта программа может вам пригодиться если вы планируете создавать 3D модели для PCB элементов.

Архив и файлы документации (Linux и Windows) в папке kicad/wings3d.

Взгляните на домашнюю страницу Wings3D чтобы узнать подробнее о программе.

pcbnew использует файлы в формате wrml (.wrl) экспортируемые из Wings3D (родной формат Wings3D - это .wings).

\secdown
\secrel{Установка \prog{Wings3D} под \win}

\menu{\winr{\url{http://www.wings3d.com/}}>Downloads>Stable Release>\win
(32/64b)}

\menu{file{wings-n.n.n.exe}}

\menu{Compononets>\checkbox QuickLaunch>Next}

\menu{Location>\file{C:/Program Files/wings3d\_n.n.n}>Next>Install>Close}% 
\secup


\secup
\secup


\chapter{Простейшие полупроводниковые элементы}

\section{Оптоэлектроника}

\section{Схемы на биполярных транзисорах} 

\section{Схемы на на полевых транзисорах}

\chapter{Операционные усилители}

\chapter{Источники питания}

\section{Батарейное питание}

\section{Линейные стабилизаторы}

\section{Импульсные преобразователи на ШИМ-контроллерах} 

\section{Цепи защиты и гашения кондуктивных помех}

\chapter{Цифровая электроника}

\chapter{Компьютерные интерфейсы}

\section{Поколение 90х: COM, LPT, ISA}

\subsection{Резервный программатор AVR ``пять проводков''}

\section{Сеть CAN}

\section{Интерфейсные модули USB}

\subsection{Универсальный высокоскоростной конвертер FTDI FT2232H}

\subsection{JTAG-адаптер}

\subsection{Отладочный модуль CAN}

\section{Интерфейсные модули Ethernet}

\chapter{ПЛИС}

\chapter{Датчики}

\chapter{Электропривод и исполнительные устройства}

\part{Основы конструирования РЭС}

\chapter{Пакеты моделирования на основе OpenFOAM}

\chapter{Обеспечение теплового режима}

\chapter{Электромагнитная совместимость}

\section{Кондуктивные помехи}

\section{Компоновочные модели и оптимизация кабельной сети}

\part{Технология РЭС}

\chapter{Трассировка плат и подготовка производства в KiCAD}

\section{Технология ЛУТ (Лазерный УТюг)}

\section{Технология фоторезиста}

\section{Формат Gerber и подготвка промышленного производства}

\chapter{FreeCAD}

\section{Чертеж}

\section{Эскиз}

\section{Деталь}

\section{Сборка}

\section{Автогенерация конструкторской докуметации}

\section{Скрипты и пользовательские расширения}

\chapter{Эксплуатация станочного оборудования}

\chapter{Основы ЧПУ и цифрового производства}

\section{CAM-пакеты для FreeCAD}

\part{Основы теории систем автоматического управления}

\chapter{Математический аппарат}

\section{Передаточная функция}

\section{Устойчивость САУ}

\section{Сети Петри}

\section{Автоматы Маркова}

\chapter{Релейное управление}

\chapter{Пропорциональные САУ}

\chapter{ПИДn-регуляторы}

\part{Разработка ПО для встраиваемых систем}

\chapter{Вспомогательные скрипты на языке Python}

\chapter{Make: управление сборкой проектов}\label{make}

\chapter{VCS: cистемы контроля версий}\label{vcs}

\section{CVS}

\section{Subversion}

\section{Git}

\subsection{GitHub}

\chapter{Основы Си и \cpp}

\subsection{Установка MinGW (win32)}

\section{Особенности \cpp\ в embedded}

\chapter{LLVM и разработка собственных компиляторов}

\section{Лексический и синтаксический анализ}

\section{Применение flex/bison для разбора текстовых форматов данных}

\section{Компилятор Паскаля}

\chapter{Сборка кросс-компилятора GNU toolchain}

\part{Микроконтроллеры \cmx}

\part{Периферия}

\part{Встраиваемый \emlinux}

\chapter{cross}

\chapter{BuildRoot}

\chapter{Особенности OpenWrt}

\chapter{Библиотека SDL}

\section{Реализация microGUI}

\chapter{Приложения для X Window}

\chapter{Программирование сетевых приложений}

\chapter{Сборка кросс-компиляторя GNU мальтийским крестом}

\chapter{\eclipse}\label{eclipse}

\includegraphics[height=0.5\textheight]{logo/eclipse.png}

\section{Установка \eclipse\ под \win}

\menu{\winr>\url{https://eclipse.org/}>Download}

\menu{Eclipse Luna release for>\win}

Качаем архив базовой системы:

\menu{Eclipse IDE for Java Developers>\win\ 32/64 Bit}

Или сразу сборку CDT\eclipse:

\menu{Eclipse IDE for C/C++ Developers>\win\ 32/64 Bit}

\section{Установка \eclipse\ под \linux}

\menu{\winr>\url{https://eclipse.org/}>Download}

\menu{Eclipse Luna release for>\win}

Качаем архив базовой системы:

\menu{Eclipse IDE for Java Developers>\linux\ 32/64 Bit}

Или сразу сборку CDT\eclipse:

\menu{Eclipse IDE for C/C++ Developers>\linux\ 32/64 Bit}

\bigskip

Устанавливаем в систему Java-рантайм:

\begin{verbatim}
sudo aptitude install openjdk-7-jre
\end{verbatim}

Распаковывем полученный архив
\file{eclipse-java-luna-SR1-linux-gtk-x86\_64.tar.gz}
в \file{\$HOME}:

\begin{verbatim}
cd ~
tar zx < Downloads/eclipse-java-luna-SR1-linux-gtk-x86_64.tar.gz 
ls -la eclipse/eclipse
-rwxr-xr-x 1 user user 74675 Авг 13 16:12 eclipse/eclipse
\end{verbatim}

Прописываем запуск \eclipse\ в ваш оконный менеджер или \file{.blackboxmenu}
с параметром \file{-noSplash} для лечения глюка с запуском на x64-битных
системах:

\lst{.blackbox.menu}{}{ide/eclipse_blackbox.menu}

\section{Установка CDT}

\href{https://eclipse.org/cdt/}{\prog{CDT}}\ --- расширение \eclipse\ для
разработки на Си/\cpp, редактирования make-файлов. Это расширение критически
важно для вашей работы, поэтому ставить его обязательно, или сразу качать сборку
CDT\eclipse.
\bigskip

\menu{\eclipse>Help>Install New Software\ldots}

\menu{Work with>Add>Add repository}

\menu{Name>CDT}

\menu{Location>\url{http://download.eclipse.org/tools/cdt/releases/8.5}}

\menu{OK}

\menu{Work with>CDT}

\menu{CDT Main Features>\checkbox\ C/C++ Development Tools}

\menu{CDT Optional Features}

Парсер файлов исходников на диалекте С99:
\menu{\checkbox\ C99 LR Parser}

Поддержка \prog{gcc}\ в режиме кросс-компиляции:
\menu{\checkbox\ GCC Cross Compiler Support}

Аппаратная отладка через \prog{gdb}:
\menu{\checkbox\ GDB Hardware Debugging}

\menu{Next>Next>Accept>Finish}

\section{Установка TeXlipse}

Если планируете работать с документацией в формате \latex, установите расширение
\href{http://texlipse.sourceforge.net/}{\prog{TeXlipse}}:
\bigskip

\menu{\eclipse>Help>Install New Software\ldots}

\menu{Work with>Add>Add repository}

\menu{Name>TeXlipse}

\menu{Location>\url{http://texlipse.sourceforge.net/}}

\menu{OK}

\menu{Work with>TeXlipse}

Это расширение поддерживает подсветку синтаксиса, автодополнение, построение
динамического оглавления, автокомпиляцию по сохранению, и несколько визардов
создания проекта.

\section{Редактирование файлов в формате XML и производных}

Установите пакет \eclipse\ WST:

\menu{Help>Install New Software}

\menu{Work with:>Luna - http://download.eclipse.org/releases/luna}

\menu{Filter:>WST>Eclipse WST>Next>Next>Restart>OK}

\section{Проверка орфографии}

\cp{http://www.simplecoding.org/proverka-orfografii-v-eclipse.html}

То, что проверка орфографии очень удобная вещь вряд ли нужно объяснять. Есть
конечно люди, которые не обращают на неё внимание, но это чаще всего из-за
экономии времени и отсутствия удобных средств проверки.

Действительно, удобная автоматическая проверка орфографии есть в офисных
пакетах, но мне сложно представить разработчика, который будет переносить
комментарии в Word и обратно \smiley.

Поэтому очень удобно иметь \emph{проверку правописания прямо в IDE}. И \eclipse\
в этом смысле полностью соответствует ожиданиям.

Долго объяснять что к чему нет смысла. Проверка орфографии встроена в \eclipse\
и если вы пишите только на английском, то может быть не захотите ничего менять.

Кроме того, есть
\href{http://www.102degrees.com/blog/2007/07/09/spell-checking-in-eclipse-pdt/}{статья
Aaron'а} (en) в которой автор рассказывает о подключении дополнительных словарей
и плагине \file{eSpell}.

Но \emph{русских словарей в дистрибутиве нет}, а при подключении внешних есть
нюансы. Поэтому мы максимально подробно рассмотрим \emph{подготовку и добавление
русских словарей}.

Первый вопрос. В каком виде должны быть словари и где их взять?

Тут всё просто. Формат словаря\ --- обычный текстовый файл, в котором каждое
слово начинается с новой строки. И нам вполне подойдут свободно распространяемые
словари \file{aSpell}.

Установка состоит из \ref{aspellecl}\ шагов:
\begin{enumerate}
  \item качаем \href{}{aSpell}\ и словари для нужных языков
  
  \menu{\winr>\url{http://aspell.net/win32/}>}
  
  \menu{Binaries>Full installer}
  
  \menu{Precompiled dictionaries>English}
  
  \menu{Precompiled dictionaries>Russian}
  
  \item устанавливаем сначала \file{aSpell}, потом отдельно каждый словарь
  
  \menu{\file{Aspell-0-50-3-3-Setup.exe}>Setup GNU Aspell>Next>License>Next}
  
  \menu{Directory>\file{C:/GnuWin32/Aspell}>Next>Next}
  
  \menu{Additional>Next>Install>Next>\uncheckbox\ View manual>Finish}
  
  \menu{\file{Aspell-en-0.50-2-3.exe}>Aspell English Dictionary>Next>License>Next}
  
  \menu{Directory>\file{C:/GnuWin32/Aspell}>Next>Next>Install>Finish}
  
  \menu{\file{Aspell-ru-0.50-2-3.exe}>Aspell Russian Dictionary>Next>License>Next}
  
  \menu{Directory>\file{C:/GnuWin32/Aspell}>Next>Next>Install>Finish}
  
  \item делаем дамп словарей, перекодируем из koi8r в utf8 и объединяем
  
  \menu{\winr cmd}

\begin{lstlisting}
cd \GnuWin32\Aspell
bin\aspell dump master en > en.dict
bin\aspell dump master ru > ru.koi8
iconv -f koi8-r -t utf-8 < ru.koi8 > ru.dict
copy en.dict + ru.dict enru.dict
\end{lstlisting}
  
  \item \label{aspellecl} настраиваем \emph{spell-checker} \eclipse
  
  \menu{\eclipse>Window>Preferences>Editors>Text editors>Spelling}
  
  \menu{User defined dictionary>\file{C:/GnuWin32/Aspell/enru.dict}}
  
  \menu{Encoding>UTF-8}
  
  \menu{Apply>OK}
  
\end{enumerate}



\part{Подготовка публикаций в \latex}

\cp{https://ru.wikipedia.org/wiki/LaTeX}

LaTeX (по-русски произносится \textbf{лат\'eх})\ --- наиболее популярный набор
макрорасширений (или макропакет) системы компьютерной вёрстки \TeX, который
облегчает набор сложных документов. В типографском наборе форматируется как
\LaTeX.

Главная идея \latex\ состоит в том, что авторы должны думать о содержании, о
том, что они пишут, не беспокоясь о конечном визуальном облике (печатный
вариант, текст на экране монитора или что-то другое). Готовя свой документ,
автор указывает логическую структуру текста (разбивая его на главы, разделы,
таблицы, изображения), а \latex\ решает вопросы его отображения. Так содержание
отделяется от оформления. Оформление при этом или определяется заранее
(стандартное), или разрабатывается для конкретного документа.

В практическом смысле использование \latex\ позволяет (в порядке уменьшения
важности):
\begin{itemize}
  \item с помощью макросов и \TeX-программирования реализовывать любые стили и
  самую сложную верстку, существует множество готовых пакетов для верстки
  графических химических формул, разнообразных схем, транскрипционных знаков,
  внезапно электронных схем, цветных листингов и т.п. 
  \item автоматизировать работу с документами: пересобирать выходные файлы через
  \make, генерировать части документов с помощью своих скриптов\note{отчеты,
  стандартные формы, результаты работы любых программ}
  \item получить выходой документ в .pdf .html .txt .PostScript .djvu \ldots с
  кликабельными ссылками, анимированными, а иногда и интерактивными элементами
  \item не использовать файлы документов в закрытом формате
  \item легко держать набор файлов в \vcs
  \item не покупать текстовый процессор
\end{itemize}

Особенно важен пункт про сложную верстку: она всегда нужна в крупных технических
публикациях, особенно в учебной литературе, или отчетных работах. Вам
обязательно понадобиться вставлять графики экспериментальных данных, тематически
специфичные схемы, листинги, выходные данные работы ваших пограмм и т.п.

Традиционно \latex\ любим математиками, и всеми кто готовит публикации с большим
количеством формул и перекрестных ссылок: после небольшого обучения формулы
вводятся с листа со скоростью набора текста, особенно если ваш редактор умеет
\hyperref[autocomplition]{автодополнение}, и никакой мышиной возьни.

Естественно всякие чисто автоматические вещи типа автонумерации ссылок и формул,
сборки оглавлений и индексов, цветовая подсветка синтаксиса в листингах
программ, размещение \hyperref[floatfig]{плавающих иллюстраций} и т.п.
выполняются автоматически \TeX-процессором в пакетном режиме, и на выходе
получается красивый печатный или электронный (.pdf) документ.

Единственная область, не удобная в \latex-верстке\ --- создание сложных таблиц.
Для этого были созданы визуальные редакторы, позволяющие отрисовать структуру
таблицы мышью, а затем заполнить готовый шаблон данными.

\section{Установка MikTeX (win32)}
\section{Структура документа}
\subsection{Заголовочный файл или блок}
\subsection{Стили документа}
\subsection{Пакеты}
\subsection{Автор и название}
\subsection{Верстка титульных страниц}
\subsection{Оглавление}
\section{Верстка слайдов}
\section{Список литературы и цитирование}
\section{Команды секционирования: часть, глава, раздел,..}
\section{Таблицы}
\section{Формулы}
\section{Перекрестные ссылки и гипессылки}
\section{Листинги скриптов и текстовых данных}
\section{Подготовка иллюстраций}
\subsection{Графики GNUPLOT}
\subsection{Схемы и графы в GraphViz}


\end{document}
