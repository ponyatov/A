\part{Символьная и численная математика}

В практике любого инженера математика занимают главную роль. Без хорошего знания
математики, причем практически всех областей, от школьной до дифференциального
исчисления, работать в этой области практически невозможно.

Прежде всего свободное знание математики, физики, и химии необходимо для чтения
любой литературы, если вам нужно разобраться в какой-либо прикладной области.
Очень часто приходится реализовывать некоторые численные методы вычислений,
выполняющиеся в вашем устройстве в реальном времени, для управления процессами,
обработки сигналов с датчиков, принятия решений о включении исполнительных
устройств и т.п. Ну и конечно вы не сможете создать само устройство, не понимая
принципы его работы \smiley. Это конечно не относится к различным простейшим
устройствам типа таймеров или простой автоматики, но стоимость заказов такого
типа $\rightarrow 0$.

Если вы хотите поднять или восстановить свой уровень знания базовых наук (а
заодно и английского), удобно воспользоваться ресурсом
\url{https://www.khanacademy.org/}: это знаменитая on-line академия \textbf{Khan
Academy}, имеющая как набор видеолекций по базовым техническим наукам, так и
большую батарею тестов для проверки ваших знаний. Не забывайте периодически
проходить все тесты, чтобы поддерживать свои знания рабочими. Из недостатков\
--- отвратнейшая реализация на мобильных устройствах, часть тестов просто не
работает, а ввод ответов крайне неудобен из-за необходимости постоянно
пользоваться (полно)экранной клавиатурой и переключения на числовой ввод.

На русском языке ресурсов такого класса к сожалению пока не попадалось.
Кое-что есть кусочками, но по большей части только лекции в стиле <<книжкой по
башке>>, похоже навыков \emph{вводного}\ обучения в России просто не существует.
Если есть силы и желание, можете сами реализовать проект по созданию онлайн
системы базового образования \smiley.

\bigskip
В этом разделе собраны примеры проектов, требующие некоторых базовых знаний, а
также рассмотрено использование OpenSource программ для вычислений и обработки
данных.
\clearpage

\chapter{Пакет Maxima}

Maxima\ -- пакет CAS\note{[C]omputer [A]lgebra [S]ystem}: \term{символьной}\
математики, но также включает функционал численных вычислений и визуализации.

\section{Установка Maxima под \win}

\menu{\winr>\url{http://maxima.sourceforge.net/ru/}>Загрузка}

\menu{Maxima-Windows>\file{5.34.1-Windows}>\file{maxima-5.34.1.exe}}

\menu{\file{maxima-5.34.1.exe}>Установка>Язык>\textbf{English}>OK>Next}

\menu{License>I accept>Next>}

\menu{Install folder>по умолчанию>Next}

\menu{Components>\uncheckbox\ Language packs>Next>Next}

\menu{\checkbox\ wxMaxima>\uncheckbox\ XMaxima>Next>Install>Next>Finish}

\menu{\winstart>Maxima-5.34.1>wxMaxima>\rms>Закрепить в панели задач}

\bigskip
Дополнительная документация:
\url{http://maxima.sourceforge.net/ru/documentation.html}

\bigskip
\href{https://drive.google.com/file/d/0B0u4WeMjO894M01wZmNkSW9GRHM/view?usp=sharing}{PDF}\
для книги Ильина В.А., Силаев П.К. Система аналитических вычислений Maxima для
физиков-теоретиков\ \cite{maxphis}\ получена из файла \file{.ps}\ с помощью
сервиса \url{http://ps2pdf.com/convert.htm}.

\section{Калькулятор}


