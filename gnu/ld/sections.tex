\secrel{Команда SECTIONS}\secdown

The SECTIONS command tells the linker how to map input sections into output
sections, and how to place the output sections in memory.

The format of the SECTIONS command is:

\begin{verbatim}
SECTIONS
{
  sections-command
  sections-command
  …
}
\end{verbatim}

Each sections-command may of be one of the following:

\begin{itemize}
  \item an ENTRY command (refer to Section 4.4.1 Setting the Entry Point)
  \item a symbol assignment (refer to Section 4.5 Assigning Values to Symbols)
  \item an output section description
  \item an overlay description
\end{itemize}

The ENTRY command and symbol assignments are permitted inside the SECTIONS
command for convenience in using the location counter in those commands. This
can also make the linker script easier to understand because you can use those
commands at meaningful points in the layout of the output file.

Output section descriptions and overlay descriptions are described below.

If you do not use a SECTIONS command in your linker script, the linker will
place each input section into an identically named output section in the order
that the sections are first encountered in the input files. If all input
sections are present in the first file, for example, the order of sections in
the output file will match the order in the first input file. The first section
will be at address zero.

\secrel{Описание выходных секций}

The full description of an output section looks like this:

\begin{verbatim}
section [address] [(type)] : [AT(lma)]
  {
    output-section-command
    output-section-command
    …
  } [>region] [AT>lma_region] [:phdr :phdr …] [=fillexp]
\end{verbatim}

Most output sections do not use most of the optional section attributes.

The whitespace around section is required, so that the section name is
unambiguous. The colon and the curly braces are also required. The line breaks
and other white space are optional.

Each output-section-command may be one of the following:

\begin{itemize}
  \item 
a symbol assignment (refer to Section 4.5 Assigning Values to Symbols)
  \item 
an input section description (refer to Section 4.6.4 Input Section Description)
  \item 
data values to include directly (refer to Section 4.6.5 Output Section Data)
  \item 
a special output section keyword (refer to Section 4.6.6 Output Section Keywords)
\end{itemize}

\secrel{Имя выходной секции}

The name of the output section is section. section must meet the constraints of
your output format. In formats which only support a limited number of sections,
such as a.out, the name must be one of the names supported by the format (a.out,
for example, allows only .text, .data or .bss). If the output format supports
any number of sections, but with numbers and not names (as is the case for
Oasys), the name should be supplied as a quoted numeric string. A section name
may consist of any sequence of characters, but a name which contains any unusual
characters such as commas must be quoted.

The output section name /DISCARD/ is special; refer to Section 4.6.7 Output
Section Discarding.


\secrel{Описание выходных секций}
\secrel{Описание входных секций}
\secrel{Выходная секция данных}
\secrel{Ключевые слова выходных секций}
\secrel{Отброс выходных секций}

\secrel{Атрибуты выходных секций}
\secrel{Описание оверлея}
\secup
