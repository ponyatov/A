\secrel{Формат объектного файла GNU ELF}\label{elf}

\begin{tabular}{l l l}
формат файла & elf32-i386 \\
архитектура & i386 \\
HAS\_SYMS & в файл включена отладочная информация \\& об идентификаторах
(``символах'')\\
start address & 0x00100000 & стартовый адрес загрузки ядра 1\,Мб\\
\end{tabular}

\bigskip
Бинарный код делится на \termdef{секции}{секция ELF}, или
\termdef{сегменты}{сегмент}:

\begin{tabular}{l l}
.text & машииный код программы \\
.rodata & данные: константы \\
.eh\_frame &\\
.data & данные: инициализированные массивы, строки \\
.bss & данные: пустые массивы под которые выделяется память при старте \\
.comment &\\
\end{tabular}

\bigskip
\begin{tabular}{l l}
CONTENTS & \\
ALLOC & \\
LOAD & \\
READONLY & \\
CODE & \\
DATA & \\
\end{tabular}

