\secrel{Линкер GNU ld}\secdown

\secrel{Использование ld}
\secrel{Обзор}

\secrel{Вызов}\secdown
\secrel{Параметры командной строки}
\secrel{Опции, специфичные для целевой платформы i386 PE}
\secrel{Переменные среды}
\secup

\secrel{Скрипты линкера}\label{ldscripts}

\secrel{Основные понятия скрипта линкера}
\secrel{Формат скрипта компоновщика}
\secrel{Пример простого скрипта  компоновщика}

\secrel{Команды простого скрипта компоновщика}\secdown
\secrel{Установка точки входа}
\secrel{Команды работы с файлами}
\secrel{Работа с форматами объектный файлов}
\secrel{Другие команды сценария компоновщика}
\secup

\secrel{Присвоение значений символам}\secdown
\secrel{Простые назначения}
\secrel{PROVIDE}
\secup

\secrel{Команда SECTIONS}\secdown
\secrel{Описание выходных секций}
\secrel{Имя выходной секции}
\secrel{Описание выходных секций}
\secrel{Описание входных секций}
\secrel{Выходная секция данных}
\secrel{Ключевые слова выходных секций}
\secrel{Отброс выходных секций}
\secrel{Атрибуты выходных секций}
\secrel{Описание оверлея}
\secup

\secrel{Команда MEMORY}
\secrel{Команда PHDRS}
\secrel{Команда VERSION}

\secrel{Выражения в скриптах компоновщика}\secdown
\secrel{Константы}
\secrel{Имена символов}
\secrel{Счетчик адреса}
\secrel{Операторы}
\secrel{Вырачисления}
\secrel{Часть выражения}
\secrel{Встроенные функции}
\secup

\secrel{Неявные скрипты компоновщика}

\secrel{Машинно-зависимые особенности}\secdown
\secrel{ld и H8/300}
\secrel{ld и семейство Intel 960}
\secrel{поддержка взаимодействия между кодом в режимах ARM и Thumb}
\secrel{ld и поддержка 32-разрядных ELF HPPA}
\secrel{ld и MMIX}
\secrel{ld и MSP430}
\secrel{поддержка различных версий TI COFF}
\secrel{ld и WIN32 (cygwin/mingw)}
\secrel{ld и процессоры Xtensa}
\secup

\secrel{BFD}\secdown
\secrel{Как это работает: очерк о BFD}
\secrel{Потеря информации}
\secrel{Канонический формат объектных файлов BFD}
\secup

\secup
