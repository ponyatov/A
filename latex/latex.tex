\part{Подготовка публикаций в \latex}

\cp{https://ru.wikipedia.org/wiki/LaTeX}

LaTeX (по-русски произносится \textbf{лат\'eх})\ --- наиболее популярный набор
макрорасширений (или макропакет) системы компьютерной вёрстки \TeX, который
облегчает набор сложных документов. В типографском наборе форматируется как
\LaTeX.

Главная идея \latex\ состоит в том, что авторы должны думать о содержании, о
том, что они пишут, не беспокоясь о конечном визуальном облике (печатный
вариант, текст на экране монитора или что-то другое). Готовя свой документ,
автор указывает логическую структуру текста (разбивая его на главы, разделы,
таблицы, изображения), а \latex\ решает вопросы его отображения. Так содержание
отделяется от оформления. Оформление при этом или определяется заранее
(стандартное), или разрабатывается для конкретного документа.

В практическом смысле использование \latex\ позволяет (в порядке уменьшения
важности):
\begin{itemize}
  \item с помощью макросов и \TeX-программирования реализовывать любые стили и
  самую сложную верстку, существует множество готовых пакетов для верстки
  графических химических формул, разнообразных схем, транскрипционных знаков,
  внезапно электронных схем, цветных листингов и т.п. 
  \item автоматизировать работу с документами: пересобирать выходные файлы через
  \make, генерировать части документов с помощью своих скриптов\note{отчеты,
  стандартные формы, результаты работы любых программ}
  \item получить выходой документ в .pdf .html .txt .PostScript .djvu \ldots с
  кликабельными ссылками, анимированными, а иногда и интерактивными элементами
  \item не использовать файлы документов в закрытом формате
  \item легко держать набор файлов в \vcs
  \item не покупать текстовый процессор
\end{itemize}

Особенно важен пункт про сложную верстку: она всегда нужна в крупных технических
публикациях, особенно в учебной литературе, или отчетных работах. Вам
обязательно понадобиться вставлять графики экспериментальных данных, тематически
специфичные схемы, листинги, выходные данные работы ваших пограмм и т.п.

Традиционно \latex\ любим математиками, и всеми кто готовит публикации с большим
количеством формул и перекрестных ссылок: после небольшого обучения формулы
вводятся с листа со скоростью набора текста, особенно если ваш редактор умеет
\hyperref[autocomplition]{автодополнение}, и никакой мышиной возьни.

Естественно всякие чисто автоматические вещи типа автонумерации ссылок и формул,
сборки оглавлений и индексов, цветовая подсветка синтаксиса в листингах
программ, размещение \hyperref[floatfig]{плавающих иллюстраций} и т.п.
выполняются автоматически \TeX-процессором в пакетном режиме, и на выходе
получается красивый печатный или электронный (.pdf) документ.

Единственная область, не удобная в \latex-верстке\ --- создание сложных таблиц.
Для этого были созданы визуальные редакторы, позволяющие отрисовать структуру
таблицы мышью, а затем заполнить готовый шаблон данными.

\section{Установка MikTeX (win32)}
\section{Структура документа}
\subsection{Заголовочный файл или блок}
\subsection{Стили документа}
\subsection{Пакеты}
\subsection{Автор и название}
\subsection{Верстка титульных страниц}
\subsection{Оглавление}
\section{Верстка слайдов}
\section{Список литературы и цитирование}
\section{Команды секционирования: часть, глава, раздел,..}
\section{Таблицы}
\section{Формулы}
\section{Перекрестные ссылки и гипессылки}
\section{Листинги скриптов и текстовых данных}
\section{Подготовка иллюстраций}
\subsection{Графики GNUPLOT}
\subsection{Схемы и графы в GraphViz}
