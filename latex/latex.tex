\part{Подготовка публикаций в \latex}

\cp{https://ru.wikipedia.org/wiki/LaTeX}

LaTeX (по-русски произносится \textbf{лат\'eх})\ --- наиболее популярный набор
макрорасширений (или макропакет) системы компьютерной вёрстки \TeX, который
облегчает набор сложных документов. В типографском наборе форматируется как
\LaTeX.

Главная идея \latex\ состоит в том, что авторы должны думать о содержании, о
том, что они пишут, не беспокоясь о конечном визуальном облике (печатный
вариант, текст на экране монитора или что-то другое). Готовя свой документ,
автор указывает логическую структуру текста (разбивая его на главы, разделы,
таблицы, изображения), а \latex\ решает вопросы его отображения. Так содержание
отделяется от оформления. Оформление при этом или определяется заранее
(стандартное), или разрабатывается для конкретного документа.

В практическом смысле использование \latex\ позволяет (в порядке уменьшения
важности):
\begin{itemize}
  \item с помощью макросов и \TeX-программирования реализовывать любые стили и
  самую сложную верстку, существует множество готовых пакетов для верстки
  графических химических формул, разнообразных схем, транскрипционных знаков,
  внезапно электронных схем, цветных листингов и т.п. 
  \item автоматизировать работу с документами: пересобирать выходные файлы через
  \make, генерировать части документов с помощью своих скриптов\note{отчеты,
  стандартные формы, результаты работы любых программ}
  \item получить выходой документ в .pdf .html .txt .PostScript .djvu \ldots с
  кликабельными ссылками, анимированными, а иногда и интерактивными элементами
  \item не использовать файлы документов в закрытом формате
  \item легко держать набор файлов в \vcs
  \item не покупать текстовый процессор
\end{itemize}

Особенно важен пункт про сложную верстку: она всегда нужна в крупных технических
публикациях, особенно в учебной литературе, или отчетных работах. Вам
обязательно понадобиться вставлять графики экспериментальных данных, тематически
специфичные схемы, листинги, выходные данные работы ваших пограмм и т.п.

Традиционно \latex\ любим математиками, и всеми кто готовит публикации с большим
количеством формул и перекрестных ссылок: после небольшого обучения формулы
вводятся с листа со скоростью набора текста, особенно если ваш редактор умеет
\hyperref[autocomplition]{автодополнение}, и никакой мышиной возьни.

Естественно всякие чисто автоматические вещи типа автонумерации ссылок и формул,
сборки оглавлений и индексов, цветовая подсветка синтаксиса в листингах
программ, размещение \hyperref[floatfig]{плавающих иллюстраций} и т.п.
выполняются автоматически \TeX-процессором в пакетном режиме, и на выходе
получается красивый печатный или электронный (.pdf) документ.

Единственная область, не удобная в \latex-верстке\ --- создание сложных таблиц.
Для этого были созданы визуальные редакторы, позволяющие отрисовать структуру
таблицы мышью, а затем заполнить готовый шаблон данными.

\section{Установка MikTeX под \win}
\section{Структура документа}
\subsection{Заголовочный файл или блок}
\subsection{Стили документа}
\subsection{Пакеты}
\subsection{Автор и название}
\subsection{Верстка титульных страниц}
\subsection{Оглавление}
\section{Верстка слайдов}
\section{Список литературы и цитирование}

\latex\ умеет мощную подсистему управления цитированием и списками литературы.
В простейшем случае, например при написании единственной статьи, раздел
\term{библиографии}\ можно создать в том же документе, добавив в конец
\verb|thebibliography|:

\begin{verbatim}
\documentclass{article}

\documentclass[oneside,12pt]{book}

% e-book format
\usepackage[paperwidth=210mm,paperheight=148mm,margin=10mm]{geometry}

% Cyrillization
\usepackage[T1,T2A]{fontenc}
\usepackage[utf8]{inputenc}
\usepackage[english,russian]{babel}
\usepackage{indentfirst}

% font setup for screen reading
\renewcommand{\familydefault}{\sfdefault}
\normalfont

% pdflatex options
\usepackage[unicode,colorlinks,linkcolor=blue,bookmarks=true]{hyperref}
\usepackage[pdftex]{graphicx}
\usepackage[usenames,dvipsnames,svgnames]{xcolor}

% listings
\usepackage{verbatim}
\usepackage{listings}
\lstset{
basicstyle=\small, % or \tiny \small or \footnotesize
extendedchars=true,inputencoding=utf8, % i18n
frame=single, % show frames around
numbers=left, numberstyle=\small,numbersep=1mm,% line numbering
tabsize=4, % tab style
keywordstyle=\color{Blue},%\texttt,
keywordstyle={[2]\color{Green}},%\texttt,
keywordstyle={[3]\color{Brown}},%\texttt,
keywordstyle={[4]\color{Red}},%\texttt,
keywordstyle={[5]\color{Blue}},%\texttt,
commentstyle=\color{Cyan}%\texttt%,
% showspaces=false
}

\usepackage{lstmk}\lstdefinestyle{mk}{language=mk}
\usepackage{lstrc}\lstdefinestyle{rc}{language=rc}

\newcommand{\lst}[3]{\lstinputlisting[title=\href{#2}{#1}]{#3}}
\newcommand{\lstx}[4]{\lstinputlisting[title=\href{#2}{#1},language=#4]{#3}}

% software menu & keys
\usepackage[os=win]{menukeys} 
\usepackage{amssymb} % windows key
\newcommand{\winstart}{$\boxplus$}
\newcommand{\winr}{\keys{\winstart+R}}
\newcommand{\file}[1]{\textbf{\textsf{#1}}}
\newcommand{\lms}{$\lhd$}
\newcommand{\dblms}{$\lhd\lhd$}
\newcommand{\rms}{$\rhd$}
\newcommand{\checkbox}{$\boxtimes$}
\newcommand{\uncheckbox}{$\square$}

% disable oneliner page breaks
\usepackage[defaultlines=2,all]{nowidow}

% books bib management
\usepackage{biblatex}
\addbibresource{../bib/python.bib}
\addbibresource{../bib/eskd.bib}
\addbibresource{../bib/electronics.bib}
\addbibresource{../bib/latex.bib}
\addbibresource{../bib/sat.bib}
\addbibresource{../bib/math.bib}
\addbibresource{../bib/sysdesign.bib}

\usepackage{makeidx}
\makeindex

% extra char sets
\usepackage{wasysym} % smileys

% set lists style
% \usepackage{enumitem}
% \setlist{nosep}

% misc

% \usepackage{titling}

\newcommand{\email}[1]{$<$\href{mailto:#1}{#1}$>$}
\newcommand{\internet}{Internet}

\newcommand{\cm}[1]{Cortex-M#1}
\newcommand{\cmx}{\cm{x}}

\newcommand{\linux}{Linux}
\newcommand{\emlinux}{em\linux}

\newcommand{\cpp}{$C^{+}_{+}$}
\newcommand{\py}{Python}

\newcommand{\vcs}{\hyperref[vcs]{VCS}}
\newcommand{\make}{\hyperref[make]{Make}}
\newcommand{\spice}{ngSPICE}
\newcommand{\latex}{\LaTeX}

\newcommand{\eclipse}{\textcircled{$\equiv$}\textsc{eclipse}}
\newcommand{\vim}{(g)Vim}

\newcommand{\note}[1]{\footnote{\ #1}}
\newcommand{\cp}[1]{\note{копипаста \url{#1}}}

\newcommand{\win}{\winstart Windows}

\newcommand{\mk}{МК}

\newcommand{\ram}{RAM}


\newcommand{\pref}[1]{/стр.\pageref{#1}/}

% selecting
\usepackage{framed}
\newcommand{\term}[1]{\textcolor{Green}{#1}}
\renewcommand{\emph}[1]{\textcolor{Blue}{#1}}
\newcommand{\prog}[1]{\textcolor{Brown}{#1}}
\newcommand{\pack}[1]{\textcolor{Magenta}{#1}}

% math
\usepackage{cancel}

% titles

\hypersetup{
	pdftitle={Азбука халтурщика-ARMатурщика},
	pdfauthor={ruOpenWrt, HackSpace <<Чебураторный завод>>, Консорциум хоббитов
	России, Bill Collis (Часть 1)}, 
	pdfsubject={https://github.com/ponyatov/Azbuka}
}


\author{Вася Пупкин}
\title{Пример статьи с цитатами}

\begin{document}
\maketitle
	
В статье используются книги: \cite{A} и \cite{B}
	
\begin{thebibliography}{99}

\bibitem{A} Книга А

\bibitem{B} Книга B
	
\end{thebibliography}
\end{document}
\end{verbatim}

Но если вы регулярно работаете с документацией, или часто пишете статьи,
возникает естественное желание вынести весь список литературы в отдельную базу
данных, прописать авторов, названия, издательства и т.п. Это делается с помощью
программы \file{biber}\ и пакета \file{biblatex}.

Пример использования этой системы вы легко найдете в исходниках этой книги:

\begin{itemize}
  \item файл \file{header.tex} содержит секцию подключения пакета и подгрузки
  библиофайлов:
  \begin{verbatim}% books bib management
\usepackage{biblatex}
\addbibresource{../bib/python.bib}
\addbibresource{../bib/eskd.bib}
...\end{verbatim}
  \item библиофайлы хранятся в \textbf{соседнем}\ репозитории \file{../bib},
  склонированном с \url{https://github.com/ponyatov/bib}.
  \item порядок вызова \file{pdflatex}\ и \file{biber}\ см. \file{Makefile}
\end{itemize}

Для оформления библиографии в нужном стиле см. примеры \cite{bibiso}.

\section{Команды секционирования: часть, глава, раздел,..}
\section{Таблицы}
\section{Формулы}
\section{Перекрестные ссылки и гипессылки}
\section{Листинги скриптов и текстовых данных}
\section{Подготовка иллюстраций}

Подготовка иллюстраций\ --- одна из самых геморных тех в создании документации,
и ее верстке для бумажных и электронных изданий.

Предпочтение нужно отдавать векторным форматам, за исключеним фотоиллюстраций.
В идеале скриншоты также хорошо бы переводить в векторыный формат, но пока
инструмент для этого не найден, поэтому выходные файлы будут пухнуть в объеме.

Для подготовки векторных иллюстраций: схем, графиков, диаграмм и т.п.
используйте пакеты, принимающие на вход программы на специализированном языке
программрования, легко читаемым человеком. В этом случае у вас сохраниться
отслеживать изменения, читая логи \vcs.

Обратите внимание на возможность использования стилевых файлов на весь проект
(для всех иллюстраций в книге например). Их использование даст профессиональный
вид продукту, при этом сохраниться возможность взять и переформатировать 100500
схем в 10-томнике, поменяв шрифт, цвета, толщины линий, зазоры между элементами
и т.п. 

Пользуйтесь только относительными единицами размеров, и привязывайтесь к
размерам шрифтов, это даст возможность использовать готовую иллюстрацию в
нескольких проектах с разными размерами бумаги и наборами используемых шрифтов.

\subsection{Графики GNUPLOT}

Самый постой способ получит график простой аналитической функции или
экспериментальных данных\ --- воcпользоваться утилитой
\href{http://gnuplot.info/}{GNUPLOT}.

\bigskip
Оценить возможности можно вот по этому
\href{http://upload.wikimedia.org/wikipedia/commons/b/b2/Gnuplot.ogv}{видео}

\bigskip
Примеры
\href{http://commons.wikimedia.org/wiki/Category:Gnuplot\_diagrams}{на
википедии}

\bigskip
Примеры выложенные вместе с текстом
\href{http://commons.wikimedia.org/wiki/Category:Images\_with\_Gnuplot\_source\_code}{на
языке gnuplotа}

\subsection{Схемы и графы в GraphViz}

Для отрисовки графов и схем, легко к ним сводящихся, можно использовать пакет
\file{GraphViz}\ и язык \file{Dot}. 

\subsection{PGF/TikZ}

Сложные графики можно рисовать с помощью пакета \file{PGF/TikZ}, но для его
работы нужна установленная \latex-система. Этот пакет предназначен прежде всего
для набора и верстки изданий с множеством сложных схем.

\subsection{GLE}

GLE\ --- универсальный язык описания векторных графических объектов с
элементами языка программирования. Поддерживает вычисления, типовые конструкции
программирования (циклы, условия, рекурсию).

\begin{itemize}
  \item\href{http://glx.sourceforge.net/examples/2dplots/index.html}{графики}
  \item\href{http://glx.sourceforge.net/examples/3dplots/index.html}{3D графики}
  \item\href{http://glx.sourceforge.net/examples/diagrams/index.html}{диаграммы}
  \item\href{http://glx.sourceforge.net/examples/fractals/index.html}{фракталы}
  \item\href{http://glx.sourceforge.net/examples/electronic/index.html}{электронные
  схемы}
  \item\href{http://glx.sourceforge.net/examples/other/index.html}{исчо}
\end{itemize}


\section{Верстка электронных изданий}

Для электронных изданий, предназначенных для чтения с различных экранов как
компьютера, так и портативных устройств, сущестует ряд ограничений и
рекомендаций, из-за особенностей экранов: малый размер, низкое разрешение,
поддержка цвета (TFT vs e-Ink) и т.п.: \cite{ebooktex}

\bigskip
Установка полей в .PDF:

\nopagebreak
\begin{verbatim}
\hypersetup{
	pdftitle={Азбука халтурщика-ARMатурщика},
	pdfauthor={ruOpenWrt, HackSpace Чебураторный завод, Bill Collis (part 1)}
}
\end{verbatim}
