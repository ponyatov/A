\secrel{Рабочая среда разработчика встраиваемых систем}\secdown\secdown

\secrel{Операционная система с набором типовых утилит}

  среда для запуска рабочих программ, просмотра электронной документации, поиска
  информации в \internet, запуска ПО поставляемого с измерительной аппаратурой
  (цифровые осциллографы, генераторы сигналов, логические и сигнальные
  анализаторы)

\secrel{САПР электронных устройств (EDA CAD)}

  используется для разработки схем, моделирования работы устройства,
  разводки печатных плат (ПП) и межплатных соединителей, и подготовки
  технологических файлов для производства ПП

\secrel{Пакет расчета и симуляции электронных схем: SPICE}
  
  выполняется симуляция работы схем, расчет рабочих режимов, подбираются
  номиналы элементов, и моделируется работа аналоговой части устройств
   
\secrel{САПР общего назначения}

  создаются модели и чертежи конструкции устройств, прорабатывается
  компоновка, и проверяется работа электро-механических узлов

\secrel{Система управления версиями: \termdef{VCS}{VCS}}

  VCS предназначены для хранения полной истории изменений файлов проекта, и
  позволяют получить выгрузку проекта на любой момент времени, вести несколько
  веток разработки, получить историю изменений конкретного файла, или сравнить
  две версии файла (\term{diff})

\secrel{Текстовый редактор или интегрированная среда разработки (IDE)}

  редактирование текстов программ и скриптов сборки (компиляции) с цветовой
  подсветкой синтаксиса (в зависимости от языка файла), \term{автодополнением}\
  и вызовом программ-утилит нажатием сочетаний клавиш. Также включает различные
  вспомогательные функции, например отладочный интерфейс и отображение объектов
  программ.

\secrel{Пакет кросс-компиляции GNU toolchain}

  Пакет кросс-компилятора, ассемблера, линкера и других утилит типа make,
  objdump,.. для получения прошивок из исходных текстов программ.

\secdown
\secrel{Утилита Make}
\secrel{binutils}
\secrel{Ассемблер GNU AS}
\secrel{Линкер LD}
\secrel{Утилиты работа с файлами формата ELF}
\secrel{Компилятор GCC}
\secup

\secrel{ПО для программатора, JTAG-адаптера}

  загрузка полученной прошивки в целевое устройство, редактирование памяти,
  внутрисхемная отладка в процессе работы устройства, прямое измение сигналов на
  выводах процессора (граничное сканирование и тестирование железа).

\secrel{Симулятор для отладки программ без железа}

  может использоваться как ограниченная замена реального железа
  для начального обучения, и для отладки программ, не завязанных на работу
  железа.

\secrel{Система верстки документации}

  Для документирования проектов и написания руководств нужна система верстки
  документации, выполняющая трансляцию текстов программ и файлов
  документации в выходной формат, чаще всего \file{.pdf} и \file{.html}.
   
\secup\secup
