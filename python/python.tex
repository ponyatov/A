\secrel{Вспомогательные скрипты на языке \py}

% \begin{tabular}{p{0.1\textwidth} p{0.8\textwidth}}
% \includegraphics[width=0.1\textwidth]{python/logo.png}&
% \emph{
% Название языка произошло вовсе не от вида пресмыкающихся. Автор назвал язык в
% честь популярного британского комедийного телешоу 1970-х «Летающий цирк Монти
% Пайтона». Впрочем, всё равно название языка чаще ассоциируют именно со змеёй,
% нежели с передачей\ --- пиктограммы файлов в KDE или в Microsoft Windows и даже
% эмблема на сайте \url{http://www.python.org} (до выхода версии 2.5) изображают
% змеиные головы.
% }
% \\
% \end{tabular}
% \bigskip
% 
% \py\note{в оригинале читается \textbf{п\'{а}йтон}, но давно русифицировался как
% \textbf{пит\'{о}н}}\ --- высокоуровневый язык программирования общего
% назначения, ориентированный на повышение производительности разработчика и
% читаемости кода.
% 
% \py\ удобно применять для написания различных вспомогательных скриптов.
% Часто его используют при разработке сложных программных систем для написания
% первых версий. В процессе работы над большими программами часто перерабатываются
% большие объемы кода, поэтому для ускорения разработки требуется максимально
% высокоуровневый язык. После того как архитектура программы стабилизируется,
% узким местом становится производительность, и программу переписывают на более
% низкоуровневом компилируемом языке, чаще всего \cpp.
% 
% Написание программ упрощают:
% 
% \begin{itemize}
%   \item \textbf{объектно-ориентированное программирование} облегчает разработку
%   программ, позволяет переопределить стандартные операторы для пользовательских
%   типов данных, упрощая синтаксис
%   \item \textbf{динамическая типизация} не требуется заранее упределять
%   переменные, они создаются простым присваиванием
%   \item \textbf{обработка исключений} для секции кода можно определить
%   обработчик ошибок
%   \item \textbf{высокоуровневые структуры данных}\ --- списки, словари (набор
%   элементов ключ:значение), очереди
%   \item богатая стандартная библиотека и множество дополнительных библиотек на
%   все случаи
% \end{itemize}
% 
% \section{Установка под \win}\label{pywinstall}
% 
% \bigskip
% \menu{
% \winr
% >
% \url{http://www.python.org}
% >
% Downloads
% >
% \href{https://www.python.org/ftp/python/2.7.8/python-2.7.8.msi}{Python 2.7.8}
% }
% 
% \bigskip
% \menu{\file{python-2.7.8.msi}
% >
% Setup>
% for all users/for me
% }
% 
% \menu{Destination Directory > \file{C:/Python/} > Next}
% 
% \nopagebreak
% \bigskip
% \includegraphics[width=0.45\textwidth]{python/install/037.png}
% \includegraphics[width=0.45\textwidth]{python/install/038.png}
% \bigskip
% 
% \menu{Customize > Python > Add python.exe to PATH > Next > Finish}
% 
% \nopagebreak
% \bigskip
% \includegraphics[width=0.45\textwidth]{python/install/039.png}
% \includegraphics[width=0.45\textwidth]{python/install/045.png}
% \bigskip
% 
% \section{Запуск}
% 
% Из командной строки: \menu{\winr cmd > python}
% 
% \bigskip
% \includegraphics[width=0.9\textwidth]{python/install/046.png}
% \bigskip
% 
% Простейшая среда IDLE\note{на GUI-библиотеке Tkinter, идущей в комплекте}:
% \bigskip
% 
% \menu{\winstart > Программы > Python 2.7 > IDLE (Python GUI)}
% 
% \menu{Панель задач > IDLE > \rms > Закрепить в панели задач > \lms }
% 
% \bigskip
% \includegraphics[width=0.9\textwidth]{python/install/047.png}
% \bigskip
% 
% \dblms\ по файлу скрипта:
% \bigskip
% 
% \menu{\winr > notepad \file{/tmp/py.py}}
% 
% \lst{/tmp/py.py}{}{python/install/py.py}
% 
% \menu{\winr > /tmp/py.py}
% 
% \bigskip
% \includegraphics[width=0.9\textwidth]{python/install/048.png}
% \bigskip
% 
% Открытием файла скрипта в IDLE:
% \bigskip
% 
% \menu{\winr > \file{/tmp/}}
% 
% \menu{\file{py.py} > \rms > Edit with IDLE}
% 
% \menu{меню > Run > Run Module \keys{F5}}
% 
% \bigskip
% \includegraphics[width=0.9\textwidth]{python/install/049.png}
% 
% \includegraphics[width=0.9\textwidth]{python/install/050.png}
% \bigskip
% 
% \section{Дополнительные материалы}
% 
% \cite{pyotkidach} Г. Россум, Ф.Л.Дж. Дрейк, Д.С. Откидач,
% \href{http://rus-linux.net/MyLDP/BOOKS/python.pdf}{Язык программирования Python}
% 
% \cite{pythink} Аллен Дауни
% \href{https://drive.google.com/file/d/0B0u4WeMjO894Q2hWV1QwOFFQOVk/view?usp=sharing}{Думать
% на языке \py: Думать как компьютерный специалист}

