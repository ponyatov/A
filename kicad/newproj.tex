
\bigskip
Лучше всего для каждого проекта использовать раздельные папки; в противном
случае система может сбиться с толку, если файлы из разных проектов будут лежать
в одной папке. Проделайте следующие шаги:

\begin{enumerate}
  \item Запустите программу KiCad \icon{kicad/icon_kicad.png} 
  \item Создайте новый проект
\icon{kicad/new_project.png}
  \begin{itemize}
    \item На панели инструментов KiCad выберите левую иконку с подсказкой
\menu{Начать новый проект}, используйте команду меню
\menu{Файл>Новый>Пустой} или сочетание клавиш \keys{Ctrl+N}.
  \item Создайте папку проекта \file{DarkSensor}
    \item
В диалоге \menu{Создать новый проект} выберите созданную папку
выберите только что созданную папку \file{DarkSensor} и
введите имя проекта \menu{\file{DarkSensor}} и нажмите \menu{Сохранить}.
  \end{itemize}
	\item
Если папка проекта содержит какие-то файлы, будет выведено окно выбора:
создать подпапку с именем проекта \menu{Yes}, или записать файл проекта
в указанную папку как есть \menu{No}. Нажмите No.
    \item
Сохраните проект кнопкой \menu{Сохранить текущий
проект}, \menu{Файл>Сохранить} или \keys{Ctrl+S}.
	\item
В папке появится файл \file{SpindleDriver.pro}, содержащий установки вашего
проекта. Файл имеет тектовый формат, поэтому при необходимости его можно открыть
в любом редакторе и вручную аккуратно подправить, например скорректировать
настройки зазоров печатной платы.
\end{enumerate}
