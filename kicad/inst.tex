\section{Установка под \win}\label{kicadinstwin}

\menu{\winr{\url{http://www.kicad-pcb.org/}}>Download>\winstart}

\menu{\winr{\url{http://kicad.nosoftware.cz/}}>
\file{KiCad\_testing\-201x.xx.xx-BZRxxxx\_Win\_full\_version.exe}}

\bigskip

\menu{Installer Language>\emph{English}>Ok} 
в русифицированном инсталляторе кривые шрифты

\menu{KiCAD 20xx.xx.xx Setup>Next}

\menu{Лицензия>Agree}

\menu{Components>\checkbox\ все>Next}

\menu{Location>\file{C:/KiCad}>Install}

\menu{Completing Setup>\uncheckbox Wings3D>Finish}

\section{Установка под \linux}\label{kicadinstlin}

\begin{verbatim}
root# aptitude install kicad-doc-ru kicad
\end{verbatim}

\lst{+++\ $\sim$/.blackbox/menu}{}{tmp/kicad.bbmenu}

Для добавления библиотек, поставляемых с этой книгой, сделайте \emph{git
checkout}:

\begin{verbatim}
user:~$ git clone --depth=1 -o gh https://github.com/ponyatov/Azbuka.git Azbuka
\end{verbatim}

\menu{\prog{kicad}>\prog{eeschema}>Настройки>Библиотека}

\menu{Пользовательские пути поиска>Добавить>\file{/home/user/Azbuka/kicad/lib}}

\menu{Файлы библиотек>Добавить>R,L,C,SPICE,DA\_POWER,..>Открыть>OK}

\bigskip
Для проверки работы библиотек можете открыть проект

\menu{\prog{kicad}>Файл>Открыть>\file{~/Azbuka/bcollis/led1/led1.pro}}

или сразу схему

\menu{\prog{eeschema}>Файл>Открыть>\file{~/Azbuka/bcollis/led1/led1.sch}}
