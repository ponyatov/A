\secrel{Установка}\secdown

\secrel{\win}\label{kicadinstwin}

\menu{\winr{\url{http://www.kicad-pcb.org/}}>Download>\winstart}

\menu{\winr{\url{http://kicad.nosoftware.cz/}}>
\file{KiCad\_testing\-201x.xx.xx-BZRxxxx\_Win\_full\_version.exe}}

\bigskip

\menu{Installer Language>\emph{English}>Ok} 
в русифицированном инсталляторе кривые шрифты

\menu{KiCAD 20xx.xx.xx Setup>Next}

\menu{Лицензия>Agree}

\menu{Components>\checkbox\ все>Next}

\menu{Location>\file{C:/KiCad}>Install}

\menu{Completing Setup>\uncheckbox Wings3D>Finish}

\secrel{\linux}\label{kicadinstlin}

\begin{verbatim}
root# aptitude install kicad-doc-ru kicad
\end{verbatim}

\lst{+++\ $\sim$/.blackbox/menu}{}{/tmp/kicad.bbmenu}

\secrel{Настройка библиотек}

Для добавления библиотек, поставляемых с этой книгой, сделайте \emph{git
clone} или \emph{git pull}:

\begin{verbatim}
user:~$ git clone --depth=1 -o gh https://github.com/ponyatov/Azbuka.git Azbuka
user:~$ cd Azbuka
user:~/Azbuka$ git pull
\end{verbatim}

\menu{\prog{kicad}>\prog{eeschema}>Настройки>Библиотека}

\menu{Пользовательские пути поиска>Добавить>\file{/home/user/Azbuka/kicad/lib}}

\menu{Файлы библиотек>все стандартные>Удалить>OK}

\menu{Файлы библиотек>Добавить>R,L,C,SPICE,DA\_POWER,..>Открыть>OK}

\bigskip
Для проверки работы библиотек можете открыть проект

\menu{\prog{kicad}>Файл>Открыть>\file{~/Azbuka/bcollis/led1/led1.pro}}

или сразу схему

\menu{\prog{eeschema}>Файл>Открыть>\file{~/Azbuka/bcollis/led1/led1.sch}}

\secrel{Дотфайлы}

Посколько программа изначально писалась как юниксовая, пользовательские
настройки хранятся в dot-файлах:

\lst{~/.kicad}{}{kicad/kicad.dotfile}

\begin{description}
\item[KicadFrame*] размеры и положение окна менеджера проектов
\item[WorkingDir] каталог с текущим рабочим проектом
\item[fileN] список последних проектов (\ \menu{Файл>Последние файлы}\ )
\end{description}

\lst{~/.eeschema}{}{kicad/eeschema.dotfile}

\begin{description}
\item[SchematicFrame*] размеры и положение окна \prog{eeschema}
\item[fileN] список последних схем (\ \menu{Файл>Последние файлы}\ )
\end{description}

\secrel{Глобальные шаблоны}

После установки \kicad\ можно скорректировать файлы глобальных шаблонов, чтобы
новые проекты создавались сразу с нужными настройками, прежде всего с нужным
нам набором библиотек:

\begin{verbatim}
sudo vim /usr/share/kicad/template/kicad.pro
\end{verbatim}

\lst{/usr/share/kicad/template/kicad.pro}{}{kicad/kicad.template}

\begin{description}
\item[LibDir] каталог библиотек, установите на свой или корпоративный/групповой
\item[LibNameN] задается список библиотек по умолчанию, приопишите ваш типовой
набор
\item[eeschema/libraries] схемные библиотеки \eeschema
\item[pcbnew/libraries] библиотеки надстеков для печатных плат \pcbnew
\end{description}

\secup
