
\secrel{Моделирование схем в ngSPICE}\secdown

\cp{http://www.mithatkonar.com/wiki/doku.php/kicad/kicad\_spice\_quick\_guide}

\cp{http://physics.gmu.edu/~rubinp/courses/407/ngspice.pdf}

\prog{SPICE}: [S]imulation [P]rogram with [I]ntegrated [C]ircuit
[E]mphasis\ --- пакет программ симуляции и расчета электронных схем.
Это мощная программа, используемая в разработке устройств и \emph{интегральных
схем}, для расчета режимов работы, оптимизаии, и предсказания поведения схем.

\bigskip
SPICE может выполнять несколько видо схемотехнических расчетов, самые важные
из которых:

\begin{itemize}
  \item
Нелинейный анализ по постоянному току: вычисление передаточной характеристики по
постоянному току
  \item 
Нелинейный анализ переходных процессов: вычисление токов и напряжений как
функции времени в условиях большого сигнала
  \item 
Линейный AC анализ: вычисление выхода как функции от частоты. Выводится
\term{bode plot}
  \item 
Анализ шума 
  \item
Расчет чувствительности
  \item 
Анализ искажений 
  \item
Фурье-анализ: вычисление и отображение частотных спектров   
  \item 
Анализ Монте-Карло
\end{itemize}  

\secrel{Доступные SPICE-пакеты}

\begin{itemize}
\item \href{http://ngspice.sourceforge.net/}{ngspice}: самый доступный
бесплатный OpenSource SPICE-движок.
\item
\href{https://www.gnu.org/software/gnucap/}{gnucap}\note{уже включен в
виндозную сборку KiCAD}:
Не совсем SPICE, но пытается быть синтаксически совместмой.
\item \href{http://www.spiceopus.si/}{SpiceOpus}: Коммерческая, но
удобная, особенно в плане вывода графиков.
\item \href{http://www.linear.com/designtools/software/}{LTSpice}:
Популярная бесплатная коммерческая версия от Linear Technology для \win.
Работает в \prog{WINE}. Поставляется в комплекте с графической оболочкой,
но требуется проверка файлов расчетных заданий, которые она создает. 
\item
\href{http://www.cadence.com/products/orcad/pspice_simulation/Pages/default.aspx}{PSpice}:
\win-only, дорогой коммерческий пакет, стандарт de-facto для
профессионального применения в USA в составе тяжелых EDA-продуктов.
Они традиционно предоставляют обрезанную \emph{gratis}-версию для студентов.
Также имеет в комплекте GUI.
\item \url{http://www.ngspice.com/}\ --- on-line вариант \prog{ngspice}, удобен
для начального обучения
\item Какие-то еще?
\end{itemize}

\secrel{Установка \prog{ngspice} под \win}

\menu{\winr\url{http://ngspice.sourceforge.net/download.html}>\prog{ng-spice-rework}>\file{ngspice-xx\_xxxxxx.zip}}

\menu{unzip \file{ngspice-xx\_xxxxxx.zip}>\file{C:/spice}}

\menu{\winstart>Компьютер>\rms>Свойства>Дополнительные
параметры системы}

\menu{Переменные среды>\file{PATH=\ldots;C:/spice/bin}}

\secrel{Установка \prog{ngspice} под \linux}

\begin{verbatim}
root# aptitude install ngspice
\end{verbatim}

\secrel{Библиотеки компонентов со SPICE-элементами для \prog{KiCAD}}

\begin{itemize}
\item Библиотека базовых SPICE-компонентов поствляется с KiCAD. Эта
библиотека\ --- хороший вариант для начальных экспериментов.
Библиотека не подключена по умолчанию, вы должны сделать это вручную сами через
пенеджер библиотек. На Debian Linux это файл\\
\file{/usr/share/kicad/library/pspice.lib}\note{PSpice\ --- популярная
коммерческая версия SPICE}

\item Mithat Konar \email{webs@mithatkonar.com} разрабатывает (очень медленно)
\href{https://bitbucket.org/mithat/kicad-spice-library}{собственую
библиотеку} с некоторыми модификациями.

\item В комплекте с этой книгой поставляются библиотеки, адаптированные для
SPICE.

\end{itemize}

\secrel{Настройка проекта}

\begin{enumerate}
\item Создайте новый проект как обычно.
\item Откройте \prog{Eeschema} и удалите все библиотеки, подключаемые по
умолчанию.
\item Вручную добавьте одную из SPICE-библиотек, или набор библиотек для
из этой книги. ОБратите внимание, что SPICE-библитека из поставки \prog{KiCAD}
по умолчанию не подключается к пректу.
\item Укажите расчетный SPICE-движок, который вы хотите использовать:

\menu{\prog{eeschema}>Меню>Инструменты>Сформировать список цепей}

\menu{вкладка Spice>Команда симулятора:>\prog{ngspice}}

\end{enumerate}

\secrel{Как это работает}

\begin{enumerate}
  \item
Рисуйте вашу схему, соблюдая несколько рекомендаций:  
\begin{itemize}
  \item
Для именованных цепей используйте \emph{глобальные метки} вместо локальных. 
В списке цепей глобальные идентификаторы цепей включаются как есть, а
локальные метки модифицируются, что делает сложным последующие ссылки на них при
SPICE-моделировании.
  \item
Используйте компонент [0] из библиотеки \file{spice}, вместо обычного комопнента
[GND]: ”0” официальное имя главной земли в файлах PSpice. Некоторые SPICE-движки
умеют транслировать $GND \rightarrow 0$, другие нет.
\end{itemize}
  \item
Укажите режимы сиуляции, которые вы хотите выполнить, и генерацию вывода,
который хотите отобразить, добавив на схему текстовый блок (т.е.
``комментарий'') c необходимыми директивами в синтаксисе
\href{http://newton.ex.ac.uk/teaching/cdhw/Electronics2/userguide/sec5.html}{SPICE и Nutmeg}
с некоторыми добавками. Например, для выполнения \term{расчета по постоянному
току} и вывода сигнала в точке \verb|vout|, добавьте блок:
\begin{lstlisting}
+PSPICE
.control
ac dec 66 1kHz 120kHz
plot vdb(vout)
set units = degrees
plot vp(vout)
.endc
\end{lstlisting}
\begin{itemize}
  \item
Первая строка ``+PSPICE '' указывает kicadу добавить текст \emph{в конец}
сформированного \file{.cir}-файла. \emph{В текущей версии KiCAD есть баг,
который требует обязательного пробела после +SPICE}.
  \item
Соответственно строка ``-PSPICE " добавляет текст \emph{в начало} .cir-файла.
  \item
Для поборников OpenSource, не желающих видеть ссылка на коммерческий PSpice,
предусмотрены директивы-синонимы $\pm$``GNUCAP ''. Я думаю это то же самое что и
$\pm$``PSPICE ``, но не уверен на 100\%, проверьте в документации.
  \item
Да, вам потребуется немного изучить синтакис 
\href{http://newton.ex.ac.uk/teaching/cdhw/Electronics2/userguide/sec5.html}{SPICE
and Nutmeg}. Это нетрудно.
\end{itemize}
  \item
Запустите симуляцию:
\begin{itemize}
  \item
Кликните кнопку или пункт меню \menu{Сформирвать список цепей}.
  \item
Выберите вкладку \menu{Spice}, и убедитесь что включен крыжик \menu{\checkbox\
Формат по умолчанию}. Вам нужно сделать это только один раз, настройки
запоминаются.
  \item
Заполните полный путь с программе симуляции, типа
\file{C:/spice/bin/ngspice.exe} со всеми путями и расширениями, KiCAD пока не
научился запускать симулятор через \verb|PATH|.
  \item
Нажмите кнопку \menu{Запустить симулятор}.
\end{itemize}
\end{enumerate}

\secup
