\chapter{Лексический и синтаксический анализ}

Очень часто в практике возникает необходимость работы с данными в текстовых
форматах\ --- plain text файлы, в которых в каком-либо формате\note{на языке
разметки, или DDL [D]ata [D]efinition [L]anguage}\ описаны данные. И от вас
требуется реализовать разбор такого файла, выделяя элементы данных, чтобы в
дальнейшем после их преобразования например записать текстовый файл в другом
формате.

В таком виде хранятся результаты расчетных программ, работащих в пакетном
режиме, данные с измерительных систем, поток данных с приемников
GPS\note{протокол NMEA 0183}, очень популярный мета-формат XML со всеми его
частными случаями типа HTML, XLIFF\ref{xliff}, OpenDocument, тексты программ для
станков с ЧПУ,\ldots

С некоторыми хинтами точно так же можно работать и с бинарными файлами,
преобразовав их сначала в текстовую форму (в простейшем случае просто сделав hex
dump).

В некоторых случаях необходимо написание трансляторов форматов (текстовых)
данных, или даже интерпретаторов/компиляторов языков программирования.

\bigskip
Все эти техники с использованием стандартных утилит \prog{flex}\ и
\prog{bison}\ будут кратко описаны в этой главе.
Подробнее эти техники рассмотрены в книгах\ref{lexlit}, особенно стоит отметить
талмуд \textbf{DragonBook}\ref{exdragon}.

\section{Лексер и лексический анализ, утилита \prog{flex}}

\begin{framed}
\term{Лексер}\ --- программа или ее часть, которая 
\begin{enumerate}
  \item 
получает на вход исходные
данные в виде сплошного потока одиночных сиволов, 
  \item 
группирует символы согласно
набору правил, и 
  \item 
отдает на выходе символы, уже сгруппированные в \term{лекс\'{е}мы}\ или
\term{ток\'{е}ны}.
\end{enumerate}
Цель лексера\ --- подготовить последовательность лексем для входа другой
программы.
\end{framed}

\term{Лексический анализ}\ --- процесс программного разбора входной
последовательности символов\note{например, такой как исходный код на одном из
языков программирования} с целью получения на выходе последовательности групп
символов\ --- \term{токенов}, имеющих собственное смысловое
значение\note{подобно группировке букв в слово}. Как правило, лексический анализ
производится в соответствии набора правил определённого \term{формального,
искуственного или компьютерного языка}.

\begin{framed}
\term{Язык}, а точнее его
\term{грамматика}, задаёт определённый набор лексем, которые могут встретиться
на входе лексера, и набор правил, по которым их следует группировать.
\end{framed}

Традиционно принято организовывать процесс лексического анализа, рассматривая
входную последовательность символов как поток одиночных символов. При такой
организации \term{лексер}\ самостоятельно управляет выборкой отдельных символов
из входного потока.

Распознавание лексем с учетом грамматики обычно производится путём их
идентификации согласно идентификаторам токенов, определяемых грамматикой языка.
При этом любая последовательность символов входного потока (лексема), которая
согласно грамматике не может быть идентифицирована как токен языка, обычно
рассматривается как специальный \term{токен-ошибка}.

\bigskip
Каждый \term{выделенный} токен можно представить в виде парной структуры,
содержащей
\begin{enumerate}
  \item идентификатор токена и
  \item саму последовательность символов лексемы, выделенной из входного
потока\note{запись строки, числа и т. д.}.
\end{enumerate}

\bigskip
Рассморим обработку текстового файла: описания простой графической фигуры в
текством формате \href{https://ru.wikipedia.org/wiki/IGES}{IGES}: универсальном
формате обмена данными для САПР. IGES-файл состоит из 80-символьных
ASCII-записей\note{длина записи произошла из эры перфокарт, типовая ширина
вывода терминала или принтера}. Текстовые строки представлены в
<<Холлерит>>-формате\ -- число символов в строке, плюс латинская буква [Н] и
сама строка, например\ --- <<4HSLOT>>. Рассмотрим  очень короткий IGES-файл 1987
года, включающий в себя лишь сущности пары точек (POINT, тип 116), пары
полуокружностей (CIRCULAR ARC, тип 100) и двух линий (LINE, тип 110). Можно его
считать эскизом паза под шпонку на валу, или шаблоном некруглого монтажного
отверстия.

\lst{sample.iges}{https://github.com/ponyatov/Azbuka/raw/master/prog/lexer/sample.iges}{prog/lexer/sample.iges}

\section{Компилятор Паскаля}

\section{Дополнительная литература}\label{lexlit}

\label{exdragon}\cite{dragonbook} \textbf{Книга Дракона}: Ахо, Сети, Ульман
Принципы построения компиляторов.

\bigskip

\href{http://habrahabr.ru/post/99162/}{Habr: Компиляция. 1: лексер}

\href{http://habrahabr.ru/post/99298/}{Habr: Компиляция. 2: грамматики}

\href{http://habrahabr.ru/post/99366/}{Habr: Компиляция. 3: бизон}

\href{http://habrahabr.ru/post/99397/}{Habr: Компиляция. 4: игрушечный ЯП}

\href{http://habrahabr.ru/post/99466/}{Habr: Компиляция. 5: нисходящий разбор}

\href{http://habrahabr.ru/post/102597/}{Habr: Компиляция. $5\frac{1}{2}$: llvm
как back-end}

\href{http://habrahabr.ru/post/99592/}{Habr: Компиляция. 6: промежуточный код}

