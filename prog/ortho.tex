\section{Ортогональность}

\cp{http://ru.wikibooks.org/wiki/\%D0\%9E\%D1\%80\%D1\%82\%D0\%BE\%D0\%B3\%D0\%BE\%D0\%BD\%D0\%B0\%D0\%BB\%D1\%8C\%D0\%BD\%D0\%BE\%D1\%81\%D1\%82\%D1\%8C}

\term{Ортогональность}\ очень важна, если вы хотите создавать системы, которые
легко поддаются проектированию, сборке, тестированию и расширению. Однако этому
принципу редко обучают непосредственно. Часто он является лишь скрытым
достоинством других разнообразных методик, которые вы изучаете. Это неправильно.
Как только вы научитесь непосредственно применять принципы ортогональности, вы
сразу заметите, как улучшилось качество создаваемых вами систем.

\paragraph{Что такое ортогональность?}

Термин "ортогональность"\ заимствован из геометрии. Две линии являются
ортогональными, если они пересекаются под прямым углом, например, оси координат
на графике. В терминах векторной алгебры две \emph{такие линии перемещения
являются независимыми}. Если двигаться параллельно оси X вдоль одной из линий,
то проекция движущейся точки на другую линию не меняется. Этот термин был введен
в информатике для обозначения некой разновидности независимости или
несвязанности. В грамотно спроектированной системе программа базы данных будет
ортогональной к интерфейсу пользователя: вы можете менять интерфейс пользователя
без воздействия на базу данных и менять местами базы данных, не меняя
интерфейса. Перед тем как рассмотреть преимущества ортогональных систем,
познакомимся с неортогональной системой.

\paragraph{Неортогональная система}

Предположим, вы находитесь в экскурсионном вертолете, совершающем полет над
Гранд-Каньоном, когда пилот, который совершил ошибку, наевшись рыбы за обедом
внезапно вскрикивает и теряет сознание. По счастливой случайности это
происходит, когда вы парите на высоте 30 метров. Вы догадываетесь, что рычаг
управления общим шагом несущего винта обеспечивает подъем машины, так что, если
его слегка опустить, вертолет начнет плавно снижаться. Однако когда вы пытаетесь
сделать это, то осознаете, что жизнь\ --- не такая уж простая штука. Вертолет
клюет носом, и вас начинает вращать по спирали влево. Внезапно вы понимаете, что
управляете системой, в которой каждое воздействие имеет побочные эффекты. При
нажатии на левый рычаг вам придется сделать уравновешивающее движение назад
правым рычагом и нажать на правую педаль. Но при этом каждое из этих действий
вновь повлияет на все органы управления. Неожиданно вам приходится жонглировать
невероятно сложной системой, в которой любое изменение влияет на все остальные
управляющие воздействия. Вы испытываете феноменальную нагрузку: ваши руки и ноги
находятся в постоянном движении, пытаясь уравновесить все взаимодействующие
силы. Органы управления вертолетом определенно не являются ортогональными.

\paragraph{Преимущества ортогональности}

Как показывает пример с вертолетом, неортогональные системы сложнее изменять и
контролировать. Если составляющие системы отличаются высокой степенью
взаимозависимости, то невозможно устранить какую-либо неисправность лишь на
локальном уровне.

\bigskip
\emph{Исключайте взаимодействие между объектами, не относящимися друг к другу}
\bigskip

Мы хотим спроектировать компоненты, которые являются самодостаточным
независимыми, с единственным, четким назначением. Когда компоненты изолированы
друг от друга, вы уверены, что можно изменить один из них, не заботясь об
остальных. Пока внешние интерфейсы этого компонента остаются неизменными можете
быть спокойны, что не создадите проблем, которые распространятся по
\emph{всей}\ системе. С созданием ортогональных систем у вас появятся два
больших преимущества: увеличение производительности и снижение риска.

\paragraph{Увеличение производительности}

\begin{itemize}
\item Изменения в системе локализуются, поэтому периоды разработки и
тестирования сократятся. Легче написать относительно небольшие, самодостаточные
компоненты, чем один большой программный модуль. Простые компоненты могут быть
спроектированы, запрограммированы, протестированы и затем забыты\ --- не нужно
непрерывно менять существующий текст по мере того, как к нему добавляются новые
фрагменты.

\item Ортогональный подход также способствует многократному использованию
компонентов. Если компоненты имеют определенную, четкую сферу ответственности,
они могут комбинироваться с новыми компонентами способами, которые не
предполагались при их первоначальной реализации. Чем меньше связанность в
системах, тем легче их перенастроить и провести их обратное проектирование.

\item При комбинировании ортогональных компонентов происходит заметное
увеличение производительности. Предположим, что один компонент способен
осуществлять М, а второй\ --- N различных операций. Если эти компоненты
ортогональны и комбинируются, то в сумме они способны осуществить $M\times N$
различных операций. Но если два компонента не являются ортогональными, они будут
перекрываться, и результат их действия будет меньшим по сравнении с
ортогональными компонентами. Вы получаете большее количество функциональных
возможностей в пересчете на единичное усилие, если комбинирует между собой
ортогональные компоненты.
\end{itemize}

\paragraph{Снижение риска}

Ортогональный подход приводит к снижению уровня риска, присущего любой
разработке.

\begin{itemize}
\item Ошибочные фрагменты текста программы изолируются. Если модуль содержит
ошибку, то вероятность ее распространения на всю систему уменьшается. Кроме
того, ошибочный фрагмент может быть извлечен и заменен новым (исправленным).

\item Конечный продукт (система) становится менее хрупким. Проблемы,
появляющиеся при внесении небольших изменений и устранении недочетов на
определенном участке, не проходят дальше этого участка.

\item Ортогональная система способствует повышению качества тестирования,
поскольку облегчается проектирование и тестирование отдельных ее компонентов.

\item Вы не будете слишком сильно привязаны к определенному субподрядчику,
программному продукту или платформе, поскольку интерфейсы между компонентами,
производимыми фирмами-субподрядчиками, не будут играть главенствующей роли в
проекте.
\end{itemize}
