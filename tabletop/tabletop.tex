\secrel{Настольные станки}\label{tabletop}\secdown

На основе \cp{http://rutracker.org/forum/viewtopic.php?t=3126529}\bigskip

\copyright\ \emph{Joe Martin}, illustration by \emph{Craig Libuse}

\emph{Tabletop Machining}

A basic approach to making small parts on miniature machine tools

\bigskip

\copyright\ \emph{Джо Мартин}, иллюстрации \emph{Craig Libuse}

Базовые навыки изготовления мелких деталей на миниатюрных настольных станках

\clearpage

% \secrel{A special note to engineers reading this book}
% 
% \secrel{Machining for engineen and engineering for machinists}
% 
% At first glance the subtitle on the cover of this book
% could be a bit deceiving. What does tabletop
% machining have do with engineering you may ask?
% Compare it to a book that has been written about
% the ocean. The seas could he described from the
% perspective of a young man who has just sailed
% around the world in a twenty-five foot sailboat or
% by a merchant seaman who has spent his career
% aboard a giant ocean liner. Each would have an
% entirely different view of what the ocean was all
% about. In a stenn, the chap in the small boat would
% write ahout surviving broken masts and
% mountainous seas while the merchant seaman might
% write about seasick passengers. I believe you would
% learn more ahout the ocean from the young man in
% the small boat, because in a sense he was more
% involved in his subject. He was not just on it, he
% was in it.
% 
% \secrel{Navigating the seas of machining}
% 
% The ocean in this case is the world of machining.
% The craftsman using tabletop machine tools is like
% the sailor in a small boat, while the professional
% machinist with his big CNC shop tools is like the
% world-traveling seaman. The process of producing
% complex, accurate parts cannot be described by
% looking in the window of a quarter million dollar
% CNC machine. It would be like a merchant seaman
% working in the engine room trying to describe a
% stonn in the Atlantic Ocean by telling you how much
% extra fuel the ship used. The professional's view of
% the subject may be so cluttered with details that it is
% difficult to sort the things you really need to know
% to sail in rough seas or make good parts. I t is the
% craftsman working with small tools, turning the
% cranks by hand, who will have the most to tell you
% about the real world of working with metal.
% 
% \secrel{looking at engineering from the craftsman's perspedive}
% 
% With the aid of computers, parts can easily be drawn
% that can't be built. CAD prvgrams allow a designer
% to put a perfect .0001 " radius on the inside comer of
% a pocket cut in tool steel. Hopefully after reading
% this book you will not ask a toolmaker to do it, but
% if you do, you'll at least know it is going to cost a
% great deal of money to try. Working with metal is
% far more difficult than one would imagine. A false
% impression is gained by looking at the beautiful yet
% inexpensive machined parts that we deal with daily.
% They have been produced in very large quantities,
% and that five-dollar part you may consider a "ripoff'
% could easi ly cost five hundred dollars if you
% had to manufacture just one. New engineers will
% often think a toolmaker is a failure when the
% seemingly simple part they design ends up costing
% a thousand dollars to make. Most engineers wi ll
% eventually have to deal with the craftsman who tum
% their ideas into reality, and in reading this book I
% would hope you come away with a new perspective
% of what is really involved in producing a machined
% part or a product. An alternate subtitle for the book
% might have been "Things they should have taught
% you in engineering school but didn't". This book
% might be considered your textbook for a course
% called "Reality WI".
% 
% \secrel{Seeing produdion from the point of view of both
% the engineer and machinist}
% 
% My perspective on machining could be considered
% unique because, in order to survive, I have had to
% deal with every aspect of product design from
% engineering to prototyping to tooling to
% manufacturing to sales. In this book I have tried to
% pass along the logic I used to solve the associated
% problems. Understanding how a craftsman thinks
% and works is an essential part of getting projects
% done. Unless you are willing to build your designs
% yourself, you are going to have to learn how to deal
% with the craftsman who will actually build them.
% The more you know about their methods.
% personalities and unique problems, the better your
% chances are for success. Smooth sailing.
% 
% \bigskip\copyright\ Joe Martin
% 
% \secrel{About the Joe Martin}
% 
% Joe Martin worked in the construction trades after graduating from high school,
% but his real love was always building and fl ying radio controlled model
% airplanes. When he decided to turn his hobby into a business and start his own
% company making components for the radio control industry, he had to learn about
% machining and toolmaking on his own. He simply couldn afford to hire anyone else
% to set up the tools and make the molds. He has designed and taken to market
% numerous products and owned several companies over the years. He began his
% association with Sherline Products as an importer of Australi an-built lathes in
% the early 1970's. Since then, Joe's company has grown to become the sole
% manufacturer and worldwide distributor of Sherline machine tools.
% 
% Joe was one of the founders of the sport ofFonnula One model aircraft
% competition as well as one of its early champions. His competitive nature seems
% to find its way into whatever form of fun he pursues. He has been a winner in
% sports from model airplane competition to ocean sai lboat racing and, most
% recently, automobile racing.
% 
% Never one to be a spectator in life, he has tried and mastered many skills. In
% this book, he passes on to you some of his hard-won knowledge about machining.
% His down-to-earth s ty le is not hi gh ly polished. In fact, if you could say
% that life has put a finish on him, it would probably be described as "" ground
% or honed .. . ve ry company making components for the radio control industry, he
% had to learn about machining and toolmaking on his own. He simply couldn afford
% to hire anyone else to set up the tools and make the molds. He has designed and
% taken to market numerous products and owned several companies over the years. He
% began his association with Sherline Products as an importer of Australi an-built
% lathes in the early 1970's. Since then, Joe's company has grown to become the
% sole manufacturer and worldwide distributor of Sherline machine tools.
% Joe was one of the founders of the sport ofFonnula One model aircraft
% competition as well as one of its early champions. His competitive nature seems
% to find its way into whatever form of fun he pursues. He has been a winner in
% sports from model airplane competition to ocean sai lboat racing and, most
% recently, automobile racing. Never one to be a spectator in life, he has tried
% and mastered many skills. In this book, he passes on to acc urate but not slick.
% I think his heartfe lt love of good too ls and miniature machining will be
% apparent to all who read this book. Working with him these past 25 years is
% certainly an experience I would not have wanted to miss.
% 
% \bigskip\copyright\ Craig Libuse
% 
% \secrel{Dedication}
% 
% \note{The photo COII/positioll ahol'e ix ajoillt effort. The photo a/Carl II'OS
% taken by his wife Barbara. The photo o/Swall Lake. MOil/alia. a /m'odle spot oj
% Carl's, was takeu byfrieud WaYl1e Arll/s/rOllg. The two images were composed ill
% PIIO/OShopl by artist £Ioille lolli/IS}
% 
% Carl Hammons, my friend and business partner
% for thirty years. died September 11 , 1997 as I
% was writing this book. We shared thousands of
% lunches and coffee breaks over the years we worked
% together, and much of the knowledge I have passed
% on in this book came from Carl. Carl and I shared
% the rare distinction of having been partners not just
% once, but twice. We both played different roles in
% putting together the product line, and without him
% it just isn't going to be as much fun.
% 
% When we joined forces for the second time. we had
% an agreement that eliminated any need to financially
% justify the purchase of a new piece of equipment.
% We would buy machines that interested us and find
% a job for them later. The laser engraver was a
% perfect example of thi s, but now we couldn't get
% along without it. It may seem contrary to smart
% business practice, but that's the way we did it. I have
% no regrets, for we were always the happiest when
% we were confronted with a new set of technical
% problems. Therefore. I dedicate this book to Carl
% Hammons: my business partner, my friend.
% 
% \bigskip
% I should also credit the English teachers in the
% Cranston, Rhode Island school system for forcing a
% not-so-willing student enrolled in the "boys general
% class" to learn enough about our language to dare to
% take on the task of expressing difficult concepts in
% simple words. I graduated in 1953. You, the reader,
% will be the ultimate judge of their (and my) success
% in this undertaking.
% 
% \bigskip\copyright\ Joe Martin
% 
% \secrel{Правила техники безопасности}
% \secrel{Предисловие}
% \secrel{Введение}
% \secrel{Галерея фотографий проектов}
% \secrel{Профиль Мастера\ --- Скотти Хьюитт... 24}
% 
% \secrel{РАЗДЕЛ I\ --- О станочной обработке}
% \secdown
% \secrel{1. Получение информации об обработке...... 27}
% \secrel{2. Вам нужен токарный станок. фрезерный станок или оба ? ........ 29}
% \secrel{3. Материалы для металлообработки... 39}
% \secrel{4. Процессы для металлообработки}
% \secdown
% \secrel{4.1 Термообработка ............ ..  ...... 45}
% \secrel{4.2 Финиширование ...... 46}
% \secrel{4.3 Литье................ 49}
% \secrel{4.4 Другие способы формования металла... .. .. . .. .. 5 1}
% \secrel{4.5 Соединение металлов: пайка и сварка ......... . . .. . .. . 52}
% \secup
% \secrel{5. Использование ручных инструментов и абразивов .......... .. 57}
% \secrel{6. Режущий инструмент для металлообработки}
% \secdown
% \secrel{6. I Общие замечания о вырезании идолов . , 63}
% \secrel{6.2 Режущий инструмент для токарного и фрезерного станков}
% \secrel{6.3 Режущий инструмент для токарных станков  ................. 72}
% \secrel{6.4 Режущий инструмент для фрезеровки ...... ....... .... 81}
% \secup
% \secrel{7. Измерительные инструменты и измерения... 85}
% \secrel{8. Смазочно-охлаждающие жидкости и масла... 99}
% \secrel{9. Основные термины мехобработки .... . 101}
% \secrel{10. Смазка и обслуживание станка .... 107}
% \secup
% \secup
% 
% \secrel{SECTION 2-LATHE OPERATIONS}
% \secdown
% \secrel{Craftsman Profile-Jerry Kieffer ..... .. .. .. . 112}
% \secrel{Jerry Kieffer's Flying Pendulum Clock .. . .. 114}
% \secrel{I. Lathe work holding ... .. . .. . .. .. .. . .. . . 115}
% \secrel{2. Lathe operating instnlclions . . . . ......... 121}
% \secrel{3. Tail slock lools and operations . . . . . ...... 141}
% \secrel{4. Ri ser blocks .. . . . . . . . . . . . . ........... 145}
% \secrel{5. Supporting long or thin work . . ... ... . . .. 149}
% \secrel{6. Gelling started in thread cutting. . . . .. 157}
% \secrel{7. Knurled fini shes. . . . . . . . .. .. . ....... 167}
% \secrel{8. Watchmaking and clockmaking tools . ... . 171}
% \secrel{9. Milling operations on a lathe . . . . .. . . . ... 177}
% \secup
% 
% \secrel{SECTION 3-MIll1NG OPERATIONS}
% \secdown
% \secrel{Craftsman Profile- Augie Hiscano . ........ . 180}
% \secrel{I. Holding parts for milling. . . . . . . . . . . . . 183}
% \secrel{2. Mill operating in structions.. . . . ....... . 191}
% \secrel{3. Squaring up a block ...... ........ .. .... 205}
% \secrel{4. The rotary table and index ing attachment ..... 209}
% \secrel{5. Gears and Geartrains ................... 2 19}
% \secrel{6. Accessories for milling}
% \secrel{• Horizontal milling conversion .... . ..... 235}
% \secrel{• Rotary column attachmcnt . . . . . . .. 237}
% \secup
% 
% \secrel{SECTION 4-OTHER MACHINING TOPICS}
% \secdown
% \secrel{Craft sman Profiles-Dan Lutz and Paul White . 242}
% \secrel{I. Setting up a small workshop .... .... .... . 247}
% \secrel{2. Lathe and mill alignment and adj ustments .. 25 1}
% \secrel{3. Enginering drawings . .... .. ............ 259}
% \secrel{4. Frequently asked questions . . . . .. . . . .. . . 265}
% \secrel{5. Making a bus iness o ut of a hobby .. .Joe Martin's}
% \secrel{and Sherline's story ................... 273}
% \secrel{6. Using CNC in a home shop ... ........... 309}
% \secup
% 
% \secrel{SECTION S-PROJECTS AND RESOURCES}
% \secdown
% \secrel{Cra ftsman Profi le-Bob Bres lauer ........... 3 16}
% \secrel{Machini st's tips .......................... 3 18}
% \secrel{I. Plans and projects you can build ........... 3 19}
% \secrel{1. Miniature Tap Handle ... a beginning}
% \secrel{project you can use in your shop ........... 32 1}
% \secrel{2. Mill vise "sofe jaws. . . . . . . . . . . . . . . . 325}
% \secrel{3. Lay ing out a circul ar hole pattern}
% \secrel{fordrilling, a handy skill to learn. . .. . 327}
% \secrel{4. "Millie" ... a small oscillating steam}
% \secrel{engine by Ed Warren, a simple project}
% \secrel{from the pages of Modeltec magazine) ..... . 329}
% \secrel{5. Ordering plans for the Little Ange l}
% \secrel{hit 'n miss engine ... an advanced}
% \secrel{machining project by Bob Shores .......... 333}
% \secrel{2. Contests and awards for tabletop machini sts . . 335}
% \secrel{3. Exploded views and part number li sting}
% \secrel{ Model 4000 and 4400 Lathes .. ......... 340}
% \secrel{ Model 5000 and 5400 Milling machines . . . 341}
% \secrel{ Model 2000 8-Direction Mill Column ..... 342}
% \secrel{ Part number listing ... . ... . ........ . .. 343}
% \secrel{4. A simple RPM gage for your latheor mill ... . 345}
% \secrel{Harold Cli sby and the first Sherline lathe .... . 346}
% \secrel{S. Index ............. .... ...... . . . 347}
% \secrel{ Conversion faclors .. . ... . . .. .. ........ 350}
% \secup

\secup
